\documentclass[11pt]{article}
\usepackage{amsmath, amsthm, amsbsy, rotating,float}
\usepackage{graphicx,algorithm,algorithmic}
\usepackage{setspace,enumerate}
\usepackage[margin=1.2in]{geometry}
\usepackage{color}


\usepackage{pdflscape}
\renewcommand{\arraystretch}{.5}

\usepackage{graphicx,caption,subfig}

\floatstyle{ruled}
\newcommand{\norm}[1]{\left\lVert#1\right\rVert_p}
\newcommand{\normi}[1]{\left\lVert#1\right\rVert_\infty}
\newcommand{\normt}[1]{\left\lVert#1\right\rVert_2}
\newcommand{\red}[1]{{\textcolor{red}{#1}}}
\doublespacing

\author{Esteban D\'{i}az}
\title{Public speaking: the good, the bad, and the road ahead}{}

\begin{document}
%Compile a list of at least 5 habits of speakers that you find pleasing and productive,
%and a list of at least 5 habits that are ineffective and irritating. Write a plan with at least
%five action items for how you will train yourself to acquire the good habits and avoid the bad
%ones. Make sure the actions items are SMART (Specific, Measurable, Actionable, Realistic,
%and Time-bound). Hand this in on the last day of class.
%

\maketitle
Public speaking is probably the most effective way to convey to our community what we do. 
 Being such an important tool, we must force ourselves to constantly improve in 
our speaking habits. The five speaking habits I admire the most, which I try to constantly
improve on are:
\begin{itemize}
  \item Be the center of the presentation: I really admire those speakers who don't look
        at their slides, and as they speak the slides are changing in the background seamlessly. 

  \item Good supporting material: this includes to have good graphics with legible fonts. 

  \item I think is great when speakers tell upfront what they do, how they do it and what is 
        the main outcome. Then, the talk is about the intermediate details.
  
  \item Engaging the audience: the speaker is inclusive with 
        the audience. This can only happen if the speaker thinks very good  
        about his/her audience. 

  \item Analysis of the audience: each audience responds differently to a given 
        talk. Hence, it makes sense to modify a presentation accordantly. 

\end{itemize}

There are many speaking habits that are unpleasant, distractive, and unproductive. My top
5 list of them is the following:
\begin{itemize}
  \item Not taking seriously the presentation: going overtime and not rehearsing. 
  \item Illegible graphs: if the speaker is supporting his talk with graphic 
        material, then it must be legible.
  \item Too much information on slides: this can be very distracting. 
  \item Monotone speech: this can make a presentation quite boring. 
  \item Reading from slides. 
\end{itemize}

How we communicate with our peers and the community is a great part of our professional growing. 
 Whatever we can do to improve ourselves in this aspect of our skills should be a priority. 
 I would like to improve several characteristics of my public speech. Particularly, I am working
on these ones:

\begin{itemize}
  \item Improve my confidence as speaker: I feel I have improved over the years on this one. 
        I want to show that I am not nervous even if I am. For me, the best way to fight against
        my nerves is to practice, practice, and practice some more. With each presentation in front 
        of a big audience, I feel my stage fright retracts a little bit. 

  \item Be the center of the presentation. I constantly try to be the main information carrier. 
        To do so, my supporting graphic material must be minimal and yet be helpful enough
        to give me a clear roadmap while I speak.

  \item Think carefully about what to say: this is very important to me since I am a non-native
        English speaker. As such there are phrases and words that I usually practice in advance
        to help the audience against my sometimes heavy accent. Diane Witters have been very helpful
        to me in this regard. 

  \item Make eye contact: I find hard to make eye contact with the audience while I speak. Sometimes,
        I have ``lost look'' or I look to the ground. The most effective tool I have found to fight against
        this habit is to find friendly people in the audience and talk to them. 

  \item Simplicity: I like simple talks, making a very complex subject sound easy. It is hard to abstract
        yourself from your work, but as a speaker you must do so. I think the audience appreciates 
        this kind of talks. I constantly think about how to convey a message, an idea, or a result 
        as simply as possible.  
\end{itemize}




\end{document}
