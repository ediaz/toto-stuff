\documentclass[11pt]{article}
\usepackage[utf8]{inputenc}
\usepackage{amsmath, amsthm, amsbsy, rotating,float}
\usepackage{graphicx,algorithm,algorithmic}
\usepackage{setspace,enumerate}
\usepackage[margin=1.2in]{geometry}
\usepackage{color}


\usepackage{pdflscape}
\renewcommand{\arraystretch}{.5}

\usepackage{graphicx,caption,subfig}

\floatstyle{ruled}
\newcommand{\norm}[1]{\left\lVert#1\right\rVert_p}
\newcommand{\normi}[1]{\left\lVert#1\right\rVert_\infty}
\newcommand{\normt}[1]{\left\lVert#1\right\rVert_2}
\newcommand{\red}[1]{{\textcolor{red}{#1}}}
\doublespacing

\author{Esteban D\'{i}az}
\title{What Makes a Scientist?}{}

\begin{document}

\maketitle

I interviewed Carla Carvajal, who recently obtained a MSc degree at CSM, 
Satyan Singh (3rd year PhD student), Pauline Kessouri post-doc in the Geophysics Department, 
and Professor Miguel Bosch. I tried to have a varied range of academic positions, and 
also a good gender mix. I asked several questions to motivate some general discussion
subjects. Here, I try to summarize the outcome of our discussions. 


\section{Experience background}
Carla Carvajal earned her BSc degree in Geophysical Engineering in 2008 from  the Universidad Simón Bolívar, then 
she worked for 3 years in seismic processing and seismic imaging R\&D. In 2011 she started
her MSc. degree in Geophysics at CSM. She graduated in 2013 and later started working at
Noble Energy. 

Satyan Singh earned his BSc degree in Petroleum Geosciences in  2008 from the University Of The West Indies, then 
he worked for one year at BG and decided to pursue a MSc degree at Texas A\&M University. In 2011 he finished
and came to CSM to pursue a PhD degree in Geophysics. He is now on his 3rd year. 

Pauline Kessouri did her studies in Paris were she earned a combined engineering and master degree in Geophysics,
 she also hold a master degree in archeology. She then continued to her PhD, which she finished 6 months ago. 
She is a post-doc in the Geophysics Department at CSM where she continuing her research on electro-magnetic phenomena. 

Miguel Bosch earned his BSc in Physics in 1989 in Universidad Simón Bolívar, later he obtained a MSc 
degree in Geophyiscs in Universidad Central de Venezuela, and a PhD in Geophysics at the Institute de Physique du Globe 
de Paris. He held a post-doc position as visitin fellow at Cambridge University in 2000-2001. He is a Professor 
in the Applied Physics Department at Universidad Central de Venezuela. He also has a company that offers
consultancy projects on reservoir Geophysics. 

\section{Research and career paths}
In her undergraduate thesis, Carla did research on stochastic post-stack seismic inversion. Her work involved great 
amount of time, research, and code implementation. That was her first experience with research. 
Her thesis subject was suggested by her advisor. 
 Then, during her seismic processing job she was away from research. Later, she had R\&D position as
seismic imaging researcher where she could experience a completely different subject. Then, at RCP she changed
subjects again, this time to reservoir characterization. Different from her undergrad thesis, at CSM she was
able to explore different subjects and decide which one to choose. At Noble Energy she also works with reservoir 
characterization. Research wise her career path has been quite interesting, she had the ability to change 
subjects and perform well on all of them. She says that her experience was extremely beneficial prior 
returning to academia for her MSc degree.

Satyan also has a very interesting research path, in his undergrad thesis he did geology play analysis. Then,
at Texas A\&M he did modeling and analysis of borehole waves. At CWP, Satyan started doing some velocity analysis
for diffracted waves, and later switched gears into a very ambitious and interesting seismic imaging project. 
 He has demonstrated to be a very effective researcher, with the ability of understanding and mastering 
each of his projects while producing very good outcomes. 


Pauline specialized in Geophysics during her undergrad degree. Then, she became also interested in Archeology, 
reason why she also has a master's degree on the subject. Her interest in Archeology and Geophysics
seem a good mix, and her idea was to apply geophysical methods to archeology investigations. That was
the main driver for her PhD, where she developed a device to measure a certain range of electro-magnetic 
frequencies. Even though the initial idea was to apply the measurements of this new device to archeology 
research, this changed along the way during her PhD. 

Professor Bosch has been always interested in geophysical inversion, subject that he has pursued all his 
career. He says his first project for his undergrad thesis was suggested from his advisor. Luckily enough
he really liked the subject and expanded on it.
 While some researchers choose a narrow path, he kept his research inside the very big cap 
of ``geophysical inversion''. Particularly, he focus on inversion theory under statistical and 
prior constraints, either with deterministic or stochastic methods. 

\section{Work-life balance}
Carla and Pauline expressed her point of view about the work-life balance in academia. They were particularly
concerned about the conflict between the role of women in a family with the workload requirements. Both
feel that academia although very demanding, it gives a lot of flexibility with the schedule.
 Pauline would like to get a Professor position in the near future. The road to this position is extremely
demanding, and the pressure to publish papers is quite high. But she enjoys a lot what she do. She thinks in the
future, with children she will have to better balance her academic and personal life. 

Miguel says that having the support from his wife was extremely important during is PhD studies and the tenure
track period for his professor position. He managed to do a lot of work and concurrent research projects while spending
quality time with his family. He says that organizing himself with time-managing was very important, he is 
very methodical about this. 


\section{Communication in the academic environment}
 Carla, Pauline, and Satyan are from the times with a lot of ways for communicating. Yet, still in our times
the preferred way is face to face meetings. They all use email as an effective way for exchanging 
ideas. 

Pauline had two advisors with very different personalities, interests, and approaches to do 
research. The communication with one of her advisors was not very fluent, while the other advisor had 
very political ways of addressing her project which involved collaboration with different departments. 

Carla's experience with her undergrad advisor was very fluent, mostly consisted of bi-weekly meetings. Later
on her MSc she had to perform more independently and relied on scattered meeting with her advisor. 

Satyan had interacted with 4 advisors,  two during his MSc and two during his PhD. Probably he has 
experience every advising style ranging from very close superposition to little supervision. I think his experience
as advisee have made him very flexible and easy to work with. Now as 3rd year student he has a lot of freedom
for his research. 

Miguel's times were different during his master and PhD studies. He had meetings with his advisor, and sometimes
 he had to exchange letters. Then with the internet, that changed a lot how he communicates with his students. 
 Although he tries to have a face to face interaction, due to his travels he often uses video calls 
with his students. 


\end{document}
