\documentclass[11pt]{article}
\usepackage{amsmath, amsthm, amsbsy, rotating,float}
\usepackage{graphicx,algorithm,algorithmic}
\usepackage{setspace,enumerate}
\usepackage[margin=1.2in]{geometry}
\usepackage{color}


\usepackage{pdflscape}
\renewcommand{\arraystretch}{.5}

\usepackage{graphicx,caption,subfig}

\floatstyle{ruled}
\newcommand{\norm}[1]{\left\lVert#1\right\rVert_p}
\newcommand{\normi}[1]{\left\lVert#1\right\rVert_\infty}
\newcommand{\normt}[1]{\left\lVert#1\right\rVert_2}
\newcommand{\red}[1]{{\textcolor{red}{#1}}}
\doublespacing

\author{Esteban D\'{i}az}
\title{Bibliography managing}{}

\begin{document}
%Read chapters 9 and 11 of the book. Do some research to choose a database for
%archiving references, and choose a database program that you start using. Add references to
%it for the remainder of the semester. Write a brief report in which you explain which data
%base program you have started using and why you made this choice. Describe your experience
%in using this program. Document this all in a brief report, maximum 1 page, and make sure
%I have your report by May 2nd. N.B. remember that a spreadsheet is not a database, and
%that a spreadsheet does not connect with word processing software to create a bibliography.
%
\maketitle

After some online reviewing I chose ``Bibdesk'' as my bibliography database manager. 
 ``Bibdesk'' is devoted to the MacOS system, so it is probably the most native software 
for my computer. 

I really liked the simplicity of its interface. Similarly to Jabref, the program that Roel demo
in class, it allows for simple search based on keywords. It also contains a tree to the left of the 
main screen which organizes the database based on keywords. I realized that my bibtex files
 did not have many keywords. I updated most of the references, so now I can search by either
a particular subject, author, or keyword. I can also add notes to the reference, which makes 
quite efficient the reference review after a long period of time without reading the
article. I used to keep summaries of papers in a separate folder. Now, I can have my summaries
and references in the same place. 

I think this software (or any other bibliography manager) is way more efficient than how
I use to do it: by manually editing a pattern searching a text file with my favorite editor, VI. 

Another good feature that I haven't explore in detail is the pdf document manager. You can add a local
file, or a url link to the reference. This allows the user not only to have access to the reference, 
but also to the article itself. This and the previously mentioned features makes the database quite interactive. 

\end{document}
