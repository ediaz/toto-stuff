
\author{Esteban D\'{i}az}
\title{Improving the FWI gradient through data selection in the frequency domain}

\maketitle

In class we have seen that the gradient formulation in FWI relies on the 
Born approximation. This assumption imposes serious limitations on the velocity
models we can recover. As rule of thumb, the Born assumption is said to be broken
when the travel time misfit between observed and modeled data is bigger than
half a period, $\Delta t > T/2$. However, despite how big is the velocity perturbation to recover,
 the travel time misfit might be within the half-cycle for some distances. 
This observation have been used by several authors  \citep{liu2014,alejo} as a QC for FWI. The 
basic idea in the data selection is to back-propagate the parts of the data that are
compliant with the Born approximation. As the model improves, the mask relax more and more, 
and at the end probably most of the data can be included.

In the frequency domain the half-cycle assumption translates in a phase difference bigger 
than $\Delta \phi > \pi$. This observation was used \cite{qa-fwi}
to QC the initial velocity models before inversion starts. They say not to start a FWI 
project if this Born conditions are not met.

In this report, I use the same criterion to identify cycle-skipped phases as \cite{qa-fwi}
does. Whereas he does an interpretation, I implement an automatic approach for identifying the 
regions, and use it to create a mute zone mask. 
  I implement two maskig approaches.
 In the first one, I use the magnitude of the phase difference gradient $\nabla \Delta \phi$
to create a mute based on the limit of the data one can use for the current model. The 
gradient is defined in the shot-receiver domain. 

In the second approach, I convert Shot-Receiver data to Shot-Offset data, then I calculate 
the magnitude as: $\left | \partial_h \Delta \phi \right |$. 
Both masking approaches should be the frequency-domain equivalent of \cite{liu2014,alejo}.

\section{Results}
 Figures~\ref{fig:truemod}-\ref{fig:vdiff} show the true velocity model, the starting model,
and the difference, respectively. Figure \ref{fig:vdiff} is the ``perfect'' update direction,
 which I use as benchmark for the different results. 

Figure~\ref{fig:phase-diff} shows the phase difference between modeled and 
observed data, measured as: $\Delta \phi = arg(d_{mod}d_{obs}^*) \in [-\pi,\pi]$ in the
Shot-Receiver domain \citep{qa-fwi}.
 Figure~\ref{fig:mag} shows the magnitude of the gradient of the phase difference, where 
the gradient is defined with the source and receiver positions as follows:
\[
\nabla \Delta \phi := \partial_S\Delta\phi  \hat{S} + \partial_R \Delta\phi \hat{R}.
\]
To compute the inclusion mask, I do it automatically by setting it to 1 along the Receiver
axis until I find a spike, then is 0. Then, I smooth the mask so the inclusion/rejection region
varies smoothly. The mask is shown in Figure~\ref{fig:mask}.  

In the second masking approach (which came from discussions during the presentation), I transform
the S-R data to S-h mapping. Figure~\ref{fig:phase-diff-sh} shows the phase difference mapped
in S-h domain. Figure~\ref{fig:magx} shows the cycle-skip attribute. Figures~\ref{fig:grad-magx-mask-sh}-
\ref{fig:grad-magx-mask-sr} show the second mask in the S-h and S-R domains, respectively. 


Figure~\ref{fig:gr-ddiff} shows the gradient computed using the direct difference as 
adjoint source. Figures~\ref{fig:gr-mddiff} and \ref{fig:gr-m2ddiff}  show
 the gradient using the data selection approach proposed here for masks 1 and 2, respectively. 
Figures~\ref{fig:gr-sm-ddiff}, \ref{fig:gr-sm-mddiff}, and \ref{fig:gr-sm-m2ddiff}
 show the smoothed versions for direct difference and data-selection approaches, respectively. 

Which one is better? Our benchmark is Figure~\ref{fig:vdiff}. Without smoothing
it is hard to appreciate the direction of update, this is probably due to the ringing that
comes from the fact we are using only one frequency for the gradient $f=5Hz$. 

Figure~\ref{fig:gr-sm-mddiff} is much better for the left side 
of the model. Also, is better in the upper faulted part of the model (blue
update), whereas the conventional update extends the blue update further 
to the left. 


The approach for automatic masking can be easily extended to 3D, in this case the 
magnitude gradient mask would be calculated along surface coordinates. For streamer
data, the gradient could be calculated along the shot position and offset axis. 


\begin{figure}
\centering
  \subfloat[]{
    \includegraphics[width=0.5\textwidth]{hw03/Fig/true.pdf} 
    \label{fig:truemod}}

  \subfloat[]{
    \includegraphics[width=0.5\textwidth]{hw03/Fig/starting.pdf}
    \label{fig:starting}} 

  \subfloat[]{
    \includegraphics[width=0.5\textwidth]{hw03/Fig/diff.pdf}
    \label{fig:vdiff}}

  \subfloat[]{\hspace*{.91cm}\includegraphics[width=0.47\textwidth]{hw03/Fig/bar.pdf}}
  \caption{True Marmousi model (a), starting velocity model (b), difference between starting and true model
 (c),  and colorbar for velocity models (d).}
\label{fig:d2}
\end{figure}

\begin{figure}
\centering
  \subfloat[]{
  \includegraphics[width=0.3\textwidth]{hw03/Fig/phase-diff-res.pdf} 
  \label{fig:phase-diff}
  }
  \subfloat[]{
  \includegraphics[width=0.3\textwidth]{hw03/Fig/mag.pdf} 
  \label{fig:mag}
  }
  \subfloat[]{
  \includegraphics[width=0.3\textwidth]{hw03/Fig/grad-mag-mask.pdf} 
  \label{fig:mask}
  }
  \caption{Phase angle difference (a), magnitude of the gradient of the phase (cycle-skip indicator) (b) and the mute filter (c) based on attribute (b).}
\end{figure}


\begin{figure}
\centering
  \subfloat[]{
  \includegraphics[width=0.5\textwidth]{hw03/Fig/phase-map.pdf}   
  \label{fig:phase-diff-sh}
  }
  \subfloat[]{
  \includegraphics[width=0.5\textwidth]{hw03/Fig/magx.pdf}  
  \label{fig:magx}
  }

  \subfloat[]{
  \includegraphics[width=0.5\textwidth]{hw03/Fig/grad-magx-mask-sh.pdf}  
  \label{fig:grad-magx-mask-sh}
  }
  \subfloat[]{
  \includegraphics[width=0.3\textwidth]{hw03/Fig/grad-magx-mask-sr.pdf} 
  \label{fig:grad-magx-mask-sr}
  }

  \caption{Phase angle difference (a), $\left|\partial_h \Delta \phi \right |$ (cycle-skip indicator) (b) and the mute filter in 
shot-offset domain (c), and the same mask in the S-R domain (compare with Figure~\ref{fig:mask}.}
\end{figure}



\begin{figure}
\centering
  \subfloat[]{
    \includegraphics[width=0.3\textwidth]{hw03/Fig/dmod.pdf} 
    \label{fig:dmod}}
  \subfloat[]{
    \includegraphics[width=0.3\textwidth]{hw03/Fig/data.pdf} 
    \label{fig:dobs}}

  \subfloat[]{
    \includegraphics[width=0.3\textwidth]{hw03/Fig/data-diff-res.pdf} 
    \label{fig:ddiff}}
  \subfloat[]{
    \includegraphics[width=0.3\textwidth]{hw03/Fig/masked-data-diff-res.pdf} 
    \label{fig:mddiff}}

  \caption{Real part of modeled data (a), data from true model (b), data difference (c),
   and the masked data residuals (d).}
\end{figure}

\begin{landscape}
\begin{figure}
\centering
  \subfloat[]{
    \includegraphics[width=0.5\textwidth]{hw03/Fig/grad-data-diff-res.pdf} 
    \label{fig:gr-ddiff}}
  \subfloat[]{
    \includegraphics[width=0.5\textwidth]{hw03/Fig/grad-sm-data-diff-res.pdf} 
    \label{fig:gr-sm-ddiff}}

  \subfloat[]{
    \includegraphics[width=0.5\textwidth]{hw03/Fig/grad-masked-data-diff-res.pdf} 
    \label{fig:gr-mddiff}}
  \subfloat[]{
    \includegraphics[width=0.5\textwidth]{hw03/Fig/grad-sm-masked-data-diff-res.pdf} 
    \label{fig:gr-sm-mddiff}}

  \subfloat[]{
    \includegraphics[width=0.5\textwidth]{hw03/Fig/grad-masked2-data-diff-res.pdf} 
    \label{fig:gr-m2ddiff}}
  \subfloat[]{
    \includegraphics[width=0.5\textwidth]{hw03/Fig/grad-sm-masked2-data-diff-res.pdf} 
    \label{fig:gr-sm-m2ddiff}}
\caption{ Original gradient (a), smoothed original gradient (b), proposed gradient with mask1 (c), smoothed proposed gradient with mask1 (d),
        proposed gradient with mask2 (e), and smoothed proposed gradient with mask2 (f). All gradients should be compared with Figure~\ref{fig:vdiff}. All gradient plots have been clipped together.}

\end{figure}
\end{landscape}


\bibliographystyle{plain}
\bibliography{biblio} 

