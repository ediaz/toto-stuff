
\author{Esteban D\'{i}az}
\title{Limits of the Born approximation}{a simple example}

\inputdir{hw01}

In this simple model, we have two arrivals: direct, and reflected from a deeper
velocity discontinuity. For computing the kernels, we use a homogeneous background model
which we perturb by some scalar $s$. Hence, the model is only capable of 
creating the direct arrival. 

In order for us to create meaningful updates we need our modeled and observed
data to be within half-cycle $T$. The peak frequency  of our source wavelet is $f_p=10Hz$, hence our most representative period is $T=0.1s$. The distance between 
source and receivers is 4km, which means that our observed direct arrival is at 2s. 
For us to break the Born approximation (produce data shifts beyond half-cycle) we need 
 a background velocity model that produces arrivals before 1.95s or after 2.05s:
\[
 \frac{4km}{2.05s} = 1.9512km/s = v_{min},
\] 
and
\[
 \frac{4km}{2.05s} = 2.0512km/s = v_{max}.
\] 
Any perturbation outside those bounds will result on meaningless updates. What we
will see for either case beyond the limits is the auto-correlation of the 
modeled data. This means that we loose the sense of direction of the update. 
 Figure 2 shows the modeled data for different scalar perturbations outside
the velocity bounds. Figure 3 shows the corresponding sensitivity kernels for
such perturbations. As predicted, all the gradients show the same (meaningless) direction of
update.

For comparison, Figure 4 shows 2 gradients within the meaningful limits. Note that
the contribution of the reflected arrival did not change, only the contribution from
direct arrival did. 


\plot{dat}{width=0.5\textwidth}{observed data}



\multiplot{7}{dbk-085,dbk-090,dbk-095,dbk-100,dbk-105,dbk-110,dbk-115}%
{width=0.35\textwidth}{Modeled data for perturbations ranging from 85\% (a) to
115\%(g) in 5\% intervals}




\multiplot{7}{grad-085,grad-090,grad-095,grad-100,grad-105,grad-110,grad-115}%
{width=0.3\textwidth}{Sensitivity kernels for perturbations ranging from 85\% (a) to
115\%(g) in 5\% intervals}


\multiplot{2}{grad-099,grad-101}
{width=0.3\textwidth}{Sensitivity kernels for 99\% (a) and 101\% (b). Note the change
of direction} 

