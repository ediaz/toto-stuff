%\section{Theory review}
\section{Image domain tomography}

In this section, we review the image-domain wavefield tomography 
 using extended-image gathers~\citep{rickett:883,SavaVasconselos}. These kind of gathers
highlight the spatial and temporal consistency between wavefields by exploring
 the focusing information in the image domain. 
 The moveout in the gather is sensitive to
velocity perturbations, and hence can be optimized. The most
general extended gather can be defined as follows:
\beq
R(\xx,\hh,\tau)= \sum_{{e}} \sum_{t} \US({e},\xx - \hh,t-\tau) \UR({e},\xx+\hh,t+\tau),
\label{eq:eic}
\eeq
 where $\hh$ is the space-lag vector, $\tau$ is the time-lag, $\xx$ the image location, $e$ the
experiment index, $\US$ the 
 source wavefield, and $\UR$ the receiver wavefield. Given the increase of dimensions 
for extensions in all directions and time, one can take advantage of a sparse sampling 
of the extended images in a subset $\xx_c$ instead of the full image space $\xx$. One smart
way to  decide the locations $\xx_c$ is by placing the observation points at the reflector
locations \citep{cullison}.  
The source wavefield $\US$
is produced by forward extrapolation of the source function, whereas
the receiver wavefield $\UR$ is produced by backward propagation of the data
at the receiver location. Note that the process for computing the extended
image is linear with respect to one of the wavefields. Hence, we can 
define the extended image in a matrix vector form as: 
\beq
  {\bf r} = {\bf I}_{s} {\bf u}_r = {\bf I}_{r} {\bf u}_s,
\label{eq:eicops}
\eeq
where ${\bf I}_s$ and ${\bf I}_r$ are the imaging operators for the source and receiver wavefield,
respectively. 

Eventhough the commutation of the forward mappings (Equation~\ref{eq:eicops})
 produces the same result ($\bf r$), the adjoint 
mapping satisfies different equations. ${\bf I}_s^\top$ implements:
\beq
  \tilde{\UR}(e,\xx,t) = \sum_{\tau,\hh}R(\xx-\hh,\hh,\tau) \US(e,\xx - 2\hh,t-2\tau),
\eeq
whereas ${\bf I}_r^\top$ satisfies:
\beq
  \tilde{\US}(e,\xx,t) = \sum_{\tau,\hh}R(\xx+\hh,\hh,\tau) \UR(e,\xx + 2\hh,t+2\tau).
\eeq

The source and receiver wavefields follow the wave equation, which in our case
is 
\beq
  \swe{u(\xx,t)} = f(\xx,t),
\eeq
where $\m=s^2(\xx)$ is the squared slowness. The source wavefield $\US$ is computed by forward
progration problem, whereas the receiver wavefield $\UR$ is created by backward (adjoint)
modeling. This can be summarized by the following system of equations
\beq \label{eqn:FSV}
\MAT{ \L(m,t) & 0 \\ 0  & \L^\top(m,t)}
\MAT{ \us \\ \ur} =
\MAT{ \fs \\ \fr} ,
\eeq
with $\fs$ being the source function and $\fr$ the recorded seismic data, and $\L$ and $\L^\top$ 
the forward and adjoint (reverse) propagators, respectively. 

The focusing  information in extended images can 
tell a great deal about the velocity model. In a kinematically-consistent 
model, the correlation between source and receiver wavefields is 
maximum around $(\hh,\tau)=(\vec{0},0)$. Under velocity innacuracies, the
extended gathers exhibit a moveout that is indicative of 
 errors in the model~\citep{YangSava:moveout}. Hence,
 the energy on the gathers can be brought around zero lag through an
inversion process in which the velocity model can be updated. A typical 
objective function \citep{ShenSymes.geo.2008,Wiktor,tony:gp15} based on 
extended images resembles 
\beq
 J_1(\m) = \norm{P(\hh,\tau) R(\xx,\hh,\tau)}^2,
\eeq
where $P$ is a penalty operator whose purpose is to 
highlight the events in the image that show velocity innacuracies. 
 The penalty operator is usually defined proportional 
to the lag:
\beq
  P(\hh,\tau) = \sqrt{ \hh^2 +(v\tau)^2},
\eeq
where $v$ is the local velocity. This penalty function is 
highly idealized since produces a zero residual if the 
extended image is a perfect spike (i.e. $R(\xx,\hh,\tau)=\delta(\xx,\hh,\tau)$).
 In order to relax such requirements, \cite{tony_seg:cwp12} propose
a penalty function based on illumination, which depends on the 
acquisition setup, data bandwidth, and on the velocity model itself. 
 To compute the action of the illumination (bluring  
operator) one can cascade a demigration/migration process 
applied to spikes placed at the extended image locations $\xx_c$:
\beq
  \tilde{R}(\xx_c,\hh,\tau) = {\bf M} {\bf M}^\top \sum_{i} \delta(\xx-\xx_c^i,\hh,\tau)
\eeq
with $\bf M^T$ being a demigration operator and $\bf M$ the extended imaging process. 
 The synthetic image $\tilde{R}$ contains the action of the source function, the acquisition, 
and the model at the image locations $\xx_c$. Once $\tilde{R}$ is obtained, one can 
proceed and compute the penalty by 
\beq
  \tilde{P}(\xx_c,\hh,\tau) = \frac{1}{Env(\tilde{R}(\xx_c,\hh,\tau))},
\eeq
where $Env$ is an envelope function that captures the energy of the synthetic  
image $\tilde{R}$.
Effectively, $\tilde{R}$ contains the extended point spread functions (PSF) of the model, 
and is computed similarly to \cite{valenciano:2009} and \cite{FletcherLSM}.

\rfgs{gceimg-v100},\ref{fig:gceimg-v090} show a CIP gather extracted in the 
correct model and in a slow model, respectively. One can observe the moveout in the gather for the
incorrect model, and also the best posible focusing one can achieve in this model from the \rfg{gceimg-v100}. 
Finally, \rfgs{pDSO-cip},\ref{fig:pILL-cip} show the penalty functions using the conventional
method and the illumination-based one, respectively.

In order to include the zero-lag information one can add and extra term in the objective 
function, similarly to \citep{ShenSymes.geo.2008,Wiktor}. Alternatively, one can 
include the zero-lag term directly into the inversion by changing the goal 
of the objective function to a maximation problem~\citep{Zhang}:
\beq
 J_2(\m) = -\norm{H(\hh,\tau) R(\xx,\hh,\tau)}^2.
\label{eq:j2}
\eeq
By adding the negative in front of the norm, we turn the problem back to minimization. The
operator $H$ has an opposite purpose than operator $P$, i.e. $H = 1-P$ , 
now its main task is to ``highlight'' the energy around zero lag. By using $J_2$ instead of 
$J_1$ we include all the lags in a seamless manner into the inversion. 
%%
\inputdir{ximages}
\multiplot{4}{gceimg-v100,gceimg-v090,pDSO-cip,pILL-cip}{height=0.2\textheight}%
{(a) is an example of a CIP gather from the correct model, (b)
 a CIP gather with 10\% velocity error,  (c) conventional penalty
function, and (d) illumination-based penalty. Note the consistency between the penalty functions
and the correct gather. }

\subsection{tomography setup}
We are interested in the gradient of Equation \ref{eq:j2}. In order to compute the gradient 
we follow the recipe from \cite{Plessix.gji2006.asm} for PDE constrained 
objective functions. First, we define the augmented functional
\beq
  A(\m,\us,\ur,a_s,a_r) = -J_2(\m) + < a^e_s,\L\us^e-\fs> +<a_r,\L^\top\ur^e-\fr>,
\label{eq:aug1}
\eeq
where $<.,.>$ defines a dot product between vectors in  the $(\xx,t)$ space, $a_s$ is the 
adjoint source wavefield, and $a_r$ is the adjoint receiver wavefield. The 
augmented functional depends on the model $\m$, the forward state variables
$\us$ and $\ur$, and the adjoint state variables $a_s$ and $a_r$.
 The first optimality condition, $\frac{\partial A }{\partial a_s}=0$, 
 produces the first equation to solve: $\L\us = \fs$. Similarly, 
by setting $\frac{\partial A }{\partial a_r}=0$ we obtain $\L\ur = \fr$. 

In order to find the adjoint source wavefield $a_s$, we can rewrite Equation \ref{eq:aug1} as:
\beq
    A(\m,\us,\ur,a_s,a_r) = - < H I_r \us, H I_r \us> +< a_s,\L\us-\fs> +<a_r,\L^\top\ur-\fr>.
\eeq
By setting $\frac{\partial A }{\partial \us}=0$ we obtain:
\beq
   \L^\top a_s = I_r^\top H^\top H I_r \us =  I_r^\top H^\top H R. 
\eeq
 The operator $H$ is diagonal and real, which means $H^\top=H$. This equation tell us to 
use the receiver wavefield $\ur$ as imaging operator together with the image residual $H^\top H R$.
 The wavefield $a_s(e,\xx,t)$ travels from the image towards the source $\fs$. 

To find the adjoint the adjoint receiver wavefield $a_r$ we redefine  Equation \ref{eq:aug1}  to 
the equivalent form
\beq
    A(\m,\us,\ur,a_s,a_r) = - < H I_s \ur, H I_s \ur> +< a_s,\L\us-\fs> +<a_r,\L^\top\ur-\fr>.
\eeq
Now, we set $\frac{\partial A }{\partial \ur}=0$ to obtain the last adjoint variable $a_r$:
\beq
   \L a_r = I_s^\top H^\top H R.
\eeq

From the last optimality condition we obtain the actual gradient:
\beq
  \frac{\partial J_2 }{\partial \m}=  -\sum_{t,e}  \frac{\partial^2 \us}{\partial t^2}a_s + \frac{\partial^2 \ur}{\partial t^2}a_r
\eeq

\inputdir{gauss_images}
\multiplot{4}{model-gauss_cip_illu-iter-010,image-gauss_cip_illu-iter-010,model-gauss_cip-iter-018,image-gauss_cip-iter-018}{width=0.45\textwidth}%
{(a) recovered model using illumination-based penalty and (b) its corresponding image, (c) inverted model using conventional DSO penalty and (d) its corresponding image}


%%%%%%%%%%%%%%%%%%%%%%%%%%%%%%%%%%%%%%%%%%%%%%%%%%%%%%%%%%%%%%%%%%%%%%%%%%%%%%%%%%%%%%%%%%%%%%%%
%%%%%%%%%%%%%%%%%%%%%%%%%%%%%%%%%%%%%%%%%%%%%%%%%%%%%%%%%%%%%%%%%%%%%%%%%%%%%%%%%%%%%%%%%%%%%%%%
%%%%%%%%%                            Data domain block                              %%%%%%%%%%%% 
%%%%%%%%%%%%%%%%%%%%%%%%%%%%%%%%%%%%%%%%%%%%%%%%%%%%%%%%%%%%%%%%%%%%%%%%%%%%%%%%%%%%%%%%%%%%%%%%
%%%%%%%%%%%%%%%%%%%%%%%%%%%%%%%%%%%%%%%%%%%%%%%%%%%%%%%%%%%%%%%%%%%%%%%%%%%%%%%%%%%%%%%%%%%%%%%%

\section{Data domain Wavefield tomography}
The construction of the tomography problem in the data domain amounts to measuring the error
(or residual) at the receiver locations. For data domain wavefield tomography, we normally
 use the data difference for the residual:
\beq
J = \frac{1}{2} \norm{\US(\xx_r,\Omega)-\fr(\xx_r,\Omega)}^2=\frac{1}{2} \norm{\Delta d}^2,
\label{eq:offwi}
\eeq
where $\xx_r$ are the receiver locations and $\Omega$ is the complex valued frequency whose
purpose we will explain later. Note that $\fr(\xx_r,\Omega) = \UR(\xx_r,\Omega)$.
Since for building the residual we only need to forward propagate the source function 
\beq
\L({m},\Omega)\US = \fs(\xx_s,\Omega),
\eeq
 we have to compute one adjoint wavefield $\as(\xx,\Omega)$.
For data-domain wavefield tomography, computing $\as$ involves backpropagating the data residual:
\beq
  \L^\top({m},\Omega) \as = (\Delta d)^{*}.
\eeq
Here $(\Delta d)^{*}$ is the complex conjugate residual and $\L({m},\Omega)$ is the acoustic wave 
equation in the frequency domain, defined as follows:
\beq
  \L(m) = -\DIV{\frac{1}{\rho(\xx)}\GRAD{}} -\frac{m(\xx)}{\rho(\xx)}\Omega^2.
\label{eq:time_we}
\eeq
Here $\rho(\xx)$ is the density of the medium. In this report, we do not invert for $\rho(\xx)$, instead
we parametrize it as a function of the velocity following \cite{Gardner}.

Once we obtain $\US(\xx,\Omega)$ and $a_s(\xx,\Omega)$, we can proceed to compute the gradient:
\beq
\nabla J(\xx) = \mathbb{R}\left\{\sum_{e} \sum_t \Omega^2{u}_s({e},\xx,\Omega)a_s^*({e},\xx,\Omega)\right\},
\label{eq:graddwt}
\eeq
here $^*$ denotes complex conjugate and $\mathbb{R}\left\{\right\}$ denotes the real part.

The data domain wavefield tomography objective function, equation \ref{eq:offwi}, is highly non-linear if
we operate with signals in the normal frequency band. Hence, 
in order to increase the chances of convergence to the global minimum,
 it is customary to implement the data-domain wavefield tomography in a multi-scale fashion. \cite{Bunks95} propose to 
first invert lower frequencies and then 
move gradually to higher frequencies. The idea is that within each scale the problem looks more linear than
when inverts all the bandwidth at once. 

An additional outer loop in the inversion is the time damping, which leads to the
so-called Laplace-Fourier domain FWI~\citep{Sirgue,shin_cha}. The purpose of this outer loop is to first 
fit earlier arrivals, and then fit later arrivals. By fitting first early arrivals (shorter travel-time)
 we reduce the risk of large phase differences between observed and modeled data which 
can cause cycle-skipping. Once the travel-time differences are solved for early arrivals, we can progressively increase $\tau$.
Introducing the time damping requires the following transformation: $\Omega = \omega +i/\tau$, with $\tau$ being the 
time damping \citep{Kamei2013}. Thus, this transformation turns the real-valued 
angular frequency $\omega$ into a complex-valued angular frequency $\Omega$. In order to get consistent observed data
with the damped modeled data, one must also scale the observed data as $\fr(\xx_r,t)=d_{obs}(\xx_r,t)e^{-t/\tau}$ before
the transformation to frequency domain. 

The low frequencies of the data are sensitive to the long wavelength (smooth) components of the 
earth model. However, if the data do not have such frequencies, data-domain wavefield tomography is unable to update such components. 
 In contrast, focusing in extended images is mostly sensitive to the 
smooth components of the model. By implementing 
a joint workflow using image-domain wavefield tomography for updating the smooth components of the model and later using data-domain wavefield tomography for 
the high resolution features of the model, we can obtain a more complete spectrum in the model. The 
first pass using image-domain wavefield tomography has the ability to stabilize the cycle-skipping problems in data-domain wavefield tomography.








% In matrix notation, the process is described by
%%
%\beq \label{eqn:FSV}
%\MAT{ \L(m,t) & 0 \\ 0  & \L^\top(m,t)}
%\MAT{ \us \\ \ur} =
%\MAT{ \fs \\ \fr} ,
%\eeq
%where $m=1/v^2(\xx)$ is the medium slowness squared, $\fs$ is the source function, 
%$\fr$ is the data at the receiver locations,
%$\L(m)$, and $\L^\top(m)$ are forward and backward wave propagators, respectively. In this
%report we use the scalar wave equation as wave operator: 
%\beq
%  \L(m) =\swe{}.
%\label{eq:time_we}
%\eeq
%
%A well-focused gather concentrates most of its energy around $\hh=0$. This 
%can be used as an optimization criterion by minimizing the energy outside $\hh=0$.
% We can accomplish this by defining an objective function
%\beq
%J = \frac{1}{2} \norm{P(\hh)R(\xx,\hh)}^2,
%\label{eq:sof}
%\eeq
%where $P(\hh)$ is the penalty function, which plays a vital role in the inversion. 
%\cite{ShenSymes.geo.2008} propose a mix between $P(\hh)=|\hh|$ and 
%$P(\hh)=\delta(\hh)$. The first penalty function corrects for most kinematic errors,
%whereas the second one improves the focusing of the image $R(\xx,\hh={\bf 0})$.
% \cite{tony_seg:cwp12} propose a penalty operator $P(\hh)$ that accounts for
%illumination, which seeks to bring the defocused data, not to $\hh=0$, but to 
%a region of acceptable focusing within the limits of illumination.
%Depending on the choice of the the penalty operator $P(\hh)$, equation \ref{eq:sof} can be either minimized
%or maximized. In this report, we use $P(\hh)=|\hh|$ as penalty operator primarly for
%computational cost reasons. This penalty
% operator defines a smooth objective function and corrects for most kinematic errors in the
%model. 
%
%Once we have the penalized gathers (image residuals), we 
%compute the adjoint sources \citep{ShenSymes.geo.2008,Wiktor}, 
%\beq
%  g_s(\xx,t) = \sum_\hh P(\hh)^2 R(\xx+\hh,\hh)\UR(\xx+2\hh,t)
%\eeq
%for the source adjoint source, and 
%\beq
%  g_r(\xx,t) = \sum_\hh P(\hh)^2 R(\xx-\hh,\hh)\US(\xx-2\hh,t)
%\eeq
%for the receiver adjoint source. 
%
%\cite{yangwave} and \cite{Shan:chevron} use an alternative formulation for the source side:
%\beq
%  \mbox{ for all $\hh$, do: }   g_s(\xx-\hh,t) += P(\hh)^2 R(\xx,\hh)\UR(\xx+\hh,t),
%  \label{eq:adj_src_src}
%\eeq
%and for the receiver side:
%\beq
%  \mbox{ for all $\hh$, do: }  g_r(\xx+\hh,t) += P(\hh)^2 R(\xx,\hh)\US(\xx-\hh,t).
%  \label{eq:adj_src_rvr}
%\eeq
%
%It turns out that both formulations are equivalent. In the first formulation, we gather information from the 
%vicinity of position $\xx$, whereas in the second one we scatter the residual in the vicinity of 
%$\xx$. In order to get the equivalence between equations 5 and 7 we 
%can simply do a change of variables $\xx' = \xx-\hh$ in 
%equation \ref{eq:adj_src_src}. Similarly, we can do the change
%of variables $\xx' = \xx +\hh$ in equation \ref{eq:adj_src_rvr} to obtain the equivalence between equations
%6 and 8.
%
%Once we have the adjoint sources, we solve
%\beq \label{eqn:FSV}
%\MAT{ \L^\top(m,t) & 0 \\ 0  & \L( m,t)}
%\MAT{ \as \\ \ar} =
%\MAT{ g_s \\ g_r} .
%\eeq
%
%The gradient of equation \ref{eq:sof} with respect to our model parameters
%is defined as follows:
%\begin{align}
%\nonumber \nabla J(\xx) = \sum_{e} \sum_t &\ddot{u}_s({e},\xx,t)a_s({e},\xx,t)+\\
%                          &\ddot{u}_r({e},\xx,t)a_r({e},\xx,t).
%\label{eq:gradiwt}
%\end{align}  
%
%\section{Data domain Wavefield tomography}
%The construction of the tomography problem in the data domain amounts to measuring the error
%(or residual) at the receiver locations. For data domain wavefield tomography, we normally
% use the data difference for the residual:
%\beq
%J = \frac{1}{2} \norm{\US(\xx_r,\Omega)-\fr(\xx_r,\Omega)}^2=\frac{1}{2} \norm{\Delta d}^2,
%\label{eq:offwi}
%\eeq
%where $\xx_r$ are the receiver locations and $\Omega$ is the complex valued frequency whose
%purpose we will explain later. Note that $\fr(\xx_r,\Omega) = \UR(\xx_r,\Omega)$.
%Since for building the residual we only need to forward propagate the source function 
%\beq
%\L({m},\Omega)\US = \fs(\xx_s,\Omega),
%\eeq
% we have to compute one adjoint wavefield $\as(\xx,\Omega)$.
%For data-domain wavefield tomography, computing $\as$ involves backpropagating the data residual:
%\beq
%  \L^\top({m},\Omega) \as = (\Delta d)^{*}.
%\eeq
%Here $(\Delta d)^{*}$ is the complex conjugate residual and $\L({m},\Omega)$ is the acoustic wave 
%equation in the frequency domain, defined as follows:
%\beq
%  \L(m) = -\DIV{\frac{1}{\rho(\xx)}\GRAD{}} -\frac{m(\xx)}{\rho(\xx)}\Omega^2.
%\label{eq:time_we}
%\eeq
%Here $\rho(\xx)$ is the density of the medium. In this report, we do not invert for $\rho(\xx)$, instead
%we parametrize it as a function of the velocity following \cite{Gardner}.
%
%Once we obtain $\US(\xx,\Omega)$ and $a_s(\xx,\Omega)$, we can proceed to compute the gradient:
%\beq
%\nabla J(\xx) = \mathbb{R}\left\{\sum_{e} \sum_t \Omega^2{u}_s({e},\xx,\Omega)a_s^*({e},\xx,\Omega)\right\},
%\label{eq:graddwt}
%\eeq
%here $^*$ denotes complex conjugate and $\mathbb{R}\left\{\right\}$ denotes the real part.
%
%The data domain wavefield tomography objective function, equation \ref{eq:offwi}, is highly non-linear if
%we operate with signals in the normal frequency band. Hence, 
%in order to increase the chances of convergence to the global minimum,
% it is customary to implement the data-domain wavefield tomography in a multi-scale fashion. \cite{Bunks95} propose to 
%first invert lower frequencies and then 
%move gradually to higher frequencies. The idea is that within each scale the problem looks more linear than
%when inverts all the bandwidth at once. 
%
%An additional outer loop in the inversion is the time damping, which leads to the
%so-called Laplace-Fourier domain FWI~\citep{Sirgue,shin_cha}. The purpose of this outer loop is to first 
%fit earlier arrivals, and then fit later arrivals. By fitting first early arrivals (shorter travel-time)
% we reduce the risk of large phase differences between observed and modeled data which 
%can cause cycle-skipping. Once the travel-time differences are solved for early arrivals, we can progressively increase $\tau$.
%Introducing the time damping requires the following transformation: $\Omega = \omega +i/\tau$, with $\tau$ being the 
%time damping \citep{Kamei2013}. Thus, this transformation turns the real-valued 
%angular frequency $\omega$ into a complex-valued angular frequency $\Omega$. In order to get consistent observed data
%with the damped modeled data, one must also scale the observed data as $\fr(\xx_r,t)=d_{obs}(\xx_r,t)e^{-t/\tau}$ before
%the transformation to frequency domain. 
%
%The low frequencies of the data are sensitive to the long wavelength (smooth) components of the 
%earth model. However, if the data do not have such frequencies, data-domain wavefield tomography is unable to update such components. 
% In contrast, focusing in extended images is mostly sensitive to the 
%smooth components of the model. By implementing 
%a joint workflow using image-domain wavefield tomography for updating the smooth components of the model and later using data-domain wavefield tomography for 
%the high resolution features of the model, we can obtain a more complete spectrum in the model. The 
%first pass using image-domain wavefield tomography has the ability to stabilize the cycle-skipping problems in data-domain wavefield tomography.
