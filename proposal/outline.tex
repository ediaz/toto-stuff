\newpage
\section{Thesis outline}

\inputdir{.}
\plot{Table}{width=\textwidth}{Thesis summary: the two rows correspond to the modeling methods
used in this thesis, whereas the two columns correspond to the applications where the modeling
is used. Within each box one can see the specific application pointing to the corresponding
chapter. The arrows connect different concepts and technologies with the corresponding chapter number. }

In this thesis I explore different ways to measure the velocity errors in the subsurface. I 
use several methods in the image-domain and in the data-domain. I discuss different
modeling operators (two-way and Marchenko). \rfg{Table} shows the links between different
combinations of modeling operators with the seismic exploration problem of interest, which
in this thesis is related either to seismic imaging or to wavefield tomography. As 
mentioned before, the two problems are interrelated. In this section I briefly 
describe all the components of my the thesis, including 
past and future projects and the links between them. 

%\subsection{1-Understanding reverse time migration backscattering: noise or signal? (completed: \cite{DiazRTM})}
%In the first project, I describe the kinematic (and focusing) properties of the 
%so-called RTM backscattering. These types of low wavenumber events
%arise in the presence of sharp boundaries in the velocity model. These boundaries 
%produce complex wave propagation which includes complex wave scattering (i.e., the
%waves travel in all directions in both wavefields). During the imaging condition
%(or the extended imaging condition), the low wavenumber event appears 
%when the wavefields are in synchronization (traveling in the same direction) and produce
%a strong correlation along the path connecting the source with the interface, and the 
%interface with the receiver. 
%Commonly, these events are seen as a drawback for the RTM method because they obstruct the image of the geologic
%structure, which is the real objective for the process.
% I show numerical and theoretical analysis for the purpose of understanding the {reverse time migration} backscattering
%energy in conventional and extended images.  I show that the synchronization and focusing information of the backscattering
% is sensitive to velocity errors; this implies that a correct velocity model produces RTM backscattering with maximum energy. Therefore, before
%filtering the RTM backscattered energy, one should try to obtain a model that maximizes it.
%
%\subsection{2-Wavefield tomography using RTM backscattering (completed: \cite{diaz2015})}
%Continuing along the same lines, in this chapter I  propose an optimization workflow where
%both the sediment velocity and the sharp boundary are updated iteratively. 
%The presence of sharp boundaries in the model leads to high and low wavenumber 
%components in the objective function gradient; the high wavenumber components
%correspond to the correlation of wavefields traveling in opposite directions,
% whereas the low wavenumber components correspond to the correlation
%of wavefields traveling in the same direction. 
%This behavior is similar to  
%reverse-time migration where the high wavenumber components represent the 
%reflectors (the signal) and the low wavenumber components represent 
%backscattering (noise). The opposite is true in tomography: the low 
%wavenumber components represent changes to the velocity model and
% the high wavenumber components are noise that needs to be filtered out.
%We use a directional filter based on Poynting vectors during the
% gradient computation to preserve the smooth components of the gradient, thus
%spreading information away from the sharp boundary.
%Our tests indicate that velocity models are better constrained when we include
%the sharp boundaries (and the associated backscattering) in wavefield tomography.
%

\subsection{1- CIP wavefield tomography with illumination compensation (in progress, to be submitted 
to Geophysical Prospecting or Geophysics: \citep{pantin2015optimizing} )}

Image domain wavefield tomography exploits focusing characteristics of extended images 
for updating the velocity field. In order to make good use of this information, one must 
understand how such images behave if the migration velocity is accurate. This is not trivial 
since focusing depends on the acquisition setup, as well as on illumination variation caused by 
the geology separating the acquisition array from the imaged structure, the data bandwidth, etc. I 
address this problem using a combination of migration/demigration to construct penalty functions that
 characterize focusing by incorporating acquisition parameters and data bandwidth. Moreover, instead of
 sampling the extended images at a preset distance along the surface, I sample the image by constructing 
common image-point gathers, which are also much more economical from a computation point of view \citep{SavaVasconselos}. Coupled
 with image residuals exploiting illumination-based penalty functions, I construct robust wavefield 
tomographc updates in areas of poor or uneven illumination \citep{tony_seg:cwp12}. Models obtained with this type of methodology 
are good starting points to more sensitive, but less robust waveform inversion methods. After retrieving
the velocity model with the illumination-based tomography, I continue the inversion using 
waveform tomography in the data-domain. I show an application of the method to synthetic examples 
and to a marine 2D dataset. The examples shown in this chapter depict a 
robust wavefield-based tomography approach from long wavenumbers (to which image-domain
CIP tomography is more sensitive) to short wavenumbers (which data-domain methods
are more sensitive). Hence, I propose a full wavenumber approach and apply it to a real dataset 
example. 

\subsection{2- Seismic tomography using local correlation functions (in progress, \cite{DiazLcorr})}

Wavefield tomography in the data domain is usually formulated by using the
data difference as the misfit criterion. The data difference, however, is 
susceptible to cycle skipping which leads to convergence into local minima. To
overcome the strong non-linearity, a multi-scale approach to the inversion is
often needed. The success of this approach relies on the low-frequency content
of the data. Instead, the data domain tomography misfit criterion
using local correlations. Correlation-based inversions are less
sensitive to local minima than difference-based inversions. Correlations,
however, are often contaminated with cross-talk between events; in addition,
the global correlations give just a general idea of the kinematic errors of
the model because of the summation along the entire time axis. Alternatively, local
correlations with Gaussian windows, advocated in this paper, are able to
extract the local kinematic errors in the misfit between modeled and observed
data. Local correlations are also less sensitive to cross-talk of seismic events
than their global correlation counterpart because the summation is performed
locally as a function of time. Less correlation cross-talk leads to cleaner
cleaner gradients. I further improve the gradients
using a penalty function that is consistent with the bandwidth of the seismic
data, which is more realistic than linear penalty functions designed to annihilate
infinite bandwidth data. This penalty function resembles the illumination-based
penalty functions applied in the previous chapter for image-domain wavefield tomography.

Local as opposed to global correlations provide cleaner measures of similarity between two functions.
The main difference is that local correlation performs a local sum instead of summing 
over all the time samples. This is the key to eliminating the comparison between
 unrelated seismic events present in the observed and modeled data. This feature 
allows us to extract local kinematic errors that are 
more instructive than the misfit extracted using global correlations.

\subsection{3- Extended imaging analysis  using Marchenko wavefields (in progress) }
In this chapter, I plan to use the wavefields reconstructed in the Marchenko framework
for extended imaging. I will analyze the links between extended images and Multi Dimensional Deconvolution (MDD). 
 From the MDD response we can extract angle-dependant reflectivity information in 
a similar way as it is done in extended imaging.
 The reflectivity response from MDD, however, can be potentially more 
consistent because it comes from an inversion framework during the imaging condition which involves temporal and spatial
deconvolution. Also, the input Marchenko wavefields are  an expression of the true 
Green's functions warped by the input background model. The redatumed wavefields are 
consistent with the relative amplitude variations present in the data. Additionally, 
 the extended images computed from Marchenko wavefields contain better illumination because
they incorporate the use of internal and surface-related multiples. 
 In this chapter I also analyze the sensitivity of the Marchenko-based extended images
to errors in the background model which is the first step for a tomography setup. 



\subsection{4- Wavefield tomography using multiple scattered waves (in progress)} 
By properly using multi-scattered waves, one can increase the sensitivity of the seismic
data to errors in the background velocity model. Usually, the surface-related and internal
multiples are handled incorrectly by imaging and tomographic operators. The 
misplaced multiples produce coherent defocused events that drive the inversion
to the wrong direction. The misplaced multiples under the Born approximation 
contain a moveout that the tomographic operator interprets as if the background velocity model is too fast.   
 The use of the Marchenko wavefields can potentially overcome this problem since 
the multiples are imaged at the same place as the corresponding primaries, not only  increasing
the illumination, but also eliminating the bias in the velocity update. 
 Despite errors or smoothness in the background model, the retrieved Marchenko fields
contain all the scattering phenomena that is observed in the data. This means that
this fields obey the wave equation in the background model, but at the same time 
show scattered waves that are an expression of the true recorded reflectivity
at the surface. Using the full scattering nature of the seismic data 
can be a big step forward in tomography because it would eliminate all
multiple removal as a preprocessing step. Instead, one can make use of the multiples
which at this stage can be regarded as an independent additional dataset. 



