\section{Introduction}

Two-way propagators are routinely used in reverse time migration
(RTM)~\citep{baysal:1514, whitmore:382, GPR:GPR413}. In general,
and especially in complex geological settings, the use of two-way
wavefields produce better images than other methods such as Kirchhoff
migration \citep{Schneider,Blestein87} and one-way equation migration
\citep{gazdag:1342,gazdag:124}. RTM's propagation engine, a two-way
wave equation, makes this imaging method robust and accurate
because it honors the kinematics of the wave phenomena by allowing
waves to propagate in all directions regardless of the complexity
in the velocity model. A drawback of the RTM propagation engine
(and any of the mentioned before) is that it fails to properly
handle multiple-scattered waves. Another type of propagator, called
Marchenko \citep{Fil2012,Behura,Wapenaar,Singh2015}, is based on an
integral solution to the inverse scattering problem first proposed
by \cite{rose2002,rose2002single}. This modeling operator correctly
handle primaries and multiple scattered waves; hence, it makes use of
the full-scattered nature of the recorded seismic data by redatuming
the surface observations inside the medium given the kinematics
provided by the background model. In this proposal, I use the two-way
equation and the Marchenko modeling framework to estimate updates
in the background velocity model which is the main driver for the
quality of the seismic image. % % %

\subsection{The role of the velocity model}

The quality of the seismic image depends on the choice of wave
propagator, but more importantly on the quality of the background
model. The better the velocity, the more focused the seismic
image becomes. By analyzing the quality of the seismic image, one
can update the background velocity model in such a way that the
image quality improves. By increasing the focusing in the seismic
images, we are effectively improving the synchronization of the
source and receiver wavefields at the image locations. Image
focusing can be observed in zero-offset sections by measuring the
continuity and image strength, or by analyzing focusing of point
diffractors. However, if the events in the image mostly consist
of continuous reflectors (layers inside the earth), it becomes
hard to assess the accuracy of the background model. In order to
retrieve the information about the kinematic error, one must extend
the image in a larger space. In the context of two-way operators,
the extension is usually performed by some extended correlation
in the image domain ~\citep{rickett:883,sava:S209,GPR:GPR888},
in the angle domain~\citep{sava:1065,yoon,Jin,yoon2,Vyas},
or in the surface offset domain \citep{giboli}. Once a
suitable extension is chosen, the velocity error information
can be extracted by measuring focusing in the extended
gathers~\citep{ShenSymes.geo.2008,Wiktor,tony_seg:cwp12,Shan:chevron,
BiondiAli:2014,diaz2015,tony:gp15}, measuring moveout in the
angle or surface offset gathers \citep{Liu2010,yiShen,Fleury} or,
alternatively, measuring the consistency between individual shot
record images \citep{perrone2015waveform}.

Another school of inversion looks for similarities between simulated
and recorded data directly at the receiver locations. This problem,
usually referred to as Full Waveform Inversion (FWI), is generally
formulated by improving the consistency between modeled and observed
data. Originally, \cite{tarantola} introduced the data difference
as a similarity estimate in the time domain. Alternatively, the
problem can be solved in the frequency domain \citep{Pratt99}.
Contrary to the image-domain formulation, data-domain inversion
is highly non-linear, i.e., the objective function has many
local minima. To overcome this non-linearity, a multi-scale
separation approach is needed \citep{Bunks95}. Within each scale
(frequency or frequency band), the problem can be more linear
if the initial model is closer to the one corresponding to the
global minimum. To tackle the intrinsic non-linearity a great deal
of research has been proposed to adjust the objective function
\citep{ShinHa.geo.2008,shin_cha,Sirgue,engquist2013application,Luo91,
warner,TariqChoi}. In theory, the resolution of FWI can approach
that of a seismic image. Hence, as a byproduct of the inversion,
the resolution of the velocity model can be used for reservoir
characterization, 4D analysis, and improving the seismic image itself
\cite{sirgue20093d}. The great benefits of FWI make solving the issue
of non-linearity an area of very active research.


\subsection{Assumptions of conventional imaging and tomography}

The methods described above rely on the Born modeling approximation,
which means that they can accurately image/describe seismic events
that scatter only once inside the earth. However, the seismic data
can contain strong multi-scattering that appears in the seismic
images in the form of coherent noise, and is characterized by
defocusing (or moveout) in the common-image gathers (CIG). To
overcome this problem, before imaging (or before tomography), the
multiples (either internal or surface-related) are removed from the
recorded data \citep{SRME,ArtWeiglein,guitton2005,Herrmann} or from
the migrated CIGs \citep{SavaGuitton,Wang,Weibull}. If the multiples
are not correctly attenuated, they can contaminate the seismic image
and be misleading during the tomographic inversion.

In Born-based imaging, we assume that the image at point $\xx$ is
constructed from the wavepath that originates at the source location
$\xx_s$, reflects at image point $\xx$, and is recorded at the
surface at point $\xx_r$. Hence, the image point $\xx$ is illuminated
by only one direction (or around two in the case of multi-pathing due
to a low velocity discontinuity). Any other wavepath that connects
the source $\xx_s$ and receiver $\xx_r$ at the image point $\xx$ is
neglected, \rfg{BornImaging}. Alternatively, if we take into account
all the possible paths that connect the source $\xx_s$ with the
receiver $\xx_r$ through the image point $\xx$, we can illuminate
this point from many other different angles, \rfg{MultiImaging}.
Illumination from multiple directions allows to better reconstruct
the representation of the scatterer at point $\xx$ during the imaging
condition process.

On the tomography side of the problem, the Born approximation implies
that the sensitivity kernel updates the velocity model along the
direct path between the source coordinate $\xx_s$ to the image point
$\xx$, and from the receiver $\xx_r$ to the image point $\xx$,
\rfg{BornTomo}. This means that the information extracted at the
image point $\xx$ contains limited sensitivity to the background
model. In contrast, within the multi-scattering framework, the
sensitivity of the image point $\xx$ to the background model is
along all possible paths that connect the source point $\xx_s$ with
receiver point $\xx_r$ at image point $\xx$, \rfg{MultiTomo}, which
implies higher sensitivity to the background model parameters.
\inputdir{.}

\multiplot{4}{BornImaging,BornTomo,MultiImaging,MultiTomo}{width=0.48\textwidth}%
{Depiction of Born and multiple scattering imaging and tomography: (a) shows the path used during Born imaging to construct
the seismic image at point $\xx$, (b) the sensitivity kernel (path) updated by Born-based tomographic updates,
(c) the possible information that can be used in addition to the Born path during  multiple scattered imaging, and (d) 
the sensitivity kernel from the multiple scattered waves.}


%
%\section{Research opportunities}
%
%I plan to use the information commonly regarded to as noise (internal and surface-related multiples) during 
%extended imaging and later during tomography. Figure 1 summarizes the potential benefit of multiply scattered 
%waves to extract angle-dependant information. It also shows the potential benefit of using such waves
%for updating the background model. In order to account for multiple scattering, I will use 
%the Marchenko modeling framework \citep{Fil2012,Behura,Fil2014,Wapenaar,Singh2015} which handles and explains 
%internal and surface-related multiples \citep{Singh2015} during the computation of the Green's functions. 
%Potentially, the use of multiple-scattered waves can improve the sensitivity of the seismic data
%to the model parameters. 
%
%The Marchenko modeling framework depends on the traveltime information computed from the background model;
%then predicts all the arrivals in the data that originate from every image point $\xx$. 
%The main advantage of the Marchenko modeling is that one is able to predict how 
%to place primaries and multiples at the same place, even with an incorrect background model. 
% However, inconsistencies between shot points at the surface can 
%image the event in different places, which produces an error in the semblance-principle sense \citep{Alyahya1989VAiterativeProfileMig}.
% This observation opens the avenue for exploring methods like extended imaging to access the velocity 
%inaccuracies in the model.
%  Another important aspect of the inverse problem is deciding how to measure the residual, i.e. what do we want to minimize? 
% The answer to this question drives the robustness and resolution of the inverse problem. 
%Regardless the choice of method, the inclusion
%of multiple scattered waves makes the problem more robust (we are properly using more data) and at the 
%same time eliminates the misleading effect that multiples have in Born-based approaches. 
%


\section{State of the art}

In this section, I review the current state of the art in the field
of imaging and tomography by discussing the trends and approaches
that have been developed in recent years. I focus my review on
wavefield-based approaches since these methods are of interest in
this proposal. The field of imaging and velocity model building is
interconnected; usually developments on one side lead to improvements
on the other. I identify three great categories for wavefield-based
approaches for velocity model building: data-domain, image-domain,
and mixed-domain.

%I also touch base on the methods that account for multiple-scattering and tomography

\subsection{Image-domain methods}

 Image-domain tomography aims to find a velocity model that
increases the image quality. Usually, every image-domain tomography
implementation utilizes (to some degree) the semblance principle
\citep{Alyahya1989VAiterativeProfileMig} which states that individual
shot record images must be similar to each other. Hence, the velocity
model is optimized by improving the consistency between shot-record
images. More recently, \cite{perrone2015waveform} use a similar
concept to find the background velocity model by minimizing the path
that is needed to warp two neighboring shot record images.


The measurement of the focusing (or inconsistency error)
nowadays is commonly done on RTM images. During tomography,
the errors are backprojected in the model, producing an
update direction which is used during tomography. In order to
keep the backprojection operator consistent with the imaging
operator, tomography methods have moved towards wavefield-based
kernels~\citep{Woodward_1992,SavaBiondi.gp.wemva1,SavaBiondi.gp.wemva
2,
ShenSymes.geo.2008,Xie2008FiniteFreqSensitityKernel,Wiktor,tony_seg:c
wp12,perrone2015waveform}.

Gathers based on subsurface wavefield correlation
lag~\citep{rickett:883,sava:S209,GPR:GPR888} show
the quality of the focusing in the subsurface.
\cite{ShenSymes.geo.2008} extend the concept of differential
semblance optimization~\citep{symes.carazzone} to wave-equation
based tomography. Their method finds a velocity model by
minimizing the defocused energy in the extended gathers. This
method has been improved to account for two-way wavefields
\citep{Wiktor,tony_seg:cwp12,Shan:chevron,BiondiAli:2014,diaz2015}
and to account for different extensions which can be obtained at
sparse locations \citep{tony:gp15}. Alternatively, the model can be
optimized by imposing flatness in angle-domain common-image-gathers
\citep{ursin,biondiAngle,Liu2010,Montel,yiShen}.


\inputdir{.}
\plot{Claerbout}{width=\textwidth}%
{Depiction (figure 1.13 from \cite{Claerbout:1985:IEI:3887}) of 
 the sensitivity of seismic data to the velocity model (0-2Hz) and to the reflectivity ($> 5$Hz). This highlights
the gap existing between tomography and imaging. Nowadays, due to advances in tomography and acquisition technologies
the gap is closing. However, there is still much to do to seamlessly move from tomography to 
imaging.}



\subsection{Data-domain methods}

Even though FWI is an old concept~\citep{lailly1983seismic,tarantola,
Pratt99,Sirgue,VirieuxFWI}, it wasn't until the 2005 EAGE blind test
exercise \citep{billette20052004} that great attention was drawn to
the method. Since then, many articles have been published on the
subject. The goal of FWI is extremely ambitious: it aims to find a
model that explains all the recorded data at the seismic receivers.
The oscillatory nature of the seismic wavefields makes the problem
highly non-linear and very difficult to solve. In order to make the
problem more linear, numerous strategies and modifications have
been proposed. The multi-scale approach \citep{Bunks95} proposes to
solve the problem in a cascaded way: from low frequencies to high
frequencies. At low frequencies, the seismic data is less oscillatory
which makes the problem more linear. In their multi-scale, approach
the result from each band is fed into the next one as opposed to
solving the problem using the full bandwidth at once. However, the
success of the approach relies on the low frequency content of the
acquired data, which unfortunately is not low enough. \rfg{Claerbout}
\citep{Claerbout:1985:IEI:3887} shows a pictorial representation
of the gap of information we observe in the seismic data. From
tomography, we can obtain low wavenumbers in the model, and from
imaging we can retrieve high wavenumbers. However, there is a gap
(2-5Hz at the time he made the sketch) which we cannot recover
yet. \cite{Sirgue} propose a similar continuation approach in the
frequency domain (rather than the time domain). They propose to use a
sparse set of frequencies that produce a continuous sampling of the
model wavenumbers (considering a 1D earth velocity gradient). More
recently, along the same lines, \cite{TariqWavenumber} suggests a
multi-scale approach which directly controls the model wavenumbers
by using a scattering angle filter; in this approach the model is
inverted by solving first for low wavenumbers and later for high
wavenumbers. \cite{Rocha2015elastic} advocate for a scattering angle
filter based on the energy norm. In order to further simplify the
data within the inversion, \cite{ShinHa.geo.2008} and \cite{shin_cha}
propose to damp the data to highlight not only low frequencies
but also the earlier arrivals. In their hierarchical approach,
the earlier arrivals are fitted first, and then the damping is
incrementally relaxed to allow more data into the inversion.

Numerous objective functions have been proposed to tackle the
non-linearity of the FWI problem. \cite{Luo91} promote the inversion
of the travel-time information, rather than the amplitude, by
minimizing the lag position of the maximum correlation between
modeled and observed data. Their approach requires manual picking of
the correlation maximum at each iteration. \cite{van2010correlation}
seek to automatize the method of \cite{Luo91} by minimizing the
energy of the correlation outside zero lag. This method resemble
those of the image domain extended image optimization mentioned
above. \cite{GJI:GJI4970} make an extensive review of several
objective functions and analyze their sensitivity for the global
seismology tomography. \cite{van2013mitigating} propose to mitigate
the local minima problem by changing the optimization approach by
relaxing the constraints on the wave equation, allowing further
matching of the data at the receivers. Along the same lines,
\cite{simon,warner} propose to minimize the energy of the optimum
Wiener filter that matches the model data with the observed data.
\cite{ChoiTariq} propose to use the unwrapped phase together with
Laplace damping to tackle the inherent non-linearity of seismic
waveforms.


\subsection{Mixed-domain methods}

Recently, an emerging trend simultaneously considers the objectives
of image-domain inversion with those of data-domain inversion.
\cite{clement} pioneered this approach by proposing to separate the
inversion into smooth components (the background model) and rough
components (the reflectivity). In performing this separation, the
background model is not contaminated by the local minima related
artifacts. \cite{Xu},\cite{HWang},\cite{Zhou01092015},\cite{Zedong}
show examples of this iterative process where the separation of
scale is performed and the background model is built together with
the reflectivity. With a similar motivation, but with an objective
function that resembles that of \cite{Luo91}, \cite{MaWarping}
propose to use the same scale separation scheme by driving the
traveltime residuals to zero, as opposed to direct waveform matching.

\cite{GPR:GPR698} proposes a least-squares migration scheme for an
extended perturbation, which is described by an extended image.
By mapping the data into this extended reflectivity domain, the
full Born data can be explained since all the kinematic errors are
captured by the image (or model) extensions. This modeling operator
has been used by \cite{BiondiAli:2014} to construct an inversion
framework where the data are always matched (because of the model
extensions), but the kinematic inaccuracies are captured by the
curvature of the gathers. \cite{FleuryPerrone} develop a framework
where both focusing in the extended images and data matching goals
are optimized simultaneously. Instead, \cite{diaz2013data} propose a
cascaded approach where the extended image is optimized in an initial
stage followed by a second stage that seeks to match the data. The
idea of the method is to increase focusing in the initial stage
(which translates into a reduced travel time misfit at the receiver
locations). Once the initial traveltime requirements for FWI are met,
the inversion continues in FWI mode.

\subsection{Towards a more robust inversion}

The approaches mentioned above increase complexity in order
to avoid local minima and introduce more constraints into the
tomographic problem. Moreover, these approaches lack an accurate
modeling operator (which is usually considered acoustic). Much
development has been focused on the multi-parameter inversion
problem which is of great importance for elastic anisotropic
models \citep{barnes2008feasibility,guasch,Kamath,espen,yuting}.
The increasing cost and algorithmic complexity of the solutions to
problems with local minima opens the need for more robust and simple
inversion schemes. Concepts from other fields like mathematics and
medical imaging contain solutions of interest to the geophysical
community. \cite{engquist2013application} for example propose using
the Wasserstein metric as a distance measure (instead of the $L_2$
metric) as a robust way to solve the problem. This metric finds the
optimum path required to transform (warp) one set into the other.
This concept is borrowed from the problem of optimal mass transport
which has applications within the image-registration community.

Another possible solution for finding the global minimum is to
explore stochastic optimization approaches, which might be able
to find models that can fill the gap present in the seismic data
(\rfg{Claerbout}). An additional benefit of such methods is that
they also provide an estimate of the model uncertainty, which can
be helpful for understanding the null space of the problem. On
the computational side, a thousands of forward simulations must
be calculated. Even though this could be done efficiently since
there is no need to store wavefields (as needed for deterministic
solutions) is still too costly for today's computers. \cite{Sen91}
show applications to 1D stochastic inversion; more recently,
\cite{stochasticPisa,SenWorkshop} show examples of 2D stochastic
waveform inversion. In the context of seismic inversion, stochastic
inversion is greatly employed for reservoir characterization
\citep{boschReview}, which could be also included in the full
waveform inversion community.

\subsection{The role of the multiples}

As mentioned in the previous section, imaging and tomography methods
usually follow the leading term in the Born series (the single
scattering). In order to consider models beyond Born, several authors
\citep{GuittonAreal,grion2007mirror,VerschuurMultiples,DanWhitmore,Ma
ndy} have proposed ideas to adapt more conventional technologies to
handle multiple-scattered waves (surface-related in most cases). The
adaptation requires prior separation of primaries and surface-related
multiples. Once the multiples have been separated, during imaging the
primaries can be used as an areal injection source for the downgoing
source wavefield, which correlates at the reflector position with the
backpropagated multiples.

\cite{rose2002,rose2002single} propose an inverse scattering approach
based on focusing and time reversal which retrieves a function
that focuses at an arbitrarily point in an unknown 1D medium.
This approach is extended to 2D and introduced to the geophysical
community by \cite{Fil2012}. Their method properly handles primaries
and internal multiples given a background velocity model with the
appropriate kinematics. It turns out the approach has direct links
with the inverse scattering framework proposed by Vladimir Marchenko
\citep{marchenko2011sturm}.

The multiples within the Marchenko modeling framework are used
together with the primaries in a global prediction procedure
\citep{Behura,Wapenaar,Singh2015}. The input information to the
process are the surface recorded data and a background velocity
model. From the background model, the first arrival is computed
(e.g., by direct finite difference modeling or by using an Eikonal
solver \citep{Behura}). Once the travel time is computed, the
iterative procedure retrieves the Green's function from a point
$\xx$ to the receivers $\xx_r$ at the surface. By computing the
Green's functions at every point in the subsurface grid and by using
reciprocity, one can obtain the Green's functions from each point
at the surface $\xx_r$ to every point $\xx$ in the subsurface. The
retrieved Green's functions are also separated into upgoing and
downgoing components. These components are used for imaging in a
way similar to the conventional one-way and RTM wavefield-based
approaches. The main difference between Marchenko wavefield and more
conventional modeling approaches is that the multiples and primaries
reflection occur at the right interface, regardless of the accuracy
of the background velocity model. However, the consistency between
the observation point $\xx_r$ still depends on the quality of the
background velocity.




