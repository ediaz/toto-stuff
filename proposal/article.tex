../../hw03/report/pcsmacros.tex

% model
\def\m{m}

% cross-correlation
\def\c{c}

% deconvolution
\def\d{d}

% image
\def\r{r}

% adjoint state variable
\def\a{a}
\def\as{{\a_s}}
\def\ar{{\a_r}}

% wavefield
\def\u{u}
\def\us{{\u_s}}
\def\ur{{\u_r}}
\def\uo{{\u_o}}

\def\UU{u}
\def\US{\UU_s}
\def\UR{\UU_r}

% source
\def\fs{{f_s}}
\def\fr{{f_r}}

% adjoint source
\def\g {{g}}
\def\gs{{g_s}}
\def\gr{{g_r}}

\def\xw{\xx,\ww}

\def\ofd{\lp \d \rp}
\def\ofm{\lp \m \rp}
\def\ofdm{\lp \d,\m \rp}
\def\ofum{\lp \u,\m \rp}
\def\ofxw{\lp \xx,\ww \rp}
\def\ofxwm{\lp \xx,\ww,\m \rp}
\def\ofxtt{\lp \xx,\tt \rp}
\def\ofxlt{\lp \xx,\hh,\tt \rp}
\def\oftt{\lp \tt \rp}
\def\oflt{\lp \hh,\tt \rp}
\def\ofl{\lp \hh \rp}
\def\ofml{\lp -\hh \rp}

\def\ofxwtt{\lp \xx,\ww,\tt \rp}

%\def\ee{\textbf{e}}
\def\ee{e}
\def\ofext  {\lp \ee,\xx,t   \rp}
\def\ofexw  {\lp \ee,\xx,\ww \rp}
\def\ofex   {\lp \ee,\xx     \rp}
\def\ofextt {\lp \ee,\xx,\tt \rp}
\def\ofexlt {\lp \ee,\xx,\hh,\tt \rp}
\def\ofexwtt{\lp \ee,\xx,\ww,\tt \rp}

\def\dspace{\mathbb{D}}
\def\uspace{\mathbb{U}}
\def\mspace{\mathbb{M}}
\def\rspace{\mathbb{R}}
\def\wspace{\mathbb{X},\mathbb{\Omega}}

\def\deld{\delta \d}
\def\delu{\delta \u}
\def\delm{\delta \m}

\def\nd{N_d}
\def\nm{N_m}

\def\A{\mathcal{A}}
\def\F{\mathcal{F}}
\def\L{\mathcal{L}}
\def\J{\mathcal{J}}
\def\H{\mathcal{H}}
\def\I{\mathcal{I}}

\def\delJ{\delta \J}

% weight
\def\W{W}
\def\Wd{\W_\d}
\def\Wm{\W_\m}
\def\K{K}

% penalty
\def\P{P}
\def\oftt{\lp \tt \rp}

% shift
\def\T{T}

\def\DOT#1#2{\left< #2 \right>_{#1}}

\def\MAT#1{
\lb \begin{matrix}
#1
\end{matrix} \rb
}

\def\clab{\kanno{12}{-3}{$R\ofx$}}
\def\cor {\kanno{75}{-3}{correct}}
\def\low {\kanno{75}{-3}{low}}
\def\hig {\kanno{75}{-3}{high}}

\newcommand*{\mytext}[2]{%
\noindent\color{#1!80}{%
\parbox{0.98\linewidth}{#2}}}%


\author{\textup{Esteban D\'{i}az}\\
Center for Wave Phenomena,  
Colorado School of Mines}

\title{Extended imaging and tomography under two-way propagators.}

\email{ediazpan@mines.edu}
\righthead{Wavefield tomography}
\lefthead{E. D\'{i}az}
\maketitle

\newif\ifreport
\reporttrue



% ------------------------------------------------------------
\begin{abstract}
Wavefield-based imaging and tomography has proven to be a robust technology
to handle the complexity of the recorded seismic wavefields. 
 By definition, wavefields are consistent with the bandwidth of the seismic data which
can be beneficial to update the desired components of the background velocity
model. Two-way operators are able to correctly handle multi-pathing,
 diving waves, and reflections from steep structures. However, the propagation
fails to handle multiple-scattered waves. I further extend my work on tomography with two-way operators
 by employing the Marchenko extrapolation framework, which not only explains primaries, internal
multiples, but also surface related multiples. Hence, this forward
modeling operator does not follow the conventional Born modeling which is normally used
during imaging and tomography. The sensitivity kernels of the tomography
follow all the possible paths that connect a point in the subsurface with 
a source in the surface which has great potential for filling low 
wavenumber gaps in the model. I plan to investigate on 
robust ways to pose the tomographic problem in order to avoid common 
local minima problems. I have an inversion framework for image domain
wavefield tomography that I plan to adapt for the new two-way modeling 
operator. I predict a great increase in the convergence compared to 
the conventional Born-based tomographic operators I have developed in the 
past.
\end{abstract}
% ------------------------------------------------------------

\section{Introduction}
% 
%
% Two-way propagators importance 
%

 
 Two-way propagators are routinely used in reverse time migration (RTM)~\citep{baysal:1514, whitmore:382, GPR:GPR413}.
 In general, and especially in complex geological settings, the use of two-way wavefields
produce better images than other methods. Imaging methods such as Kirchhoff migration and one-way equation
 migration are based on approximate solutions to the wave equation. Kirchhoff migration,
 a high frequency asymptotic solution to the wave equation, becomes inaccurate for complex velocity models.
 This technique also fails to handle multi-pathing and typically creates the images based on a 
single travel-time arrival (e.g., most energetic or first-arrival). Other methods based on approximations to the wave
 equation, such as phase shift migration~\citep{gazdag:1342}, rely on a v(z) earth model, and further
 approximations are needed to account for lateral variations~\citep{gazdag:124}.
 In addition to earth model considerations, one-way wave equation migration propagates the wavefields in 
either the upward or downward direction; this approximation becomes inexact when the waves 
propagate horizontally. Therefore, this technique fails to properly handle overturning
waves and reflections from steep-dip structures. RTM's propagation engine, a two-way wave equation, 
makes this imaging method robust and accurate because it honors the kinematics of 
the wave phenomena by allowing waves to propagate in all directions regardless of
 the velocity model or the direction of propagation. This method also takes into account,
 in a natural way, multi-pathing and  reflections from steep dips~\citep{Gray2001,Etgen2009}.
 However, none of the mentioned extrapolation engines can handle (in a natural way) multiple-scattered 
waves. 

%
%
%
\subsection{The role of the velocity model}
The quality of the seismic image depends on the choice of wave propagator, but more importantly
 it depends on the quality of the background velocity model. The better the velocity, the more
 focused the seismic image becomes. By analyzing the quality of the 
seismic image one can update the background velocity model in such a way that the image quality improves. 
 By increasing the focusing in the seismic images, we are effectively improving the synchronization
of the source and receiver wavefields at the image locations. 
 The focusing of the image can be observed in zero-offset sections by measuring the 
continuity, image strength, or the focusing of point diffractors present in the seismic data. 
 However, if the events in the image mostly consist of continuous reflectors (layers inside the earth),
it becomes hard to assess the accuracy of the background model. 
 In order to retrieve the information about the kinematic error, one must extend the 
observation (the reflectivity image) in some direction. In the context of two-way operators, 
the extension is usually performed by some extended correlation in the image domain
~\citep{rickett:883,sava:S209,GPR:GPR888}, in the angle domain~\citep{sava:1065,yoon,Jin,yoon2,Vyas}, or 
 in the surface offset domain \citep{giboli}. 
 Once a suitable extension is chosen, the velocity error information can be extracted by measuring 
focusing in the extended gathers~\citep{ShenSymes.geo.2008,Wiktor,tony_seg:cwp12,Shan:chevron,BiondiAli:2014,diaz2015,tony:gp15},
 measuring moveout in the angle or surface offset gathers \citep{Liu2010,yiShen,Fleury}, or alternatively
by measuring the consistency between individual shot record images \citep{perrone2015waveform}.

Another school of inversion looks for similarities between simulated and recorded data
directly at the receiver locations. The problem, usually referred to as Full Waveform Inversion (FWI),
is generally formulated by  improving the consistency between modeled and observed data. 
 Originally, \cite{tarantola} introduced the data difference as 
a similarity estimate in the time domain. Alternatively, the problem 
can be solved in the frequency domain \citep{Pratt99}. Contrary to 
 the image-domain formulation, data-domain inversion is highly non-linear 
i.e., the objective function has many local minima. To overcome the non-linearity,
 a multi-scale separation approach is needed \citep{Bunks95}. Within each scale (frequency
or frequency band), the problem can be more linear if the initial model
is closer to the one corresponding to the global minimum. To tackle the intrinsic
non-linearity a great deal of research have been proposed to change the objective 
function \citep{ShinHa.geo.2008,shin_cha,Sirgue,Luo91,warner,TariqChoi,engquist2013application}. In theory, the resolution
of Full Waveform Inversion can approach that of a seismic image. Hence, as a byproduct of the 
inversion, the resolution of the velocity model can be used for reservoir characterization, 4D analysis,
and to improve the seismic image itself. The great benefits
of FWI make the issue of non linearity an area of very active research. 


\subsection{Assumptions of conventional imaging and tomography}
The methods described above rely on the Born modeling approximation, which means that 
they can accurately image/describe events that bounce (scatter) only once inside the earth. However,
 the seismic data can contain strong multi-scattering  that reflects in the seismic images 
in the form of coherent noise, which presents defocusing (or moveout) in the common-image gathers (CIG) . 
 To overcome this problem, before imaging (or before tomography), the multiples (either internal
or surface related) are removed from the recorded data \citep{SRME,ArtWeiglein,guitton2005,Herrmann} or from the 
migrated CIGs \citep{SavaGuitton,Wang,Weibull}. If the multiples are not correctly attenuated, 
they can contaminate the seismic image and be misleading during the tomographic inversion. 

In Born-based imaging, we assume that the image point $\xx$ is constructed from the 
wavepath that originates at the source location $\xx_s$, bounces at image point $\xx$, and is recorded
at the surface by $\xx_r$. Hence, the image point $\xx$ is illuminated by only 
one angle of incidence $\theta$ at the image point $\xx$. 
 Any other wavepath that connects the source $\xx_s$ and receiver $\xx_r$ at
the image point $\xx$ is neglected, \rfg{BornImaging}. On the other hand, if we take into account
all the possible paths that connect the source $\xx_s$ with the receiver $\xx_r$ at image point $\xx$,
 we can illuminate this point from many different angles, \rfg{MultiImaging}. 


On the tomography side of the problem, the Born approximation implies that the sensitivity 
kernel updates the velocity model along the direct path between the source coordinate $\xx_s$ to 
the image point $\xx$, and from the receiver  $\xx_r$ to the image point $\xx$, \rfg{BornTomo}. This means
that the information extracted at the image point $\xx$ contains very limited sensitivity to
the background model. In contrast, within the multi scattering framework, the sensitivity of the 
image point $\xx$ to the background model is along the path of all possible combinations that connect
the source point $\xx_s$ with receiver point $\xx_r$ at image point $\xx$, \rfg{MultiTomo}.

\inputdir{.}

\multiplot{4}{BornImaging,BornTomo,MultiImaging,MultiTomo}{width=0.48\textwidth}%
{Depiction of Born and multiple scattering imaging and tomography: (a) shows the path used during Born imaging to construct
the seismic image at point $\xx$, (b) the sensitivity kernel (path) updated by Born-based tomographic updates,
(c) the possible information that can be used in addition to the Born path during  multiple scattered imaging, and (d) 
the sensitivity kernel from the multiple scattered waves.}



\section{Research opportunities}

I plan to use the information commonly regarded to as noise (internal and surface-related multiples) during 
extended imaging and later during tomography. Figure 1 summarizes the potential benefit of multiply scattered 
waves to extract angle-dependant information. It also shows the potential benefit of using such waves
for updating the background model. In order to account for multiple scattering, I will use 
the Marchenko modeling framework \citep{Fil2012,Behura,Fil2014,Wapenaar,Singh2015} which handles and explains 
internal and surface-related multiples \citep{Singh2015} during the computation of the Green's functions. 
Potentially, the use of multiple-scattered waves can improve the sensitivity of the seismic data
to the model parameters. 

The Marchenko modeling framework depends on the traveltime information computed from the background model;
then predicts all the arrivals in the data that originate from every image point $\xx$. 
The main advantage of the Marchenko modeling is that one is able to predict how 
to place primaries and multiples at the same place, even with an incorrect background model. 
 However, inconsistencies between shot points at the surface can 
image the event in different places, which produces an error in the semblance-principle sense \citep{Alyahya1989VAiterativeProfileMig}.
 This observation opens the avenue for exploring methods like extended imaging to access the velocity 
inaccuracies in the model.
  Another important aspect of the inverse problem is deciding how to measure the residual, i.e. what do we want to minimize? 
 The answer to this question drives the robustness and resolution of the inverse problem. 
Regardless the choice of method, the inclusion
of multiple scattered waves makes the problem more robust (we are properly using more data) and at the 
same time eliminates the misleading effect that multiples have in Born-based approaches. 



\section{What is being done now?}
In this section, I review the current state of the art in the field of imaging and tomography, 
by discussing the trends and approaches that have been developed in the past years. I focus my 
review on wavefield-based approaches since these methods are of interest in this proposal. The field of 
imaging and velocity model building is interconnected; usually developments on one side lead to 
improvements in the other.
I identify three great categories for wavefield-based approaches
 for velocity model building: data-domain, image-domain, and mixed-domain. 

%I also touch base on the methods that account for multiple-scattering and tomography



\subsection{Image-domain}
 Image-domain tomography aims to find a velocity model that increases the image quality
in some sense. Usually, every image-domain tomography implementation utilizes (in some sense)
the semblance principle \citep{Alyahya1989VAiterativeProfileMig} which  
 states that the individual shot record images should be similar to each other regardless the shot index. 
Hence, the velocity model is optimized by improving the consistency between shot-record 
images. This concept was later extended to improve the image consistency as a function
of offset (a common product of Kirchhoff migration) or scattering angle 
\citep{stork,Chavent1995MultipleMigrationFitting,Ji}. 
 Once the velocity error is measured, the update is constructed by backprojecting 
the error along the sensitivity kernel which depends on the migration modeling operator. 

With advances in theory and computing power, ray-based methods have been substituted
during imaging by wave-based methods like one-way migration~\citep{gazdag:124}
or RTM migration \citep{baysal:1514, whitmore:382, GPR:GPR413}. In order to keep 
the backprojection operator consistent, tomography methods have moved  towards 
wavefield-based kernels~\citep{Woodward_1992,SavaBiondi.gp.wemva1,SavaBiondi.gp.wemva2,
Xie2008FiniteFreqSensitityKernel}. 

Gathers based on subsurface wavefield correlation lag~\citep{rickett:883,sava:S209,GPR:GPR888}
show the quality of the focusing in the subsurface. \cite{ShenSymes.geo.2008} were the first authors
to exploit this observation. They extended 
the concept of differential semblance optimization~\citep{symes.carazzone} to wave-equation
based tomography. Their method finds a velocity model by minimizing the defocused energy in
 the extended gathers. This method has been improved to account for two-way wavefields
\citep{Wiktor,tony_seg:cwp12,Shan:chevron,BiondiAli:2014,diaz2015} and to account 
for different extensions which can be obtained at sparse locations \citep{tony:gp15}. Alternatively,
 the model can be optimized by imposing flatness in angle-domain common-image-gathers
\citep{biondiAngle,ursin,Montel,Liu2010,yiShen}. 
 The method of \cite{perrone2015waveform} does not rely on common-image-gathers, instead they
 take a similar approach to \citep{Alyahya1989VAiterativeProfileMig} to find a velocity 
model that produces consistent shot record images. Their approach seeks to minimize the
path that is needed to warp two neighboring shot record images. 

\subsection{Data-domain}

Even though full waveform inversion is an old conceptual technology~\citep{tarantola,lailly1983seismic,
Pratt99,Sirgue,VirieuxFWI}, 
it wasn't until the blind tomography exercise at the EAGE conference in 2004 \citep{billette20052004}
that full attention was drawn to the method. Since then, hundreds of works have been 
published on the subject \citep{guittonTLE}. The goal of full waveform inversion 
is extremely ambitious: it aims to find a velocity model that explains all the recorded data
at the seismic receivers. The oscillatory nature of the seismic wavefields makes the problem
highly non-linear and very difficult to solve. In order to make the problem more
linear, numerous strategies and modifications have been proposed. The multi-scale approach \citep{Bunks95}
proposes to solve the problem in a cascaded way: from low frequencies to high frequencies. The
result from each band is fed into the next one, as opposed to solving the problem using the 
full bandwidth at once. However, the success of the approach relies on the 
low frequency content of the acquired data, which unfortunately (or fortunately some my say) is not 
low enough. 
 \rfg{Claerbout} \citep{Claerbout:1985:IEI:3887} shows a pictorial representation of the gap of 
information we observe in the seismic data.  From tomography, we can obtain
low wavenumbers in the model, from imaging we can retrieve high wavenumbers. However, there is a 
gap (2-5Hz at the time he made the sketch) which we cannot recover yet. 
 \cite{Sirgue} propose a similar continuation approach in the frequency domain (rather than the time domain). 
They suggest keeping an eye on the model wavenumbers to decide which frequencies to use. More
recently, along the same line, \cite{TariqWavenumber} suggests a multi-scale approach which directly controlls
the model wavenumbers by using a scattering angle preconditioner; in this approach the model is inverted
by solving first for low wavenumbers and later for high wavenumbers. In order to further simplify the 
data within the inversion, \citep{ShinHa.geo.2008,shin_cha} propose to  damp the data
to highlight not only low frequencies but also the earlier arrivals. In their hierarchical 
approach, the earlier arrivals are fitted first, and then the damping is incrementally  relaxed to allow more
data into the inversion. 

Numerous objective functions have been proposed to tackle the non-linearity of the
FWI problem. \cite{TariqChoi,ChoiTariq} propose to use the unwrapped phase together with 
Laplace damping to tackle the inherent nonlinearity of seismic waveforms. \cite{GJI:GJI4970} makes
an extensive review of several objective functions and analyze the sensitivity of each of them
for global seismology tomography. \cite{van2013mitigating} propose to mitigate the local
minima by changing the optimization
approach which relaxes the constraints on the wave equation, allowing further matching of the data
at the receivers. \cite{Luo91} promote the inversion of the travel-time information, rather
than the amplitude, by minimizing the lag position of the maximum correlation between
modeled and observed data. Their approach requires manual picking of the correlation maximum
at each iteration. \cite{van2010correlation} seek to automatize \citep{Luo91} method
by minimizing the energy of the correlation outside zero lag. This method resemble those 
of the image domain extended image optimization mentioned above. Along the same lines, 
\cite{simon,warner} propose to minimize the energy of the optimum Wiener filter that
matches the model data with the observed data. \cite{warner} claim that their approach is 
immune to cycle skipping problems. 

\subsection{Mixed-domain}

I have been observing a trend that merges the objectives
of image-domain inversion  with those of data-domain inversion. \cite{clement} 
 pioneered  this approach by proposing to separate the inversion into smooth 
components (the background model) and rough components (the reflectivity). In performing
this separation, the background model is not contaminated by the local minima related artifacts. 
 \cite{Xu,HWang,Zhou01092015,Zedong} show examples of this iterative process where 
the separation of scale is performed and the background model is built together with the
reflectivity. Along the same lines, but with an objective function that resembles that of \cite{Luo91},
 \cite{MaWarping} propose to use the same scale separation scheme by driving 
the traveltime residuals to zero, as opposed to direct waveform matching. 
 
\cite{GPR:GPR698} propose a least-squares migration scheme for an extended perturbation, which
 is described by an extended image. By mapping the data into this extended reflectivity domain, 
 the full Born data can be explained since all the kinematic errors are captured by the lagged
extensions. This modeling operator has been used by \citep{BiondiAli:2014} to construct
an inversion framework where the data is always matched (because of the model extensions), 
 but the kinematic inaccuracies are captured by the curvature present in the gathers. 
 \cite{FleuryPerrone} develop a framework where both focusing in the  extended images and data matching 
goals are optimized simultaneously. Instead, \cite{diaz2013data} propose a cascaded 
approach where the extended image is optimized in an initial stage followed by a second stage 
that seeks to match the data.

\subsection{Towards a more robust inversion}

The approaches above have increased in complexity to avoid local minima
 and have introduced more constraints and inversion goals into the problem.  
 Not to mention, these approaches lack an accurate modeling operator
(which is usually considered acoustic). Much development has been focused on the multi-parameter inversion
problem which is of great importance for elastic anisotropic models (e.g., \cite{Kamath}). 
The increasing cost of the solutions to problems with local minima open
an avenue for more robust and simple inversion schemes. Perhaps concepts from 
other fields like mathematics and medical imaging contain solutions of interest 
to the geophysical community. \cite{engquist2013application} propose using the 
Wasserstein metric as a distance measure (instead of the $L_2$ metric) as a robust 
way to solve the problem. This metric finds the optimum path required to
 transform (warping) one set into the other. This concept is borrowed from the problem of optimal mass transport
which has applications within the image-registration community. The idea 
resembles the warping travel time inversion approach \citep{MaWarping,LuoHale}. Further 
analysis needs to be done to understand the links between the Wasserstein metric
and Dynamic Image Warping \citep{hale2013dynamic}  methods.


Another possible solution for finding the global minimum is to explore 
stochastic optimization approaches, which might be 
desirable to find models that can fill the gap present in the seismic data (\rfg{Claerbout}). An additional benefit
of such methods is that they also provide an estimate of the model uncertainty,
which can be helpful for understanding
the null space of the problem. On the computational side, a great amount of forward simulations
must be calculated. However, it could be done efficiently since 
there is no need to store wavefields (as needed for deterministic 
solutions). \cite{Sen91} show applications to 
1D stochastic inversion; more recently, \citep{stochasticPisa,SenWorkshop} show examples
of 2D stochastic waveform inversion. 

\subsection{Role of the multiples}
As mentioned in the previous section, the imaging and tomography methods usually follow
the leading term in the Born series (the single scattering). There are, however, methods
that deal with the problem of direct inversion (where all the arrivals in the seismic 
data are simultaneously inverted). Examples of these methods include the inverse scattering series \citep{Weglein2003}
or the applications of the De Wolf series \citep{Jakobsen,Ru-ShanWu,WuZheng}.

Several authors \citep{GuittonAreal,VerschuurMultiples,grion2007mirror,DanWhitmore,Mandy} have 
proposed ideas to adapt more conventional technologies 
to handle multiple-scattered waves (surface-related in most cases).
 The adaptation requires prior separation of primaries and surface-related multiples. 
Once the multiples have been separated, during imaging the primaries can be 
used as an areal injection source for the downgoing source wavefield, which 
correlates coherently with the backpropagated multiples.

The multiples within the Marchenko modeling framework are used 
together with the primaries in a global prediction procedure \citep{Fil2012,Behura,Wapenaar,Singh2015}.
The input information to the process is the surface recorded data and 
a background velocity model. From the background model, the first arrival is 
computed using an Eikonal solver. Once the travel time is computed, the 
iterative procedure retrieves the green function from a point $\xx$ to the receivers $\xx_r$ at the surface. By 
computing the green functions at every point in the subsurface grid and  by using 
reciprocity, one can obtain the green functions from each point in the surface $\xx_r$ to 
every point $\xx$ in the subsurface. The retrieved green functions are separated 
into upgoing and downgoing components. This components are used for imaging 
in a way similar to the conventional one-way and RTM wavefield-based approaches. 


\inputdir{.}
\plot{Claerbout}{width=\textwidth}%
{Depiction (figure 1.13 from \cite{Claerbout:1985:IEI:3887}) of 
 the sensitivity of seismic data to the velocity model (0-2Hz) and to the reflectivity ( $> 5$Hz). It highlights
the gap existing between tomography and imaging. Nowadays, due to advances in tomography and acquisition technologies
the gap is closing. However, there is still much to do to seamless move from tomography to 
imaging.}



\section{Review of Born-based  extended imaging/tomography}
\label{sec:tomo}

In this section, I review image-domain wavefield tomography 
 using extended-image gathers~\citep{rickett:883,SavaVasconselos}. These kinds of gathers
highlight the spatial and temporal consistency between wavefields by exploring
 the focusing information in the image domain. 
 The moveout in the gather is sensitive to
velocity perturbations, and hence can be optimized. The most
general extended gather can be defined as follows:
\beq
R(\xx,\hh,\tau)= \sum_{{e}} \sum_{t} \US({e},\xx - \hh,t-\tau) \UR({e},\xx+\hh,t+\tau),
\label{eq:eic}
\eeq
 where $\hh$ is the space-lag vector, $\tau$ the time-lag, $\xx$ the image location, $e$ the
experiment index, $\US$ the 
 source wavefield, and $\UR$ the receiver wavefield. Given the increase in dimensions 
for extensions in all directions and time, one can take advantage of sparse sampling 
of the extended images in a subset $\xx_c$ instead of the full image space $\xx$. One smart
way to  decide the locations $\xx_c$ is by placing the observation points at the reflector
locations \citep{cullison}.  
 Note that the process for computing the extended
image is linear with respect to one of the wavefields. Hence, we can 
define the extended image in a matrix-vector form as 
\beq
  {\bf r} = {\bf I}_{s} {\bf u}_r = {\bf I}_{r} {\bf u}_s,
\label{eq:eicops}
\eeq
where $\bf r$ is the vector representation of the extended image, and 
 ${\bf I}_s$ and ${\bf I}_r$ are the imaging operators for the source and receiver wavefield,
respectively. 

Even though the commutation of the forward mappings (equation~\ref{eq:eicops})
 produces the same result ($\bf r$), the adjoint 
mapping satisfies different equations. The operator ${\bf I}_s^\top$ implements
\beq
  \tilde{\UR}(e,\xx+\hh,t+\tau) += \sum_{\tau,\hh}R(\xx,\hh,\tau) \US(e,\xx - \hh,t-\tau),
\label{eq:adj1}
\eeq
whereas ${\bf I}_r^\top$ satisfies
\beq
  \tilde{\US}(e,\xx-\hh,t-\tau) += \sum_{\tau,\hh}R(\xx,\hh,\tau) \UR(e,\xx + \hh,t+\tau).
\eeq

The source and receiver wavefields follow the two-way wave equation, which in our case
is 
\beq
  \swe{u(\xx,t)} = f(\xx,t),
\eeq
where $\m=s^2(\xx)$ is the squared slowness. The source wavefield $\US$ is computed by forward
propagation of the source function $\fs$, whereas the receiver wavefield $\UR$ is created by backward (adjoint)
propagation of the recorded data $\fr$ as summarized by the following system of equations:
\beq 
\MAT{ \L(m,t) & 0 \\ 0  & \L^\top(m,t)}
\MAT{ \us \\ \ur} =
\MAT{ \fs \\ \fr} ,
\label{eq:forw}
\eeq
where $\L$ and $\L^\top$ are the matrix representation of the forward and adjoint (reverse) propagators, respectively. 

One can learn a great deal about the velocity model by looking at the focusing information
encoded in the extended images. A model with good kinematic agreement with 
the true background velocity exhibits 
a maximum correlation between source and receiver wavefields around $(\hh,\tau)=(\vec{0},0)$. Under velocity 
inaccuracies, the
extended gathers exhibit a moveout that is indicative of 
 errors in the model~\citep{YangSava:moveout}. Hence,
 the energy on the gathers can be brought to around zero lag through an
inversion process in which the velocity model can be updated. A typical 
objective function \citep{ShenSymes.geo.2008,Wiktor,tony:gp15} based on 
extended images resembles 
\beq
 J(\m) = \norm{P(\hh,\tau) R(\xx,\hh,\tau)}^2,
\label{eq:jim}
\eeq
where $P$ is a penalty operator whose purpose is to 
highlight the events in the image that show velocity inaccuracies. 

Conceptually, the gradient is constructed by spraying (backprojecting) the errors measured by equation~\ref{eq:jim}
from the image point $\xx$ to the source location $\xx_s$ and from the image point $\xx$ to the receiver
point $\xx_r$. Hence, the total gradient is constructed by integrating all the contributions of the 
kernels connecting the points $\xx_s-\xx-\xx_r$. \rfg{ss-xx-kernel-fsm}  shows an example of 
a kernel under the single-scattering assumption, one can see there is only one path
at connects an image point to the source point $\xx_s$. Similarly, there is also one path
that connects the image point $\xx$ with the receiver $\xx_r$. The width of the kernel is inversely proportional
to the bandwidth of the data. 


\section{Review of Marchenko modeling/imaging}
In this section, I briefly describe the modeling method presented in \cite{Singh2015}. Their method
describes the total green function $G(\xx_i',\xx_0,\omega)$ which includes
primaries, internal, and surface-related multiples. In their notation, $\xx_0$ is a point at the surface $z=0$, and
$\xx_i$ is a point at the depth level $z= z_i$.

For every point $\xx_i$ one wants to solve for, two focusing $f_1$ and $f_2$ solutions are first found. 
The focusing function $f_1$ is solution to the wave equation that produces a wavefield which completely 
focuses at point $\xx_i$, given data at the surface $\xx_0$. Instead, the focusing function
 $f_2$ produces a focused wavefield at the point $\xx_0$ given, a reflection response 
from level $\xx_i$. 

Once the focusing solutions are found through the algorithm described by \cite{wapenaar2014green} (and 
extended by \cite{Singh2015} to handle free surface multiples) the upgoing and downgoing 
components of the green function G are found. The directional components are complementary and satisfy 
the relation
\beq
 G(\xx,\xx_0,\omega) = G^-(\xx,\xx_0,\omega) +G^+(\xx,\xx_0,\omega).
\eeq

In their notation, the wavefield $G^+$ represents the downgoing component of the green function, 
whereas $G^-$ represents the upgoing component (the $z$ axis increases with depth). 
 Following the imaging principle \citep{unified} the image $R(\xx)$ can be constructed 
where both wavefields coincide in space and time (or frequency $\omega$): 
\beq
  R(\xx) = \sum_{\xx_0}  G^+(\xx,\xx_0,\omega)\overline{G}^-(\xx,\xx_0,\omega),
\eeq
where $\overline{G}^-$ is complex conjugate of $G^-$. Alternatively the 
image can be found through the process of Multi Dimensional Deconvolution (MDD) proposed in \cite{Joost},
which solves the following linear problem
\beq
 G^-(\xx_i',\xx_0'',t) = \int_{\partial D_i} d\xx_i \int_{-\infty}^{\infty}  G^+(\xx_i,\xx_0'',t-t') R(\xx_i',\xx_i,t') dt'.
\label{eq:mdd} 
\eeq
Once the reflection response is found, the seismic image is defined as  
\beq
R(\xx_i') = R(\xx_i',\xx_i=\xx_i',t'=0).
\eeq

\subsection{Connecting Marchenko and conventional imaging}
One can think of the Marchenko framework as a forward-modeling procedure that replaces
the wavefields $\US$ and $\UR$ used in conventional imaging. In fact, $\US$ is 
an approximation of the upgoing field $G^+$, whereas $\UR$ approximates $G^-$. 
 The main difference with conventional wavefield extrapolation 
is that $\US$ and $\UR$ are far from satisfying the 
relation 
\beq
  G(\xx,\xx_0,t) = \US(\xx,\xx_0,t) +  \UR(\xx,\xx_0,t).
\eeq

In terms of imaging, the reflectivity response in equation~\ref{eq:mdd} clearly resembles
the extended image found in equation~\ref{eq:eic}. The extended imaging condition can 
be directly applied to the Marchenko wavefields in the same way as with $\US$ and $\UR$. 

If we redefine $\xx_i' = \xx_i + \hh$ and $t'=\tau$ and substitute it in equation~\ref{eq:mdd} we obtain:
\beq
 G^-(\xx_i +\hh,\xx_0'',t) = \int_{\partial D_i} d\xx_i \int_{-\infty}^{\infty}  G^+(\xx_i,\xx_0'',t-\tau) R(\xx_i+\hh,\xx_i,\tau) d\tau,
\label{eq:mdd} 
\eeq
 which basically shows the same wavefield relation as equation~\ref{eq:adj1}.



Figures \ref{fig:00_low}-\ref{fig:02_high} show an application of the extended imaging
condition to the Marchenko wavefields. The three images indicate the sensitivity of the 
method to velocity errors of -20\%, 0\%, and +20\%, respectively. 
 The possibility of measuring velocity errors opens great avenues for research 
in tomography. The multiple scattered waves better sample the velocity model compared
to primary reflections. \rfg{ss-xx-kernel-f} shows the sensitivity kernel of 
a point scatterer $\xx=(4,1)$km in the subsurface to the experiment pair composed
by a point source $\xx_s=(1,0)$km and a receiver $\xx_r=(7,0)$km. 
A comparison of the kernels in \rfg{ss-xx-kernel-f} and \rfg{ss-xx-kernel-fsm} 
illustrate that the free surface waves (together with the primaries) 
contain a richer sampling of the subsurface.





\inputdir{marchenko}

\multiplot{3}{00_low,01_correct,02_high}{width=0.3\textwidth}%
{Preliminary results showing the sensitivity of the imaging process using the Marchenko modeling 
framework to errors in the background model: (a) a -20\% velocity error, (b) the correct 
background velocity, and (c) a +20\% velocity error. Note how the multiples below the deepest
reflector are very weak with respect to the amplitude of the imaged interfaces. }

\inputdir{kernels}
\multiplot{4}{vel,den,ss-xx-kernel-fsm,ss-xx-kernel-f}{width=0.9\textwidth}%
{Band limited sensitivity kernels: (a) shows the background velocity model, (b) is the density model, (c) is an image
kernel under the single scattering assumption connecting source $\xx_s =(1.0,0)$km  and
  image point $\xx=(4.0,1.0)$km with receiver point at $\xx_r=(7.0,0.0)$km, and (d) the sensitivity 
kernel connecting $\xx_s-\xx-\xx_r$ with every possible path given by the multiples and primaries. }




\section{related research}
\subsection{Understanding reverse time migration backscattering: noise or signal? \citep{DiazRTM}}

Reverse time migration backscattered events are produced by the cross-correlation between waves reflected from 
sharp interfaces (e.g. salt bodies). These events, along with head waves and diving waves, produce
the so-called {reverse time migration} artifacts, which are visible as low wavenumber energy on migrated images. 
Commonly, these events are seen as a drawback for the  {reverse time migration} method because they obstruct the image of the geologic
structure, which is the real objective for the process.
 In this paper, we perform numeric and theoretical analysis for the purpose of understanding the {reverse time migration} backscattering
energy in conventional and extended images.  We show that this type of backscattering contains a measure of the synchronization
 and focusing information between the source and receiver wavefields. We also show  that this synchronization and focusing information
 is sensitive to velocity errors; this implies that a correct velocity model produces RTM backscattering with maximum energy. Therefore, before
filtering the RTM backscattered energy we should try to obtain a model that maximizes it.

\inputdir{paperI}
\multiplot{4}{Cig6,Cigx6,Cip06,Image}{height=0.26\textheight}{%
Sigsbee 2a analysis: time-shift gather (a), space-lag gather (b), %
common image point (c) and RTM image (d). The vertical line and thick point shown in the RTM image shows the CIG and CIP locations respectively.}	


\subsection{Wavefield tomography using RTM backscattering \citep{diaz2015}}
The synchronization between backscattered wavefields depends on the velocity of the sediment section and the correct
interpretation of the sharp boundary. We propose an optimization workflow where
both the sediment velocity and the sharp boundary are updated iteratively. 
The presence of sharp boundaries in the model leads to high and low wavenumber 
components in the objective function gradient; the high wavenumber components
correspond to the correlation of wavefields traveling in opposite directions,
 whereas the low wavenumber components correspond to the correlation
of wavefields traveling in the same direction. 
This behavior is similar to  
reverse-time migration where the high wavenumber components represent the 
reflectors (the signal) and the low wavenumber components represent 
backscattering (noise). The opposite is true in tomography: the low 
wavenumber components represent changes to the velocity model and
 the high wavenumber components are noise that needs to be filtered out.
We use a directional filter based on Poynting vectors during the
 gradient computation to preserve the smooth components of the gradient, thus
spreading information away from the sharp boundary.
Our tests indicate that velocity models are better constrained when we include
the sharp boundaries (and the associated backscattering) in wavefield tomography.


\inputdir{paperII}
\multiplot{6}{cmodel-iter000,img_iter000,ximgiter000,cmodel-iter014,img_iter014,ximgiter014}
{width=0.31\textwidth}%
{RTM backscattering optimization: (a) is the initial model, (b) RTM image obtained with the initial model,
(c) extended image gathers from the initial model (observe the energy spread away zero offset). 
The bottom row shows the updated (d) velocity model, (e) seismic image, and (f) extended image gathers 
which show better focusing around $\lambda_x=0$.}



\subsection{Optimizing the input model for waveform inversion using image-domain
 wavefield tomography with illumination compensation \citep{pantin2015optimizing}}

Image domain wavefield tomography exploits focusing characteristics of extended images 
for updating the velocity field. In order to make good use of this information, one must 
understand how such images behave if the migration velocity is accurate. This is not trivial 
since focusing depends on the acquisition setup, as well as on illumination variation caused by 
the geology separating the acquisition array from the imaged structure, the data bandwidth, etc. We 
address this problem using a combination of migration/demigration to construct penalty functions that
 characterize focusing by incorporating acquisition parameters and data bandwidth. Moreover, instead of
 sampling the extended images at a preset distance along the surface, we sample the image by constructing 
common image-point gathers, which are also much more economical from a computation point of view. Coupled
 with image residuals exploiting illumination-based penalty functions, we obtain robust wavefield 
tomography in areas of poor or uneven illumination. Models obtained with this type of methodology 
are good starting points to more sensitive, but less robust waveform inversion methods. We show
an application of the method to synthetic examples and  to a marine 2D dataset.


\inputdir{paperIII}

\multiplot{3}{image-winit,image-wtomo,image-wfwi}{width=\textwidth}%
{RTM images from (a) the initial model, (b) illumination-based tomographic model, and (c) data-domain inversion model.}


\section{Research plan}
I propose to address the following projects:

\begin{itemize}
  \item November 2015 -- February 2016

        Analysis of the connection between Multi Dimensional Deconvolution (MDD) \citep{Wapenaar} and
        the extended imaging framework, from which we could potentially extract 
        and analyze the angle dependence of the reflectivity. From the angle 
        dependence, we could analyze the influence of seismic multiples in the 
        subsurface illumination.
        In parallel, I will perform a sensitivity analysis of the Marchenko-based 
        extended images and evaluate the changes in the curvature of the gathers. 
        The reflection response obtained with the Marchenko wavefields through 
        MDD should be better than that obtained by least squares extended imaging 
        \citep{GPR:GPR698}. 

  \item January 2016 -- July 2016 

        Development of the theoretical framework behind an inversion using 
        multiple scattered wavefields. The connection with extended imaging tomography could 
        be drawn, and hence few pieces of the framework already developed \citep{diaz2013data,diaz2015} 
        would need to be substituted. Alternatively, I want to
        explore and incorporate other measurement objectives which could 
        be done at the receiver positions. The alternative way can avoid
        extra computation of green functions inside the model. 

\end{itemize}


\section{Acknowledgements} 
%I would like to thank my mentor Dr. Paul Sava for many interesting discussions and for 
%supporting my independent quest to new topics. Also, I would like to acknowledge 
%many discussions taken over the last years about tomography and imaging topics with
%Francesco Perrone, Tongning Yang, Cl\'{e}ment Fleury, Simon Luo, Jyoti Behura,
% Nishant Kamath, Satyan Singh, and Yuting Duan. I want to thank Satyan for 
%providing the Marchenko modeling framework I will use during this part 
%of my thesis. 



% ------------------------------------------------------------
\bibliographystyle{seg}
\bibliography{SEG,SEG2,SEG3,MISC}
% ------------------------------------------------------------
