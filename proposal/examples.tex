\section{Application to a real 2D dataset}

\inputdir{rtm}
\multiplot{2}{shot,spectrum}{width=0.42\textwidth}{(a) Shot gather at $x=14$ km and 
(b) its average amplitude spectrum.}
\multiplot{3}{veldepth,RTM-veldepth,xang-veldepth,bar_veldepth4}{width=\textwidth}{%
(a) Initial velocity model, (b) corresponding RTM image, (c) angle-gathers at sparse locations,
and (d) the colorbar for velocity plots used in this report.}

\multiplot{3}{vel_fwi_deldepth,RTM-vel_fwi_deldepth,xang-vel_fwi_deldepth}{width=\textwidth}{%
(a) The data-domain wavefield tomography velocity model obtained from \rfg{veldepth}, (b) corresponding RTM image, and (c) angle-gathers at sparse locations.
The velocity ranges from 1.5km for water velocity to 4km/s in the deepest part.}

In this section we apply the cascaded workflow of image-domain wavefield tomography followed by 
data-domain wavefield tomography.
We use image-domain wavefield tomography to correct for most kinematics errors, and then 
data-domain wavefield tomography to 
refine the model and add details. We compare 4 models: (a) the initial model,
 (b) the model obtained by image domain wavefield tomography, (c) the data-domain wavefield tomography obtained from (a), and 
 (d) the data-domain wavefield tomography starting with the model obtained by image-domain wavefield tomography (b).

The dataset is a marine 2D line acquired with a variable depth cable.
The towed streamer contains increasing depths as a function of offset, which enhances the frequency
content of the data by producing a mixed notch response. Hence,
the increased cable depths improves the low frequency content at intermediate and far offsets which can be very helpful
for data-domain wavefield tomography. The cable contains offsets ranging from $0.169\text{ to }8.256$ km. \rfg{shot} shows a shot
gather from $x=14$ km and \rfg{spectrum} depicts the average amplitude spectrum for
the same gather. 

We build the initial model,~\rfg{veldepth}, by performing time-domain NMO analysis followed
by smoothing, RMS (stacking) conversion to interval velocity \citep{dix}, and 
time to depth conversion.~\rfg{RTM-veldepth} shows the RTM
 image produced by the model in~\rfg{veldepth}, and one can observe that the image is over migrated
 (high velocity) below 3km in depth.~\rfg{xang-veldepth} shows angle gathers extracted at
sparse locations in the model. Note that we do not use the angle gathers for inversion; instead,
we use the gathers as an independent quality control tool. The transformation from space-lag
gathers $R(\xx,\hh)$ to angle domain $R(\xx,\theta)$ follows the method of \cite{sava:1065}.
The angles vary from $0\text{ to }45^\circ$ for all
the gathers shown in this report.
 The moveout in the gathers confirms that the velocity is 
too fast below $3$ km. Some of the events in the gathers, however, correspond to migrated surface related 
multiples and their moveout is not indicative of velocity error.

For data-domain wavefield tomography, we use 7 frequency blocks with 5 frequencies each. The center frequency
for each block ranges from $f=2.6$ Hz to $f=8.9$ Hz. For the time damping constant, we use $\tau=1.6s$.
 The first step in data domain wavefield tomography involves estimating the source function $f_s(\Omega)$; later 
we compare the inverted source functions for each model.  
 We use 365 shots for the inversion 
with a shot interval $\Delta s=0.09375$ km. The data-domain wavefield tomography workflow is common for the two inversions.

~\rfg{vel_fwi_deldepth} shows the data-domain wavefield tomography model built from \rfg{veldepth}. The data-domain wavefield tomography process slows 
the velocity in the shallow part of the section,
close to the water bottom, introducing a sharp discontinuity in the model. In general, 
the velocity slows down in the right part of the model. One can see in the gathers, \rfg{xang-vel_fwi_deldepth},
 that in the shallow part the events get flatter with the new velocity. However, deep in the section,
 the model does not correct most kinematic problems exhibited in the gathers. This area of the model
corresponds to longer travel-times in the data, and these late arrivals are prone to cycle-skipping problems. 

We generate the model in \rfg{veldepth_iter06} using the image-domain wavefield tomography approach.
The idea of this tomographic step is to correct for the bulk of the kinematic errors in the model. 
\rfg{veldepth_iter06} shows that in the updated model, in general the model slows down, especially in the
deep part of the section. \rfg{RTM-veldepth_iter06} shows the corresponding RTM image, 
where the focusing of the image improves significantly around $z=3.5$ km.
 This observation is confirmed in \rfg{xang-veldepth_iter06}, where now the gathers are flatter throughout the section.

Finally, we update the image-domain wavefield tomography model with the data-domain wavefield tomography approach.~\rfg{vel_fwi_deldepth_iter06_2} depicts 
the updated model (compare with~\rfg{vel_fwi_deldepth}), which
changes considerably in the interval $z=4$ km to $z=6$ km. 
\rfgs{RTM-vel_fwi_deldepth_iter06_2}-\ref{fig:xang-vel_fwi_deldepth_iter06_2}
are the corresponding RTM image and angle gathers, respectively. 
Note that even though the velocity does not significantly vary the kinematics of the experiment, 
it introduces subtle structural features in the image. We can see that the structure of the line
becomes flatter with the new model (see for instance the event at $z=4$ km and $x=18 \text{ to } 24$ km). 


\multiplot{3}{veldepth_iter06,RTM-veldepth_iter06,xang-veldepth_iter06}{width=\textwidth}{%
(a) The image-domain wavefield tomography velocity model built from \rfg{veldepth}, (b) corresponding RTM image, and (c) angle-gathers at sparse locations. 
The velocity ranges from 1.5km for water velocity to 4km/s in the deepest part.}

\multiplot{3}{vel_fwi_deldepth_iter06_2,RTM-vel_fwi_deldepth_iter06_2,xang-vel_fwi_deldepth_iter06_2}{width=\textwidth}{%
(a) The data-domain wavefield tomography velocity model built from \rfg{veldepth_iter06}, (b) corresponding RTM image, and (c) angle-gathers at sparse locations. %
The velocity ranges from 1.5km/s for water velocity to 4km/s in the deepest part.}

