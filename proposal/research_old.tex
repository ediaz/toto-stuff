

\section{1-Understanding reverse time migration backscattering: noise or signal?}

The conventional imaging condition ~\citep{Claerbout:1985:IEI:3887}
is a zero time lag cross-correlation between the source wavefield and the 
receiver wavefields:
%
\beq
\R=\sum_{shots} \sum_{t} \US(\xx,t)\UR(\xx,t),
\label{eq:cic2}
\eeq
%
which honors the single scattering or Born assumption. Under this assumption the forwardscattered 
source wavefield generates secondary waves as it interacts with the medium discontinuities. These secondary waves propagate in 
space and are recorded at the surface. This assumption means that both the source and receiver
wavefields carry only transmitted energy through interfaces between layers with different elastic properties.

A wavefield extrapolated with RTM could show, depending on the complexity of the geology, waves traveling in both
upward and downward directions, such as diving waves, head waves and backscattered waves. The interaction between
 these waves contained in the source and receiver wavefields generates new events in the image which are 
commonly referred to as artifacts because they do not follow the geology (i.e., earth reflectivity), which is the objective 
of the imaging process. The correlation between forward and backscattered waves is particularly strong when
sharp boundaries are present in the velocity model (e.g., salt bodies).

If a sharp boundary is present in the model, we can decompose the source wavefield into forward scattered 
 and backscattered energy that originates at the sharp boundary:

\beq
\US(\xx,t)= \USr(\xx,t) +\USt(\xx,t),
\label{eq:ssplit}
\eeq
%
where the superscripts $^b$ and $^f$ stand for backscattered and forwardscattered wavefields, respectively. 

The same idea can be applied to the receiver wavefield:

\beq
\UR(\xx,t)= \URr(\xx,t) +\URt(\xx,t).
\label{eq:rsplit}
\eeq
%
By taking advantage of the linearity of equation~\ref{eq:cic2},  we
can split the conventional imaging condition as follows:

\begin{align}
\nonumber \R &= \Rc{ff} +\Rc{bb}\\
             &+ \Rc{bf}+ \Rc{fb}.
\label{eq:cicsplit}
\end{align}
%
Here, the first superscript is associated with the source wavefield and the second with the
receiver wavefield. For example, $\Rc{fb}$ is an image constructed with the forward scattered source wavefield
and the backscattered receiver wavefield.

By analyzing the individual contributions to the image, we can better understand how the backscattered events
are constructed in the image. This analysis is similar to the one of~\cite{fei:3130} and~\cite{liu:S29}
whose objective is to filter out the non-geological portions of the image. Here, we approach the problem 
in a broader sense by attempting to understand the physical meaning of the backscattered energy and its
uses for velocity model building.



\subsection{Extended imaging condition}

The extended imaging condition~\citep{rickett:883,sava:S209,GPR:GPR888} is similar to the conventional imaging condition
 except the cross-correlation lags between source and receiver wavefield are preserved in the output:
\beq
\Re= \sum_{shots} \sum_{t} \US(\xx - \hh,t-\tau) \UR(\xx+\hh,t+\tau).
\label{eq:eic}
\eeq
Here $\hh$ and $\tau$ represent the space-lags and time-lags, respectively, of the cross-correlation.
%
The conventional image is a special case of the extended image $\R=R(\xx,{\bf 0},0)$.

Using extended images allows one to measure the accuracy of the velocity model by analyzing the moveout of the events
 \citep{yang:S151}, and then to perform transformations from the extended to the angle domain 
\citep{sava:1065,sava:S209,sava:S131}. The extended images provide a measurement of the similarity between the source
 and receiver wavefields along space and time, so one can exploit these images to analyze and better understand the RTM backscattered events.

In equation~\ref{eq:eic} one can note an increase in the dimensionality of the image, from 3 to 7 dimensions,
if we decide to extend the image in all directions. It is common to perform the analysis 
of extended images at limited locations in order to make this methodology feasible for large datasets. 
For cost considerations, often one extension is used for common image gathers (CIG), for instance
 the time-lag axis ($\tau$) or the space-lag axis ($\lambda_x$). One can also consider 
 common image point gathers (CIP), where we 
fix an observation point ${\bf c}=(x,y,z)$ and analyze the image as a function of extensions $\hh,\tau$. If the dip is known,
not all the space extensions $\hh$ are needed.  

In the presence of sharp velocity interfaces we can use the concept of equation~\ref{eq:cicsplit}, and construct four partial 
extended images:

\beq
\Re = \Rce{ff}+\Rce{bb} +\Rce{fb}+\Rce{bf}.
\label{eq:eicsplit}
\eeq


\inputdir{paperI}

\subsection{Main results}

\multiplot{3}{cit05_ref1,citx05_ref1,cip05_ref1,img05_ref1}{height=0.24\textheight}{%
Synthetic model example: (a) time-lag gather at x=5km, (b) space-lag %
 gather at x=5km, (c) common image point at x=5km, z=1.5km and (d) %
 migrated image of one shot (in x=5km, z=0km) with receivers %
 in the surface} 

~\rfgs{cit05_ref1} to~\ref{fig:cip05_ref1} show a time-lag gather, a space-lag gather,
 and a common image point, respectively, which represent subsets at fixed surface positions (for CIGs) or fixed space positions 
(for CIPs). Despite the fact that the model has only one reflector, one can identify several events in the conventional and extended images. 
Letter ``a" indicates backscattered events, letter ``b" indicates the events produced by the cross-correlation of 
reflected wavefields, and letter ``c" indicates the cross-correlation between forward scattered wavefields.

In the paper by  \cite{DiazRTM} (corresponding to this chapter) we show the sensitivity
of the backscattered energy to errors in the background model by constructing different
objective functions which measure the backscattered energy as a function of the velocity 
error. Based on the developed analysis, we were able to describe all the events present
in the extended image arising from the salt boundary. 


\newpage
\section{2-Wavefield tomography using RTM backscattering}

In order to analyze the velocity model error, one can make use of the semblance
 principle which seeks image consistency as a function of extended image 
parameters.~\cite{stork} implements the idea using the consistency between common 
offset images.~\cite{symes.carazzone} exploit this concept using the differential
 semblance optimization (DSO) method. DSO can also be used to increase the 
flatness of angle gathers.~\cite{rickett:883} and \cite{sava:1065} show that
 common angle gathers and extended images are related by a slant stacking
 operation. Therefore, these two type of common image gathers are equivalent for
 velocity analysis.~\cite{shen:VE49} and~\cite{Yang2011} use the consistency 
criterion in extended images to formulate a tomographic problem based on 
space-lag gathers or joint space and time-lag gathers, respectively.

\subsection{Inversion with time-lag gathers}
If the velocity is correct, the time-lag gathers~\citep{sava:S209} show maximum
 focusing at zero lag. This observation derives from the fact that the source and
 receiver wavefields are synchronized at the reflector position. 
The velocity model can be improved by increasing the wavefield synchronization, 
which is equivalent to locating the events in the extended images as close as
 possible to $\tau=0$. This can be done by minimizing the following objective 
function (OF):
\beq
J = \frac{1}{2} \norm{P(\tau)R(\xx,\tau)}^2,
\label{eq:tof}
\eeq
where $P(\tau)=|\tau|$ is an operator that penalizes the energy outside $\tau=0$.
Following the notation in~\cite{Yang2011}, one can express $R(\xx,\tau)$ as
\beq
R(\xx,\tau) = \sum_{e}\sum_t T(-\tau)\US({e},\xx,t)T(+\tau)\UR({e},\xx,t),
\eeq 
where $T(\pm\tau)$ is a time-shift operator applied to the source or receiver 
wavefields. Note that this OF cannot drop to zero completely because 
in the  time-lag gathers, the wavefields correlate  
for all values of $\tau$. Here I am interested in bringing the maximum of the
correlation towards $\tau=0$. However, this OF is minimum when the velocity model is correct
 and most of the  energy in the extended image locates at $\tau=0$.

We compute the gradient of equation~\ref{eq:tof} using the Adjoint 
State Method
 (ASM)~\citep{tarantola,plessix}. The adjoint source with respect to the source 
wavefield for an experiment $e$ is
\beq
g_s(\xx,{e}) = \sum_\tau  T(-\tau) P^2(\tau) R(\xx,\tau) T(-\tau) \UR({e},\xx,t),
\label{eq:gst}
\eeq
 and the adjoint source with respect to the receiver wavefield is
\beq
g_r(\xx,{e}) = \sum_\tau T(\tau) P^2(\tau) R(\xx,\tau) T(\tau) \US({e},\xx,t).
\label{eq:grt}
\eeq
We construct the adjoint state variables by injecting the adjoint sources at the
 gather positions and by extrapolating the wavefields using the adjoint modeling
 operators. The adjoint source wavefield $a_s({e},\xx,t)$ is reconstructed
backward in time, whereas the adjoint receiver wavefield $a_r({e},\xx,t)$ is
 reconstructed forward in time. Using the state and adjoint state variables, the 
gradient with respect to the velocity model is
\begin{align}
\nonumber \nabla J(\xx) = \frac{-2}{v^3(\xx)} \sum_{e} \sum_t &\frac{\partial^2u_s}{\partial t^2}({e},\xx,t)a_s({e},\xx,t)+\\
                          &\frac{\partial^2u_r}{\partial t^2}({e},\xx,t)a_r({e},\xx,t),
\label{eq:gradt}
\end{align}
where $\frac{-2}{v^3(\xx)}\frac{\partial^2}{\partial t^2}$ corresponds to the 
derivative of the modeling operator with respect to the
 velocity model.

In the gradient expression (equation~\ref{eq:gradt}) one expects to correlate state
 and adjoint state wavefields traveling in the same direction which implies that
 the gradient is smooth. However, if backscattering is present in the wavefield,
 we obtain crosstalk producing reflectors in the gradients. The crosstalk in this 
case is generated by the correlation of wavefields traveling in the opposite 
direction. In order to attenuate the crosstalk, we can use a filter that 
preserves the components of wavefields traveling in the same direction and 
eliminates the wavefields traveling in opposite directions. We can find the 
direction of propagation using the approach of~\cite{yoon}, which constructs the 
Poynting vectors ${\bf P}(e,\xx,t)$ using the equation
\beq
{\bf P}({e},\xx,t) \propto \frac{\partial u({e},\xx,t)}{\partial t} \nabla u({e},\xx,t),
\eeq
where $u$ can be either the source or the receiver wavefield. In practice, I use 
the time-averaged Poynting vectors using a Gaussian smoothing over a small 
time window determined by the dominant period of the data
\beq
<{\bf P}({e},\xx,t)>_t = {\bf P}({e},\xx,t) *G(t).
\eeq
 Here, the symbol $*$ denotes convolution, and $G(t)$ is the Gaussian smoothing
 filter. The smoothed Poyting vector contains the propagation information of the most
energetic arrival in the wavefields, which mishandle cases like multipathing.

To keep just the wavefield components traveling in the same direction, I  
 compute a weighting function $W(\theta)$ with 
\beq
\theta(\xx,t) = \cos^{-1}\left(\frac{{\bf P}_s(\xx,t)\cdot {\bf P}_r(\xx,t)}{|{\bf P}_s(\xx,t)||{\bf P}_r(\xx,t)|}\right)
\eeq
such that we preserve the wavefield cross-correlation for which 
${\bf P}_s(\xx,t)\cdot {\bf P}_r(\xx,t)\approx 1$, i.e. when 
the direction of propagation is similar within a given tolerance. 
The weighting function can be designed using a cutoff angle, from which the 
function tapers off smoothly using a Gaussian function with standard deviation
 $\sigma$ this defines the range from which the angles are accepted.
\beq
W(\theta,a,\sigma) = \begin{cases}
       1   &\text{if $0^\circ \leq \theta < a $};  \\
       e^{-(\theta-a)^2/(2\sigma^2)} &\text{if $ a \leq \theta \leq180^\circ$}.
   \end{cases}
\eeq

Based on this filter, I change equation~\ref{eq:gradt} to
\begin{align}
\nonumber \nabla J(\xx) = \frac{-2}{v^3(\xx)} \sum_{e} \sum_t &W(\theta) \frac{\partial^2u_s}{\partial t^2}({e},\xx,t)a_s({e},\xx,t)+\\
                          &W(\theta)\frac{\partial^2u_r}{\partial t^2}({e},\xx,t)a_r({e},\xx,t).
\label{eq:Fgradt}
\end{align}
This new gradient avoids crosstalk and emphasizes wavefields traveling in the 
same direction. This method increases the cost of the correlation step. If the 
smoothing stencil along time axis is short, then efficient options like check-point
access could be used for propagating wavefields and computing the propagation
directions \citep{symes2007reverse}.

\subsection{Inversion with space-lag gathers}
One can also use the information contained in space-lag 
gathers~\citep{rickett:883}. If the velocity model is correct, then space-lag
 gathers focus at $(\lambda_x,\lambda_y)=(0,0)$. If the velocity model is incorrect, the 
gathers contain defocused energy outside $\hh=\bf{0}$. This criterion is used
by~\cite{shen:VE49,Yang2011,Wiktor} to formulate wavefield tomography using
 the OF
\beq
J = \frac{1}{2} \norm{P(\hh)R(\xx,\hh)}^2,
\label{eq:sof}
\eeq
where $P(\hh)=|\hh|$ is a penalty operator. Even with correct velocity, this OF 
does not become zero due to the band-limited nature of the data and due to 
illumination effects~\citep{tony_seg:cwp12}. Nevertheless, this OF provides an
 effective criterion for velocity updating.

We compute the gradient of equation~\ref{eq:sof} using the same workflow as the 
one used for equation~\ref{eq:tof}~\citep{Yang2011}. The adjoint sources are 
defined as
\beq
g_s(\xx,{e}) = \sum_\hh  T(-\hh) P^2(\hh) R(\xx,\hh) T(-\hh) \UR({e},\xx,t)
\label{eq:gsx}
\eeq
for the source side, and
\beq
g_r(\xx,{e}) = \sum_\hh  T(+\hh) P^2(\hh) R(\xx,\hh) T(+\hh) \US({e},\xx,t)
\label{eq:gsx}
\eeq
for the receiver side. Here $T(\pm\hh)$ is a space shifting operator applied to
 the wavefields. The only difference between the time-lag and space-lag gather 
formulation is in the OF and in the computation of the adjoint sources. The 
gradient and adjoint wavefields are computed using the same wave-equation and 
background velocity model as in the case of the time-lag gathers 
(equations~\ref{eq:gradt} and~\ref{eq:Fgradt}).

\subsection{Main results}

Figures~\ref{fig:cmodel-iter000} to \ref{fig:ximgiter014} show an application of the described method
to an inversion excercise from the Sigsbee 2A synthetic model. \rfg{cmodel-iter000}
shows the starting velocity model, \rfg{img_iter000} the corresponding seismic
image, and \rfg{ximgiter000} the corresponding gathers (note the defocusing of 
the energy away from zero lag). \rfg{cmodel-iter014} depicts the updated velocity 
model, \rfg{img_iter014} the corresponding image, and \rfg{ximgiter014} the updated
gathers. Note how after iterations the energy of the gathers is correctly focused 
at zero lag. 
 The setup proposed in this paper leads to increased focusing
of reflected energy above the boundary while at the same time
maximizes the backscattered energy. 

\inputdir{paperII}

\multiplot{6}{cmodel-iter000,img_iter000,ximgiter000,cmodel-iter014,img_iter014,ximgiter014}
{width=0.45\textwidth}%
{RTM backscattering optimization: (a) initial model, (b) RTM image obtained with the initial model,
(c) extended image gathers from the initial model (observe the energy spread away zero offset). 
The bottom row shows the updated (d) velocity model, (e) seismic image, and (f) extended image gathers 
which show better focusing around $\lambda_x=0$. The vertical lines indicate the position of the 
extracted space-lag gathers.}


