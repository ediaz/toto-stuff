\begin{abstract}
Wavefield-based imaging and tomography have proven to be robust technologies
to handle the complexity of recorded seismic wavefields. 
 By definition, wavefields are consistent with the bandwidth of seismic data and this 
fact is beneficial for updating the desired components of the background velocity
model. Aside from the data themselves, the velocity model is the most important
component of seismic imaging. A good image retrieved using an accurate model is
 needed for  interpretation and, moreover, for the correct placement of 
exploration and appraisal wells. 

 In this proposal, I use seismic wavefields extrapolated with two-way
and Marchenko operators for image-domain and data-domain
tomography.
Wavefield-based tomographic methods seek to increase the consistency
between recorded and simulated wavefields. I explore different ways
to manipulate the subsurface model (and hence the redatumed wavefields)
in such a way that the consistency increases.
 In general, I see two main families of methods: image-domain
and data-domain, which differ in the location where the inconsistencies are measured. 
 I use image-domain strategies to compare  the redatumed source  function with the corresponding 
redatumed receiver data inside the earth. In the image-domain,
the complexity of the data is reduced and follows known geological rules.

 I design tomographic methods that improve the focusing (energy concentration) 
 at the image point. In order to increase the focusing, I consider 
the acquisition setup, the data bandwidth, and the velocity model to 
construct an illumination-based criterion for the optimization of 
the common-image-point gathers tomography.
I also evaluate the consistency between modeled and recorded wavefields
at the receiver locations (data-domain) using a local correlation operator, which
 mimics certain desirable characteristics of image-domain methods. The 
local correlation operator better separates the travel time misfit compared
with its global correlation counterpart. 
Finally, I analyze the potential of Marchenko-extrapolated wavefields which
  correctly handle multiple-scattered waves for extended imaging and 
tomography. Including primaries and multiple-scattered waves (surface-related and internal multiples) can increase
the sensitivity of the tomographic operator to the background model. 

\end{abstract}
