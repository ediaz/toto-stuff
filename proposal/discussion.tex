\section{Discussion}
In the previous section we show the imaging results from different velocity models.
 In this section we do a quantitative comparison between the inverted models. 
 \rfgs{vel_fwi_deldepth}, \ref{fig:veldepth_iter06}, and \ref{fig:vel_fwi_deldepth_iter06_2}
 show that the right part of the model, $x>14$ km, does not change significantly.
 However, we can see in both data-domain wavefield tomography models that the shallow part of the right side of the model 
improves the flatness of the gathers at shallow depths (compare \rfg{xang-veldepth} 
with \rfg{xang-vel_fwi_deldepth}).

 The image-domain wavefield tomography model (\rfg{veldepth_iter06})
 does not significantly change the kinematics on the right side of the model.
 We can think of two reasons for this observation: (i) the right side of 
the model has poorer illumination, which can be confirmed by the limited
angle range in the gathers, and (ii) given that the right side of the section has a shallower 
water column, then we can expect several orders of surface related multiples. We can address (i)
by relaxing the mute in the input shot gathers and thus improving the illumination.
 Another option is to use a penalty operator $P(\hh)$ that 
takes into account the spatially-variable illumination of the data, as suggested by \cite{tony_seg:cwp12}.
  In relation to point (ii), the presence of multiples (surface-related or internal)
 violates the implicit single scattering assumption of conventional and extended images.
 \cite{Wiktor} suggest creating data that conform
to single scattering by muting the multiples in the extended images and then
demigrate the gathers. This new dataset should remove the bias of the surface
related multiples in the inversion. Another option is to 
demultiple the data prior to inversion (e.g. SRME \citep{SRME}). All these improvement ideas
remain for ongoing and future tests.

On the left side of the section we see significant changes. 
\rfgs{veldepth-left} to \ref{fig:vel_fwi_deldepth_iter06_2-left} show a detailed view
of the models for $x<14$ km. The data-domain wavefield tomography model in \rfg{vel_fwi_deldepth_iter06_2-left},
 built from from the initial model, shows some layering below the water bottom, where we can 
see a clear boundary in the model that is probably related to the events 
ranging from $z=1.5$ km and $z=2.5$ km. Given that the velocity is too fast,
 the data-domain wavefield tomography model is probably trapped in a local minimum. Hence, it cannot correct
for the kinematic errors in the model. This is confirmed in the moveout of the 
gathers, shown in \rfgs{xang-veldepth-left}-\ref{fig:xang-vel_fwi_deldepth-left}.
 There are not many differences to recognize from \rfgs{RTM-veldepth-left}-\ref{fig:RTM-vel_fwi_deldepth-left}. This confirms
 that the data-domain wavefield tomography model did not alter the kinematics. 

In contrast, when we compare previous models with the image-domain wavefield 
tomography model, \rfg{veldepth_iter06-left}, we can appreciate a considerable correction
 to the velocity. Now, the slower velocity corrects for the bulk of the kinematic
errors in the model. \rfg{xang-vel_fwi_deldepth-left} shows flat events up 
to $z=4.5$ km through the detailed section. The new velocity highlights the 
 unconformity depicted by the bright seismic event around $z=4$ km. Also, we can see
how new events get imaged between $z=2.5$ km and $z=4$ km in \rfg{RTM-veldepth_iter06-left}. 
 The updated model from data-domain wavefield tomography, depicted in \rfg{vel_fwi_deldepth_iter06_2-left}, 
now shows a sharp discontinuity in the velocity at $z=3$ km. The corresponding
 image, \rfg{RTM-vel_fwi_deldepth_iter06_2-left}, shows a flatter structure 
after the data-domain wavefield tomography update. This is interesting because we can see how despite the 
added complexity in the velocity, the structure in the image is simplified.

\rfgs{source_veldepth}-\ref{fig:source_veltomo_fwi} show the inversion of
source functions for the initial, the data-domain wavefield tomography from the initial model, the image-domain wavefield tomography model, and 
the final data-domain wavefield tomography model, respectively. Note that the source functions inverted
with the smooth model are laterally consistent. However, the consistency is improved
 in \rfg{source_veltomo}. If we compare the source inversions from data-domain wavefield tomography models 
(\rfgs{source_veldepth_fwi}-\ref{fig:source_veltomo_fwi}), we can see 
a higher lateral correlation, which confirms that the final data-domain wavefield tomography models 
better explain the kinematics of the data for direct and diving wave arrivals. 
Even though we invert each source individually, we use the average over source positions for the inversion.
 This is done because we know that in the field the air gun is shot with a constant pressure. Hence, 
we assume that the inconsistencies
 in the source functions come from the model itself. 

Analyzing the focusing in space-lag gathers, or flatness of angle gathers, is the proper quality control tool for 
image-domain methods. The equivalent tool for data-domain methods are the data residuals.
 \rfgs{fdiff-veldepth}-\ref{fig:fdiff-veldepth_iwt_fwi} show the time-domain data residuals for the four models
discussed in this report for a shot gather at $x=18.75$ km. \rfg{fdiff-veldepth} shows the data residual 
corresponding to \rfg{veldepth}. One can observe large amplitude and phase residuals trough the diving-waves 
components of the data. After updating the model the residual depicted in \rfg{fdiff-veldepth_fwi} shows
that the diving waves arrivals are better fit, specially between offsets $2.5\text{ to }5$ km. 
 \rfg{fdiff-veldepth_iwt} shows the residual corresponding to \rfg{veldepth_iter06}, we can see that 
the these residuals better explain the data than those in \rfg{fdiff-veldepth}. After updating the model,
 \rfg{fdiff-veldepth_iwt_fwi} we can see how the residuals from \rfg{vel_fwi_deldepth_iter06_2} better
fit the data than any of the previous models. Now, the direct arrivals have a good match for near
and intermediate offsets.


\inputdir{rtm}
\multiplot{4}{veldepth-left,vel_fwi_deldepth-left,veldepth_iter06-left,vel_fwi_deldepth_iter06_2-left}
{width=0.65\textwidth}{%
Detail from (a) initial velocity, (b) data-domain wavefield tomography built with initial velocity, %
(c) image-domain wavefield tomography velocity model, and (d) data-domain wavefield tomography model built with image-domain wavefield tomography model.}

\multiplot{4}
{xang-veldepth-left,xang-vel_fwi_deldepth-left,xang-veldepth_iter06-left,xang-vel_fwi_deldepth_iter06_2-left}
{width=0.65\textwidth}{%
Detail from angle gathers from (a) initial velocity, (b) data-domain wavefield tomography built with initial velocity, %
(c) image-domain wavefield tomography velocity model, and (d) data-domain wavefield tomography model built with image-domain wavefield tomography model.}

\multiplot{4}{RTM-veldepth-left,RTM-vel_fwi_deldepth-left,RTM-veldepth_iter06-left,RTM-vel_fwi_deldepth_iter06_2-left}
{width=0.65\textwidth}{%
Detail from RTM images from (a) initial velocity, (b) data-domain wavefield tomography built with initial velocity, %
(c) image-domain wavefield tomography velocity model, and (d) data-domain wavefield tomography model built with image-domain wavefield tomography model.}

\inputdir{source_inv}
\multiplot{2}{source_veldepth,source_veldepth_fwi,source_veltomo,source_veltomo_fwi}
{width=0.72\textwidth}{ Source inversions as a function of source position: (a) from initial model, (b) from data-domain wavefield tomography using initial model,
 (c) from image-domain wavefield tomography model, and (d) data-domain wavefield tomography from image-domain wavefield tomography model.}


\inputdir{data_comp}
\multiplot{4}{fdiff-veldepth,fdiff-veldepth_fwi,fdiff-veldepth_iwt,fdiff-veldepth_iwt_fwi}
{height=0.45\textheight}{%
Data domain residuals for shot position $x=18.75$ km using: (a) the inital velocity model, (b) the data-domain model built
from the inital model, (c) the image-domain model, and (d) the data-domain model built from the image-domain model.}
