\section{Conclusions}
The combination between image-domain and data-domain wavefield
tomography seeks to exploit the features of each method. The image-domain 
wavefield tomography methods are sensitive to the smooth components
of the model due to the definition of the inverse problem. Once we obtain
 a smooth model that improves focusing in the extended images,
 we can proceed to further refine the model using data-domain
 wavefield tomography. We demonstrate the cascaded workflow using
 a real 2D marine dataset. Our image-domain wavefield tomography
model corrects for most kinematic errors in the model, 
whereas
the data-domain wavefield tomography model 
corrects early arrival phase errors in the data, and
introduces  discontinuities in the model directly correlated with events in 
the image.

%y can be used as an effective 
%and consistent method with the bandlimited nature of seismic
%data. The combination between image-domain and data-domain
%wavefield tomography approaches can be thought as a model
%continuation approach. With image-domain tomography we seek
%to update the smooth components of the model that are able
%to produce well-focused images. With the data-domain approach
%we can further refine the model by updating the details and 
%short-wavelength components of the model.
%
%We apply this cascaded workflow to a real 2D marine dataset. 
%Our image-based tomography model corrects most kinematic errors in the left side 
%of the model. The data-domain update corrects for the
%shallow part of the model and introduces a sharp 
%discontinuity, directly correlated with the events around
%$z=2$km. 
%
