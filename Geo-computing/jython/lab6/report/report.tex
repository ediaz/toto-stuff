\documentclass[10pt]{article}
\usepackage{amsmath, amsthm, amsbsy, rotating,float}
\usepackage{graphicx,algorithm,algorithmic,subfigure}
\usepackage{setspace,enumerate}
\doublespacing




\author{Esteban D\'{i}az}
\title{Gaussian smoothing filters}{}

\begin{document}
\maketitle

In this report we discuss two different filters that
apply a Gaussian: one is exact and implemented via FFT's, and
the other is a good approximation with recursive exponential smoothing.

\section{Gaussian smoothing with Exponential Filters}
In this approach we apply a cascade of exponential filters to 
approximate a Gaussian. This type of filter will converge 
to the exact Gaussian as $n \rightarrow \infty$, where $n$ is 
the number of applied exponential filters.

The impulse response of a Gaussian filter is given by:
\begin{equation}
  H_G(\omega) = e^{-\frac{\sigma^2\omega^2}{2}}
\end{equation}
One can cascade several exponential filters such that:

\[
H_G(\omega) = e^{-\frac{\epsilon^2\omega^2}{2}} e^{-\frac{\epsilon^2\omega^2}{2}}\dots e^{-\frac{\epsilon^2\omega^2}{2}}
\]

Therefore, $\epsilon^2 = \frac{\sigma^2}{n}$. Now, by approximating each individual
filter to an exponential, for the low frequencies we have:

\[
\frac{a}{(1-a)^2}=\frac{\epsilon}{2} \implies a =\frac{\epsilon^2 +1 -\sqrt{2\epsilon^2+1}}{\epsilon^2}
\]
One has to pick the negative root to ensure $a<1$. Now we have the connection between 
exponential and Gaussian smoothing. The bigger the number of smoothing steps, the better
the approximation.

Figure~\ref{fig:fig1} shows the 1D impulse response of the approximation for different number
 of smoothing. One can see that $n=1$ results in a exponential smoothing response.
Although several $n$ seems to nicely approximate the Gaussian, the detail in Figure~\ref{fig:fig2}
shows that only $n=200$ is very close to the exact solution. However, in practice $n=200$ is not
feasible. One has to make a compromise between efficiency and precision.


The 2D impulse response of this filter is quite interesting. One can see in Figure~\ref{fig:fig5}(a)
that the filters is very anisotropic for few smoothing steps. For $\sigma=30$, 16 smoothing steps 
seems to do a fair approximation to the Gaussian shown in Figure~\ref{fig:fig5}(b) 

\section{2D Gaussian smoothing via FFT's}

In the frequency domain the implementation is straight forward. I use the Fft routines
provided in the Mines Java Toolkit (FftReal). When working with FFT's one has to be careful
to avoid wraparound effect due to the sampling of continuous signals (which implies periodicity in 
time and frequency). 

The impulse response of the filter (equation 1) goes to zero after $3\sigma$. Therefore, a padding 
of $3\sigma$ elements should be enough to avoid the periodicity effect in the Fourier domain.
For zero slope boundary condition one might need $6\sigma$, which is $3\sigma$ in both sides of
the array. The padding in this case involves a constant extrapolation, whereas in the zero boundary
one just pad with zeros.


In Figure~\ref{fig:fig4}, I compare the output of the Gaussian filter with zero (a) and zero slope (b)
boundary conditions. In the first one the boundaries are contaminated with zeros,
 whereas the second is well behaved at the boundaries.

One can see that the speed of this filter will decrease as the length of the filter ($\sigma$) increases.

\section{Benchmark}
For benchmarking I count how many smooths per second the filters perform. I test with three images of 
different size.

The performance of the exponential filter depends linearly with the number of smooths. The boundary 
condition on the Gaussian has little influence  on the performance (the zero boundary is slightly
faster).

For small arrays (100,101) the Gaussian smoothing is much faster than the exponential. For 
small arrays the FFT cost is probably negligible.

I also tested the dependency of the cost with $\sigma$ for the Gaussian filter. It seems
to have little influence.

Looking at the impulse responses in Figure~\ref{fig:fig5}(a) show that $n=16$ looks isotropic.
For this $n$ only for big arrays the exponential performs better.

\begin{verbatim}
(n1,n2)=(100,101)
exponential n=8 benchmarking rate=  814.170527027 smooths/s
exponential n=16 benchmarking rate=  447.303203587 smooths/s
exponential n=32 benchmarking rate=  220.821752681 smooths/s
=================================================================
Gaussian zb sigma=10 benchmarking rate=  1454.05663002 smooths/s
Gaussian zs sigma=10 benchmarking rate=  1394.85130885 smooths/s
Gaussian zb sigma=40 benchmarking rate=  1570.46336969 smooths/s
Gaussian zs sigma=40 benchmarking rate=  1430.70500119 smooths/s
Gaussian zb sigma=200 benchmarking rate=  1550.9751844 smooths/s
Gaussian zs sigma=200 benchmarking rate=  1412.54685497 smooths/s
(n1,n2)=(200,201)
exponential n=8 benchmarking rate=  531.935529414 smooths/s
exponential n=16 benchmarking rate=  266.942126947 smooths/s
exponential n=32 benchmarking rate=  130.987975304 smooths/s
=================================================================
Gaussian zb sigma=10 benchmarking rate=  716.188254513 smooths/s
Gaussian zs sigma=10 benchmarking rate=  558.075549153 smooths/s
Gaussian zb sigma=40 benchmarking rate=  712.955915626 smooths/s
Gaussian zs sigma=40 benchmarking rate=  606.281775054 smooths/s
Gaussian zb sigma=200 benchmarking rate=  707.307767067 smooths/s
Gaussian zs sigma=200 benchmarking rate=  585.721981832 smooths/s
(n1,n2)=(500,501)
exponential n=8 benchmarking rate=  197.525848796 smooths/s
exponential n=16 benchmarking rate=  94.5145210671 smooths/s
exponential n=32 benchmarking rate=  48.3296417076 smooths/s
=================================================================
Gaussian zb sigma=10 benchmarking rate=  156.71568696 smooths/s
Gaussian zs sigma=10 benchmarking rate=  155.770715106 smooths/s
Gaussian zb sigma=40 benchmarking rate=  164.336106122 smooths/s
Gaussian zs sigma=40 benchmarking rate=  153.102749134 smooths/s
Gaussian zb sigma=200 benchmarking rate=  165.487336222 smooths/s
Gaussian zs sigma=200 benchmarking rate=  157.36147791 smooths/s
(n1,n2)=(1000,1001)
exponential n=8 benchmarking rate=  62.2771396589 smooths/s
exponential n=16 benchmarking rate=  30.5387697666 smooths/s
exponential n=32 benchmarking rate=  15.09491224 smooths/s
=================================================================
Gaussian zb sigma=10 benchmarking rate=  40.6226091902 smooths/s
Gaussian zs sigma=10 benchmarking rate=  43.3885888011 smooths/s
Gaussian zb sigma=40 benchmarking rate=  37.5071697549 smooths/s
Gaussian zs sigma=40 benchmarking rate=  40.189959235 smooths/s
Gaussian zb sigma=200 benchmarking rate=  41.642366984 smooths/s
Gaussian zs sigma=200 benchmarking rate=  39.9016185693 smooths/s
\end{verbatim}




\begin{figure}[H]
    \centering
    \includegraphics[width=0.7\textwidth]{png/Exp_Gauss.png}
    \caption{Comparisson between exponential with different number of %
             smoothings and Gaussian filters:%
             red: 1, green: 5, blue: 10, cyan: 20, magenta: 50, %
             black: 200, yellow: Gaussian.}
    \label{fig:fig1}
\end{figure}

\begin{figure}[H]
    \centering
    \includegraphics[width=0.7\textwidth]{png/detail_Exp_Gauss.png}
    \caption{Detail of Fig~\ref{fig:fig1}, same color coding. }
    \label{fig:fig2}
\end{figure}

\begin{figure}[H]
    \centering
    \includegraphics[width=0.7\textwidth]{png/Exp_2d_rand.png}
    \caption{Exponential filter with 8 repetitions %
            applied to a random field, $\sigma=5$}
    \label{fig:fig3}
\end{figure}


\begin{figure}[ht!]
     \begin{center}
%
        \subfigure[Gaussian smoothing, zero boundary]{%
           \label{fig:second}
           \includegraphics[width=0.7\textwidth]{png/Gauss_2d_rand_0b.png}
        }\\ %  ------- End of the first row ----------------------%
        \subfigure[Gaussian smoothing, zero slope boundary]{%
            \label{fig:third}
            \includegraphics[width=0.7\textwidth]{png/Gauss_2d_rand_0sb.png}
        }%
%
    \end{center}
    \caption{%
        Comparisson between boundary conditions for the Gaussian
        filter applied to the same random field of Figure~\ref{fig:fig3}, $\sigma=5$.
     }%
   \label{fig:fig4}
\end{figure}


\begin{figure}[ht!]
     \begin{center}
%
        \subfigure[]{%
           \label{fig:second}
           \includegraphics[width=0.7\textwidth]{png/Exp_2d_ir.png}
        }\\ %  ------- End of the first row ----------------------%
        \subfigure[]{%
            \label{fig:third}
            \includegraphics[width=0.7\textwidth]{png/Gauss_2d_ir_0b.png}
        }%
%
    \end{center}
    \caption{%
     2D Impulse response for exponential filter (a): 1, 2, 4, 8, 16, 32, 48, 64 smooths. 
      2D Impulse response for Gaussian filter (b), $\sigma=30$.
     }%
   \label{fig:fig5}
\end{figure}







\end{document}
