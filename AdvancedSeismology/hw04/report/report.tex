\documentclass[10pt]{article}
\usepackage[margin=1in]{geometry}
\usepackage{amsmath, amsthm, amsbsy, rotating,float}
\usepackage{graphicx,algorithm,algorithmic}
\usepackage{setspace,enumerate}

\usepackage{graphicx,caption,subfigure}

\floatstyle{ruled}
\newfloat{program}{thp}{lop}
\doublespacing

\author{Esteban D\'{i}az}
\title{Ray tracing in VTI media}{}
\begin{document}

\maketitle

A ray tracing code for homogeneous VTI media is built.
I solved the transmission problem numerically at the interface:
\[
  \min_{\theta_2 \in (-90,90)} \left|\frac{\sin(\theta_1)}{V_1(\theta_1)} -\frac{\sin(\theta_2)}{V_2(\theta_2)}\right|
\]
where $V_1$ and $V_2$ are the phase velocity on the first layer and second layer, respectively. The angles $\theta_1$ and $\theta_2$ are the phase angles for the respective layer. 

Once we know the phase angle we can solve for the group angle, and compute the ray trajectory and its travel time.


For the P waves, the traveltimes at the surface increase slightly  for $\epsilon_2=-0.2$. For a negative $\epsilon$
the P waves travel slowly close to the horizontal direction, therefore is expected that the traveltime should
increase at long offsets.

For explaining the traveltime differences for SV waves we can take a look at the magnitude of $\sigma$ for both 
cases. In the first case:
    \[
     \sigma_2 = \left(\frac{V_{P0}}{V_{S0}}\right)^2(\epsilon-\delta) = -0.4.
    \]
In the second case $\sigma_2 = -2$. This value for $\sigma$ is quite large and is the main reason 
for the cusp in the traveltimes seen in Figure~\ref{fig:tt_slow_eps-2}. 

For the first case, with $\sigma_2$ relatively small, we see how the group angle decreases at the interface, this
is due to the fact that $V_{S0}$ decreases in the second layer. 




\begin{figure}
	\begin{center}
		\subfigure[]{
			\includegraphics[height=0.4\textheight]{Fig/tt_slow_eps2.png}
      \label{fig:tt_slow_eps2}
		} 
		\subfigure[]{
			\includegraphics[height=0.4\textheight]{Fig/rays_slow_eps2.png}
      \label{fig:rays_slow_eps2}
		} 
		\caption{SV-waves traveltimes (a) and ray plots (b) for $\epsilon_2=0.2$}	
  \end{center}
\end{figure}

\begin{figure}
	\begin{center}
		\subfigure[]{
			\includegraphics[height=0.4\textheight]{Fig/tt_slow_eps-2.png}
      \label{fig:tt_slow_eps-2}
		} 
		\subfigure[]{
			\includegraphics[height=0.4\textheight]{Fig/rays_slow_eps-2.png}
      \label{fig:rays_slow_eps-2}
		} 
		\caption{SV-waves traveltimes (a) and ray plots (b) for $\epsilon_2=-0.2$}	
  \end{center}
\end{figure}


\begin{figure}
	\begin{center}
		\subfigure[]{
			\includegraphics[height=0.4\textheight]{Fig/tt_fast_eps2.png}
      \label{fig:tt_fast_eps2}
		} 
		\subfigure[]{
			\includegraphics[height=0.4\textheight]{Fig/rays_fast_eps2.png}
      \label{fig:rays_fast_eps2}
		} 
		\caption{P-waves traveltimes (a) and ray plots (b) for $\epsilon_2=0.2$}	
  \end{center}
\end{figure}

\begin{figure}
	\begin{center}
		\subfigure[]{
			\includegraphics[height=0.4\textheight]{Fig/tt_fast_eps-2.png}
      \label{fig:tt_fast_eps-2}
		} 
		\subfigure[]{
			\includegraphics[height=0.4\textheight]{Fig/rays_fast_eps-2.png}
      \label{fig:rays_fast_eps-2}
		} 
		\caption{P-waves traveltimes (a) and ray plots (b) for $\epsilon_2=-0.2$}	
  \end{center}
\end{figure}

\end{document}
