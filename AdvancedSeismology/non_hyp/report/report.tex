\documentclass[10pt]{article}
\usepackage[margin=1in]{geometry}
\usepackage{amsmath, amsthm, amsbsy, rotating,float}
\usepackage{graphicx,algorithm,algorithmic}
\usepackage{setspace,enumerate}

\usepackage{graphicx,caption,subfigure}

\floatstyle{ruled}
\newfloat{program}{thp}{lop}
\doublespacing

\author{Esteban D\'{i}az}
\title{Nonhyperbolic moveout for P-waves reflection data}{}
\begin{document}

\maketitle

In this homework we invert for effective parameters
for the traveltime approximation with quartic terms (equation 7.17 from 
the book). 

For the first approach (inverting for C) we assume we know the effective
velocities. The L2 error plot is shown in Figure 1. The minimum
value corresponds to $C=1.155$. The obtained traveltime curve is very
close to the observed one, this is why I plot the difference instead
(blue curve in Figure 3). 

For the second approach we do not assume any known velocity and search
for Vnmo and Vhor. The error in this 2D space search is shown in  
Figure 2. The minimum corresponds to $Vnmo=2.088km/s$ and $Vhor=2.512km/s$.

The retrieved traveltime corresponds to the red curve in Figure 3. 

The tuning parameter C could be worthwhile looking if we know
the medium velocities. However, we do not have that luxury very often. 
The second approach resulted more effective and yields to very good 
traveltimes.

Plugging in the inverted velocities from the second approach results
in $\eta_{eff} = 0.2237$ which is very similar to the
real effective one $\eta(N) = 0.24474$. For P-wave kinematics 
the long offset traveltime is dominated by $\eta$, so as long
as we preserve this parameter we should obtain a
 very good approximation.


\begin{figure}
	\begin{center}
		\includegraphics[height=0.4\textheight]{Fig/Error_C.pdf}
  \end{center}
  \caption{L2 error as a function of parameter C}
\end{figure}

\begin{figure}
	\begin{center}
		\includegraphics[height=0.4\textheight]{Fig/Contour_L2_plot.pdf}
  \end{center}
  \caption{Contour error plot for searching for effective Vhor, Vnmo}
\end{figure}

\begin{figure}
	\begin{center}
		\includegraphics[height=0.4\textheight]{Fig/Error_traveltimes.pdf}
  \end{center}
  \caption{Differences between inverted traveltimes and%
          observed traveltimes. The red curve corresponds to the search%
          for Vhor and Vnmo. The blue curve comes from the search for C}
\end{figure}


\end{document}
