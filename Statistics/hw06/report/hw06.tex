\documentclass[10pt]{article}
\usepackage{amsmath, amsthm, amsbsy, rotating,float}
\usepackage{graphicx,subfig}
\usepackage{setspace,enumerate}
\doublespacing




\usepackage{algorithmic}





\def\der#1#2#3{
\frac{\partial #1}{\partial #2}\bigg|_{#3}
}
\def\ders#1#2{
\frac{\partial #1}{\partial #2}
}

\def\dters#1#2{
\frac{\partial^2 #1}{\partial #2^2}
}

\author{Esteban D\'{i}az}
\title{Homework 6}{}

\begin{document}
\maketitle

\section{Problem 1}

\subsection{Find an unbiased estimator of $\mu$ and compute its variance}
The sample mean is an unbiased estimator of each $X_i$. Therefore:
\[
  \bar{X} =  \frac{1}{2} (\bar{x}_1 +\bar{x}_2) -\bar{x}_3 = -1.2
\]

The variance can be approximated by combining the sample variances:
\[
  S^2 = \frac{1}{4} \left(S_1^2 +S_2^2\right) +S_3^2 =0.3988 
\]


\subsection{Compute 95\% confidence interval for $\mu$}
The confidence interval is given by:

\[
   \bar{X} \pm Z_{\alpha/2}\sqrt{\frac{1}{4} \left(\frac{S_1^2}{n1} +\frac{S_2^2}{n2}\right) +\frac{S_3^2}{n3}} = (0.00246, 0.049566)
\]

\subsection{Is there any evidence that pesticide \#3 is better than the others?}
The confidence interval is slightly positive. Therefore, one can say that pesticide
\#3 is almost as good as the other two. Possibly the three pesticides result in a 
similar outcome.



\section{Problem 2}

\subsection{normality of the data}

\begin{figure}[H]
    \centering
    \includegraphics[width=0.85\textwidth]{./lab6.png}
    \caption{QQ plot to test normality}
    \label{fig:fig1}
\end{figure}
As shown in fig~\ref{fig:fig1} the data is not normal. However this is not conclusive due to 
the small sample size ($n=20$).


\subsection{confidence interval for 95\%}

We can construct the confidence interval using a student $t-distribution$ of degree 19:

$ \bar{X} \pm t_{n-1}(0.05/2) \frac{S}{\sqrt{n}}$, this means: $\mu \in (29.07,52.71)$


\section{Problem 3}

The $95\%$ confidence interval for the difference is given by:
\[
\mu_D \in (\bar{D} \pm  t_{n-1}(0.05/2) \frac{S_D}{\sqrt{n}}=(-1.5034, 0.6778)
\]

Here, I assume: $D_i = Ground_i -Sattelite_i$. Therefore,
the result is not conclusive because it contains $0$. However,
 one can see that is more negative, which implies that $Ground_i > Sattelite_i$ is
more likely to happen.

\section{Problem 4: Suppose $X_1,X_2,X_3$ are independent $N(\mu,\sigma^2)$ and $\sigma$ is known} 

\subsection{Write down the usual $95\%$ confidence interval for $\mu$}
Let's suppose that each individual sample has its own number of samples $n_i$. 

First, we need an estimator for $\mu$, given that they come from the same distribution, one could
no an average:

\[
\mu = \frac{\bar{X}_1}{3}+\frac{\bar{X}_2}{3} +\frac{\bar{X}_3}{3}
\]

Then, for a $95\%$ CI we have:
\[
 \mu \in \frac{\bar{X}_1}{3}+\frac{\bar{X}_2}{3} +\frac{\bar{X}_3}{3} \pm Z(0.975)\frac{\sigma}{3}\sqrt{\frac{1}{n_1} +\frac{1}{n_2}+\frac{1}{n_3}}
\]



\subsection{Based on the weighted average $X_w = \frac{1}{8}X_1 + \frac{1}{8}X_2+ \frac{3}{4}X_3$ derive the $95\%$ formula for $\mu$}
In this case the only change would be in the estimated mean:
\[
\mu \in \frac{1}{8}\bar{X}_1 + \frac{1}{8}\bar{X}_2+ \frac{3}{4}\bar{X}_3 \pm Z(0.975)\sigma\sqrt{\frac{1}{64n_1} +\frac{1}{64n_2}+\frac{9}{16n_3}}
\]

\subsection{if both intervals give the $95\%$ CI, in what do they differ? Which of the intervals is better and why?}
Both have a similar assumption (some sort of average for the samples). However, the second could be biased because is putting more weight on
sample $X_3$. Maybe this could make sense if we know something extra about $X_3$ (i.e. more number of samples, more reliable measure, etc).

\section{Problem 7.3 from textbook}

a) Having a margin of error of 1 for the $95\%$ CI implies:
  \[
    Z(0.975) \sigma/\sqrt{n} = 1
  \]
  Therefore, $n = (1.96\sigma)^2 = \lceil96.04\rceil=96$

b) Given that for the estimated sample above the observed mean is $\bar{x}=6.3days$, and $sd = 4.57days$ then the $95\%$ CI is:
  \[
   \mu \in \bar{x} \pm Z(0.975) \sigma\frac{1}{\sqrt{n}} = (5.38583, 7.21417)days
  \]
\end{document}
