\documentclass[10pt]{article}
\usepackage{amsmath, amsthm, amsbsy, rotating,float}
\usepackage{graphicx,subfig}
\usepackage{setspace,enumerate}
\doublespacing




\usepackage{algorithmic}





\def\der#1#2#3{
\frac{\partial #1}{\partial #2}\bigg|_{#3}
}
\def\ders#1#2{
\frac{\partial #1}{\partial #2}
}

\def\dters#1#2{
\frac{\partial^2 #1}{\partial #2^2}
}

\author{Esteban D\'{i}az}
\title{Take home exam}{}

\begin{document}
\maketitle

\section{Problem 1}
\subsection{part I}
For such small dataset, and given that there are no evidence against 
gaussianity, we can use a t-statistic:
\[
 \bar{D} = \mu_x - \mu_y
\]
and:
\[
 \mu_D \in \bar{D} \pm t_{n-1}(\alpha/2)\sqrt{\frac{s_x^2}{6}+\frac{s_y^2}{6}}
\]

\subsection{Part II}
If n and m are very large, we can still use the Gaussian model. However,
given that there is no correlation between the random variables, then
the difference between them can be statistically irrelevant.

\[
 \mu_D \in \bar{D} \pm z_{\alpha/2}\sqrt{\frac{s_x^2}{n}+\frac{s_y}{m}}
\]

\subsection{Part III}
Since the data is non-gaussian, then I would use a non-parametric test (i.e. no 
assumption about the variables). I found that the Wilcoxon signed-rank test
might be suitable for this scenario.

\section{Problem 2}
i) The probability is 
\[P = \frac{2520}{6036}\]

ii) The probability that a victim is a male is 
\[P = \frac{24226}{30604}\]

iii) Using an fire arm and being male has a probability  of:
\[
  P = \frac{15518}{30604}
\]


\section{Problem 3}
\subsection{part I}
Compute the velocity and uncertainty using propagation of errors.
The velocity is given by:
\[
V = x \sqrt{\frac{1}{2NMO t_o}}
\]

Therefore,
\[
\ders{V}{t_o} = -\frac{x}{2} \sqrt{\frac{1}{2NMO}} \left( t_o \right)^{-3/2}
\]
\[
\ders{V}{NMO} = -\frac{x}{2} \sqrt{\frac{1}{2t_o}} \left( NMO \right)^{-3/2}
\]
where $\mathbf{\mu}=(\mu_{to},\mu_{NMO})$
Then, we have:

\[
\sigma_V  = \der{V}{t_o}{\mathbf{\mu}}^2 \sigma_{to}^2 +\der{V}{NMO}{\mathbf{\mu}}^2 \sigma_{NMO}^2
\]

Evaluating the above formulas:
\[
V = 2000.0 \pm 200.0 (m/s)
\]

\subsection{Part II}
We would have to assume a model for each random variable. Then, we can do $n$ simulations, 
and at the end we analyze the properties of $V_i$, by calculating the sample mean and sample
variance.

\section{Problem 4}

\subsection{normality test} 

From the qq-plot in Figure~\ref{fig:qq} the data seems highly correlated with a gaussian. 
I also performed a Shapiro-Wilk test. The results
shows a $W=0.984$ which is very high, this indicates that we cannot reject the null hypothesis
$H_0 : X\sim N(\mu,\sigma)$. Therefore, it is safe to assume that the data is Gaussian.


\begin{figure}[H]
    \centering
    \includegraphics[width=0.85\textwidth]{./qq.png}
    \caption{QQ plot to test normality}
    \label{fig:qq}
\end{figure}

\subsection{Confidence interval}
For this data we have:
\[
  \bar{x} = 54.33 
\]

\[
  s_x = 9.6383
\]
Then,
\[
\mu \in \bar{x} \pm z_{\alpha/2} s_x/\sqrt{24} = 54.3333 +/- 3.8561 mg/m^3
\]
\subsection{Hypothesis test}

Now I will test the null hypothesis $H_0 :\mu \leq \mu_o=50mg/m^3$, with the alternative being $H_1: \mu> 50mg/m^3$.
 
For the test statistics I use $Z$ (normal-score). The rejection region is $Z > Z_\alpha$ 
\[
Z = \frac{\bar{x}-\mu_o}{s_x/\sqrt{n}} = 2.202 
\]

Then, for a significance level $\alpha=0.05$ we have to reject $H_o$ because $Z > Z_\alpha$.

The P-value is given by $P(Z > Z_{\alpha} ) = 0.025$.

With such small $p-value$ I would reject the null.

From these tests it seems very conclusive that the concentration of Sulphur dioxide is 
above the limit for normal values. Such high concentration could generate acid rain.

\section{Problem 5}
The result of the studies are shown in Table~\ref{tab:condom}.
The first test I do is to check consistency among both studies: if both studies are
consistent, then, the difference of the proportions for the same case should contain 
zero, for a certain confidence value $\alpha$. Then, I test the Hypothesis that
there is no correlation between using or not condoms. If this is true, then the CI
should contain the zero as well (for both studies).

\subsection{test I}
Now I want to test that both studies have consistent results:
Let $\hat{p}_1 =$ Proportion of infected Italians that used condoms and
$\hat{p}_2 =$ Proportion of infected Europeans that use condoms, then

\[
ci \in (-0.0021, 0.0372)
\]

Now, let $\hat{p}_1 =$ Proportion of infected Italians that did not used condoms,
and $\hat{p}_2 =$ Proportion of infected Europeans that did not use condoms, then:

\[
ci \in (-0.060023, 0.1542106)
\]

From these results I can conclude that both studies are consistent with each other
with a confidence value $\alpha=0.05$ because both contain zero. 

Now I will compare the individual information from both studies, to see if there 
is any evidence against using condoms as a effective method against HIV infections.

For the Italian study I get:

\[
ci \in (-0.22314, -0.03268)
\]

For the European study I get:
\[
ci \in (-0.151204, -0.0455)
\]

These results indicate that the use of condoms is effective against HIV infection for 
an $\alpha=0.05$ (because the interval does not contain zero).




\begin{table}
  \centering
\begin{tabular}{c | c | c  }
  
                          & Italian     & Eurozone    \\ \hline
Infection with condom     & 3/171       & 0 \\ 
Infection without condom  & 8/55        & 12/122 \\ \hline 
\end{tabular}
\caption{Summary of the studies results}
\label{tab:condom} 
\end{table}


\section{Problem 6}

The ML estimate is $\theta=0.0357123$. Once we have $\theta$ I can proceed
to get the estimates for the counts for the different types. The results
can be summarized in Table~\ref{tab:count}.


\begin{table}
  \centering
\begin{tabular}{c | c | c  }
  
k    & Count     & Estimated count   \\ \hline 
0  & 1997  & 1953.77 \\
1  & 906  & 925.48 \\
2  & 904  & 925.48 \\
3  & 32  & 34.27 \\
\end{tabular}
\caption{Observed and estimated counts for self-fertilized heterozygotes}
\label{tab:count}
\end{table}


The estimated $\chi^2$ give by:
\[
 \chi^2 = \sum_{i=0}^{k-1} \frac{(O_i - E_i)^2}{E_i}=2.01544
\]
Then, I have $dof= 4-1 -1 =2$. So, 
\[
p-value = P[\chi^2 \geq 2.01544] = 0.3651
\]

This $p-value$ is very high, so it seems that the model has good fit with the observed data.

\section{Problem 7}
Let's assume a null hypothesis that the coin is fair, then $T \sim B(10,0.5)$, where $T$
is the number of heads. For this
case, the significance level is given by:
\[
\alpha = P[T = 0]+P[T=1] +P[T=9]+P[T=10] = 0.02148
\]

For the power of the test we have to compute the probability of $T=[0,1,9,10]$ under
the assumption that $T\sim B(n,0.6)$.
\[
power(0.6) = P[T = 0]+P[T=1] +P[T=9]+P[T=10] = 0.04804
\]


\end{document}
