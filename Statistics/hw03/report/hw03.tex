\documentclass[10pt]{article}
\usepackage{amsmath, amsthm, amsbsy, rotating,float}
\usepackage{graphicx,algorithm,algorithmic,subfig}
\usepackage{setspace,enumerate}
\doublespacing




\author{Esteban D\'{i}az}
\title{Homework 3}{}

\begin{document}
\maketitle

\section{Exercise 1}

Given the following joint distribution:
\begin{tabular}{c | c  c c }

 $P_{XY}$ & 0     & 1     &  2   \\ \hline
    0     & 0.15  & 0.05  & 0.01 \\ 
    1     & 0.10  & 0.08  & 0.01 \\ 
    2     & 0.10  & 0.14  & 0.02 \\ 
    3     & 0.10  & 0.08  & 0.03 \\ 
    4     & 0.05  & 0.05  & 0.03 \\ 
\end{tabular}

\begin{enumerate}[a.]
  \item Find the marginal probability functions for X and Y. 
        In this case we have to integrate $P_{XY}$ in the $X$ or $Y$
        direction:
        \[
        P(X=x) = \sum_{Y=y} P_{XY} = \begin{pmatrix} 0.21 & 0.19 & 0.26 & 0.21 & 0.13 \end{pmatrix} 
       \]
       and 
        \[
        P(Y=y) = \sum_{X=x} P_{XY} = \begin{pmatrix} 0.5 & 0.4 & 0.1\end{pmatrix} 
       \]

  \item Are $X$ and $Y$ independent? $X$ and $Y$ are not independent since:
      \[
        P(X,Y) \neq P(X) P (Y) 
      \]

  \item Find the mean and variance of $X$ and $Y$:
        \[  
        \mu_X = \sum_{x} P(X=x)x  = 1.86
        \] 
        \[
        var(X) = \sum_{x} P(X=x) (x-\mu_X)^2 = 1.7404
        \]
        \[
        var(Y) = \sum_{y} P(Y=y) (y-\mu_Y)^2 = 0.44 
        \]
        \[  
        \mu_Y = \sum_{y} P(Y=y)y  = 0.6
        \] 

    \item Compute the correlation coefficient:
        \[
        \rho_{XY}  = Corr(X,Y) = \frac{Cov(X,Y)}{\sqrt{var(X)var(Y)}} = \frac{\sum_x \sum_y (x-\mu_x)(y-\mu-y)P(x,y)}{\sqrt{var(X)var(Y)}} = 0.23312
        \]

    \item Find the conditional probability of $X$ given that the are no major circuit pack failures. What is the mean of this 
          conditional distribution?

          This amounts to find: 
         \[P(X | Y=0) = \frac{P(X,Y=0)}{P(Y=0)} = \begin{pmatrix} 0.3 & 0.2 & 0.2 & 0.2 & 0.1 \end{pmatrix}\] 
          The mean on the conditional is:
          \[\mu_c = \sum_{x} \frac{P(X,Y=0)}{P(Y=0)}x = 1.6 \]

    \item For the demerits assignment given by $D = 5Y+X$ find the mean value of D.
         \[
          \mu_D = 5\mu_Y +\mu_X = 4.86
          \]
    \item Find $P(D \leq 7)$. 
          \[
          P(D \leq 7) = \sum_{5Y + X \leq 7} P(x,y) = 0.77
          \]
    \item On the average, how many of these devices will have to be inspected in order to find 
          one that scores 7 or less demerits? 
    
          One would need to inspect $2 = \lceil1/P(D\leq7)\rceil+1 $ devices to find one that has a Demerit lower than 8.
         
\end{enumerate}


\section{Exercise 2}
  One can model the c.d.f in terms of the Binomial sample $X \sim B(n,p)$ with $p$ being the failure rate $p=0.01$, and $n$, the number
of realizations of $X$. In terms of the binomial solution:

\[
P_{B}(X > 5) = 1 - P_{B}(X \leq 5) = 1 - \sum_{k=0}^5 \begin{pmatrix} n \\ k \end{pmatrix} p^n (1-p)^{n-k} = 0.0330353 
\]

We can approximate $X$ to a Poisson distribution: $X \sim Pois(\lambda)$ with $\lambda=np = 6$. Therefore,

\[
P_{P}(X > 5) = 1- P_{P}(X \leq 5) = 1 -  \sum_{k=0}^5 \frac{ e^{-\lambda}\lambda^k}{k!} = 0.0362612 
\]

For the Gaussian distribution, one can do:
\[
 P (X > 5) = P\left( \frac{X -\mu}{\sigma} > \frac{5-\mu}{\sigma}     \right) 
\]

Given that this problem is explained by a binomial distribution, we know $\mu = np =6 $ and $\sigma = \sqrt{np(1-p)}=.244949$. Therefore,
once normalized the distribution it amounts to

\[
P (X >5) = 1 - \phi\left(\frac{5-\mu}{\sigma}\right) = 0.6592089
\]

One can see that the Poisson approximation is relatively good, whereas the Gaussian is not. One reason of the error in the Gaussian
could be the limited number of trials ($n=600$).

\section {Exercise 2.49} 

  Let $X_1$, $X_2$ be uncorrelated r.v.'s each with variance $\sigma^2$. Show that $Y_1 = X_1+X_2$ and $Y_2 =X_1 -X_2$ are 
  uncorrelated.

  $Y_1$ and $Y_2$ are uncorrelated if $Cov(Y_1,Y_2) =0$:

  \[
    Cov(Y_1,Y_2) = Cov(X_1 +X_2, X_1 -X_2) = 
  \]
  \[
Cov(X_1,X_1)+Cov(X_2,X_1) -Cov(X_1,X_2) -Cov(X_2,X_2) = \sigma^2 + 0 + 0 -\sigma^2 = 0 \qed
  \]


\section{Exercise 2.55}

Given the joint distribution: $ f(x,y) = 8xy $ for $0\leq x < y \leq 1$: 

\begin{enumerate}[a.]
    \item Find the marginal probabilities. Are $X$ and $Y$ independent?  

          Similarly as in Exercise 1.a:
          \[
          g(x) = \int_x^1 f(x,y) dy = 4x - 4x^2 
          \]
          and
          \[
          h(y) = \int_0^y f(x,y) dx = 4y^3
          \]

          $X$ and $Y$ are not independent since:
          \[
          f(x,y) \neq g(x)h(y)
          \]

    \item Find the conditional probability of Y given X:
        \[
          P (y|x) =\frac{f(x,y)}{g(x)} = \frac{8xy}{4x-4x^2} = \frac{2y}{1-x^2}
        \]
    
    \item Find Cov(X,Y) :

        \[
         Cov (X,Y) = E(XY) -\mu_x \mu_y 
        \]

        \[
        \mu_x = \int_0^1 x(4x-4x^2) dx = \frac{1}{3} 
        \]
        \[
        \mu_y = \int_0^1 y(4y^3) dy = \frac{4}{5} 
        \]


        \[
          E(XY) = \int_0^1 \int_0^y xy(f(x,y))dxdy = \int_0^1 \int_0^y 8x^2y^2 dxdy = \int_0^1 \frac{8}{3}y^5 dy = \frac{8}{18}
        \]

        \[
          Cov(X,Y) = \frac{8}{18} - \frac{4}{5}\frac{1}{3} =0.1777
        \]

\end{enumerate}



\section{Exercise 2.74}
  
  Let $X$ be the time to failure of a light bulb. Assume that $X$ is exponentially distributed:
    \begin{enumerate}[a.]
      \item If the mean time to failure is $10,000h$, which is the median time to failure? 
            For an exponential distribution the mean $E(x)=\frac{1}{\lambda}$, where $\lambda$ is the
            rate between two events. Therefore $\lambda=1/10000 h^-1$.
    
            \[
             F(x) = 1-e^{-\lambda x} = 0.5 
            \]

            Where $F(x)$ is the c.d.f. of the exponential distribution.
  
            This implies $ \tilde{x} = \frac{-\ln(0.5)}{\lambda}=6,931.48h$ 
    
      \item What is the probability that a bulb will last for at least 1,000h?:
  
            This is $P(x \leq 1000h)$. Thefore,
            \[
            P(x \leq 1000h) = F(1000h) = 1-e^{-\lambda x} = 1-e^{-0.1} = 0.0952
            \]
  
      \item What is the probability that the bulb will burn for 1000h extra hours
            given that it burned already for 1000h? 

            The probability is the same as before because of the memoryless property
            of the exponential distribution.

    \end{enumerate}

\section{Exercise 2.80}
    The weight in a can is normally distributed with $\mu=16.1oz$ and $\sigma=0.5oz$:

  \begin{enumerate}[a.]
    \item What is the probability that the can contains less than 16oz?
          This is:
          \[
          P(x < 16oz) = P(\frac{x-\mu}{\sigma} < \frac{16-\mu}{\sigma}=\alpha) = 1 - \phi(\alpha) = 0.4207
          \]
    \item Probability of the weight between 16 and 16.5oz? 
          \[
          P(16 < x <16.5) = \phi( \frac{16.5-\mu}{\sigma}) -\phi( \frac{16-\mu}{\sigma}) = 0.420604
          \]

    \item What is the 10th percentile of the distribution?

          Following table A.3 from the book, we find that for a normal distribution
          \[
            \phi(Z=-1.115 ) =0.10
          \]

            Then :
          \[
            Z = \frac{x-\mu}{\sigma} = -1.115 
          \]

          Then $x = {-1.115 \sigma +\mu} = 15.5425 $

  \end{enumerate}
  
  
\end{document}
