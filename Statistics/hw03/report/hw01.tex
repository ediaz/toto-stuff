\documentclass[10pt]{article}
\usepackage[utf8x]{inputenc}
\usepackage{amsmath, amsthm, amsbsy, rotating,float}
\usepackage{graphicx,algorithm,algorithmic,subfig}
\usepackage{setspace,enumerate}
\doublespacing




\author{Esteban D\'{i}az}
\title{Homework 1}{}

\begin{document}
\maketitle

\section{Problem 1}

Some bacteria testing data is given, we want to know the reliability 
of the test. The sample space $\mathbf{S}=\{A,B,T+,T-\}$ consist of four events:\\
A : event of water being free of bacteria. \\
B : event of water containing bacteria. \\
T+: event of test resulting positive. \\
T-: event of test resulting negative.

We know that the water is free of bacteria 99\% of the time. Therefore, $P(A)=0.99$. 
 We also know that the test is positive ($T+$) when the water is free of bacteria 5\% of the 
time, this implies $P(T+|A)=0.05$. Finally, the test is positive 99\% of the time when the water
is contaminated, $P(T+|B)=0.99$. 

In order to test the reliability of the method we would like to know $P(A|T+)$. This means, that
given that the test is positive, what are the chances that the water is free of bacteria, which
implies a false positive.

\[
P(A|T+) =\frac{P(T+|A)P(A)}{P(T+|A)P(A)+P(T+|B)P(B)} = 0.833
\]
with $P(B)=1-P(A)$.


\section{Problem 2.16}
Given that:

\begin{tabular}{ c c c }
  $P(A)=0.5$       & $P(B)=0.6$        & $P(C)=0.6$ \\
  $P(A\cap B)=0.3$ & $P(A\cap C)=0.2$  & $P(B\cap C)=0.3$ \\
  $P(A\cap B \cap C)=0.1$ &  &  \\
\end{tabular}

find the conditional probabilities $P(A|B)$, $P(B|C)$, $P(A|B^C)$ and \\ $P(B^C|A\cap C)$

\[
P(A|B)=\frac{P(A\cap B)}{P(A)} = 0.6 
\]

\[
P(B|C)=\frac{P(B\cap C)}{P(B)} = 0.18
\]

\[
P(A|B^C) = \frac{P(A\cap B^C)}{P(B^C)} = \frac{P(A)-P(A\cap B)}{1-P(B)} = 0.5
\]

\[
P(B^C|A\cap C) = \frac{P(A\cap C| B^C)P(B^C)}{P(A\cap C)}
\]
with 
\[
P(A\cap C|B^C)=\frac{P(A\cap C \cap B^C)}{P(B^C)}=\frac{P(A\cap C)-P(A\cap B\cap C)}{P(B^C)},
\] 
therefore,
\[
P(B^C|A\cap C) = 0.5
\]

\section{Problem 2.17}
Let $P(A)=0.4$, $P(B)=p$, and $P(A\cup B)=0.8$.


\begin{enumerate}[a)] % a), b), c), ...
\item  Find $p$ such that A and B are mutually exclusive.
\item  Find $p$ such that A and B are independent
\end{enumerate}

a) For A and B be mutually exclusive if $A\cup B=0$. Therefore, \\  $p=P(A\cap B)-P(A)=0.4$. 


b) Two variables are independent if $P(A\cap B)=P(A)P(B)$. Therefore,
 $p=P(B) = \frac{P(A\cup B)-P(A)}{P(A^C)}= 0.6667$ 


\section{Problem 2.22}
1000 adults from the city, the suburb and the country were asked if they agreed with a tax increase. 
The results are shown in the following table. Do the results indicate that the place of recidence 
and the opinion about tax increase are independant?


\begin{tabular}{c |c c| c }

          & Yes   & No    & Total \\ \hline
  City    & 100   & 300   & 400 \\ 
  Suburb  & 250   & 150   & 400 \\
  Country & 50    & 150   & 200 \\ \hline
          & 400   & 600   & 1000 \\
\end{tabular}






Now, I write the normalized version of the table:
\begin{tabular}{c |c c| c }

          & Yes    & No    & Total \\ \hline
  City    & 0.1    & 0.3   & 0.4 \\ 
  Suburb  & 0.25   & 0.15  & 0.4 \\
  Country & 0.05   & 0.15  & 0.2 \\ \hline
          & 0.4    & 0.6   & 1 \\
\end{tabular}


To demonstrate independance or not we need to show 
\[
P(City \cap Yes) = P(City) P(Yes)
\]
\[
P(Suburb \cap Yes) = P(Suburb) P(Yes)
\]
\[
P(Country \cap Yes) = P(Country) P(Yes)
\]

the table shows that this equality does not hold in any of the cases. Therefore, it indicates no independence.


\section{Problem 2.23}
Suppose that $n\geq2$ components are connected in parallel. The components operate independently of each other, 
 and the reliability of each one is $p$. Let $A_i$ denote the event that the $ith$ component functions. Show that
the system reliability is given by:
    \[
        P(A_1 \cup A_2 \cup \dots \cup A_n)=1-(1-p)^n
    \]

Demonstration:
\[
P(A_1 \cup A_2) = 1 - P(A_1^C \cap A_2^C)
\]
by adding events to the union:
\[
P(A_1 \cup A_2 \cup \dots \cup A_n) = 1 - P(A_1^C \cap A_2^C \cap \dots \cap A_n^C)
\]
Finally, provided that the events $A_i$ are independent:
\[
P(A_1 \cup A_2 \cup \dots \cup A_n) = 1 - (1-p)^n
\]



If $p=0.9$ and $n=2,3,4$ one can see that the reliability increases as we add components
to the system. Let $c$ be the number of components, then: $P(c=2) = 0.99$, $P(c=3)=0.999$ and
$P(c=4)=0.9999$. 
\end{document}
