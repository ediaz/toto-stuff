\documentclass[10pt]{article}
\usepackage{amsmath, amsthm, amsbsy, rotating,float}
\usepackage{graphicx,algorithm,algorithmic}
\usepackage{setspace,enumerate}
\usepackage{amsfonts}
\usepackage[margin=1.2in]{geometry}

\usepackage{graphicx,caption,subfig}

\floatstyle{ruled}
\newfloat{program}{thp}{lop}
\doublespacing

\def\dt{\dt}
\def\dxs{{\delta x}^2}
\def\heq{\frac{\partial u}{\partial t} -a\frac{\partial^2 u}{\partial x^2}}

\def\du#1#2{\frac{\partial^{#2} u}{\partial{#1}^{#2}}}

\def\dut#1{\frac{\partial u}{\partial {#1}}}

\def\exp#1{\mathbf{e}^{#1}}
\def\uk#1#2#3{{ U}^{#1}_{#2,#3}}
\def\ukk#1{{\bf U}^{#1}}

\newcommand{\twopartdef}[4]
{
 \left\{
   \begin{array}{ll}
     #1 & \mbox{if } #2 \\
     #3 & \mbox{if } #4
   \end{array}
 \right.
}
\def\dop#1#2{\fraction{\partial #1}{\partial #2}}

\author{Esteban D\'{i}az}
\title{2D implementation of the wave equation in the frequency domain}{}

\begin{document}
\def\pcscwp{
Center for Wave Phenomena \\ 
Colorado School of Mines \\ 
psava@mines.edu
}

\def\pcscover{
\author[]{Paul Sava}
\institute{\pcscwp}
\date{}
\logo{WSI}
\large
}

\def\WSI{\textbf{WSI}~}

% ------------------------------------------------------------
% colors
\definecolor{darkgreen}{rgb}{0,0.4,0}
\definecolor{LightGray}{rgb}{0.90,0.90,0.90}
\definecolor{DarkGray}{rgb}{0.85,0.85,0.85}

\definecolor{LightGreen} {rgb}{0.792,1.000,0.439}
\definecolor{LightYellow}{rgb}{1.000,0.925,0.545}
\definecolor{LightBlue}  {rgb}{0.690,0.886,1.000}
\definecolor{LightRed}   {rgb}{1.000,0.752,0.796}
\definecolor{Blue} {rgb}{0,0.08,0.45}



\def\red#1{\textcolor{red}{#1}}
\def\blue#1{\textcolor{blue}{#1}}
\def\green#1{\textcolor{green}{#1}}
\def\darkgreen#1{\textcolor{darkgreen}{#1}}
\def\black#1{\textcolor{black}{#1}}
\def\white#1{\textcolor{white}{#1}}
\def\yellow#1{\textcolor{yellow}{#1}}
\def\gray#1{\textcolor{gray}{#1}}
\def\magenta#1{\textcolor{magenta}{#1}}

% ------------------------------------------------------------
% madagascar
\def\mg{\darkgreen{\sc madagascar~}}
\def\mex#1{ \red{ #1 } }
\def\mvbt#1{\small{\blue{\begin{semiverbatim}#1\end{semiverbatim}}}}

% ------------------------------------------------------------
% equations
\def\bea{\begin{eqnarray}}
\def\eea{  \end{eqnarray}}

\def\beq{\begin{equation}}
\def\eeq{  \end{equation}}

% Norm symbol
\newcommand{\norm}[1]{\left|\left|#1\right|\right|}


%\def\req#1{(\ref{#1})}

\def\lp{\left (}
\def\rp{\right)}

\def\lb{\left [}
\def\rb{\right]}

\def\pbox#1{ \fbox {$ \displaystyle #1 $}}

\def\non{\nonumber \\ \nonumber}
\def\lnorm#1{\lVert#1\rVert}
% ------------------------------------------------------------
% REFERENCE (equations and figures)
\def\rEq#1{Equation~\ref{eqn:#1}}
\def\req#1{equation~\ref{eqn:#1}}
\def\rEqs#1{Equations~\ref{eqn:#1}}
\def\reqs#1{equations~\ref{eqn:#1}}
\def\ren#1{\ref{eqn:#1}}

\def\rFg#1{Figure~\ref{fig:#1}}
\def\rfg#1{Figure~\ref{fig:#1}}
\def\rFgs#1{Figures~\ref{fig:#1}}
\def\rfgs#1{Figures~\ref{fig:#1}}
\def\rfn#1{\ref{fig:#1}}

% ------------------------------------------------------------
% field operators

% trace
\def\tr{\texttt{tr}\;}

% divergence
\def\DIV#1{\nabla \cdot {#1}}

% curl
\def\CURL#1{\nabla \times {#1}}

% gradient
\def\GRAD#1{\nabla {#1}}

% Laplacian
\def\LAPL#1{\nabla^2 {#1}}

\def\dellin{
\lb
\begin{matrix}
\done{}{x} \; \done{}{y} \; \done{}{z}
\end{matrix}
\rb
}

\def\delcol{
\lb
\begin{matrix}
\done{}{x} \non
\done{}{y} \non
\done{}{z}
\end{matrix}
\rb
}

\def\aveclin{
\lb
\begin{matrix}
a_x \; a_y \; a_z
\end{matrix}
\rb
}


% ------------------------------------------------------------

% elastic tensor
\def\CC{{\bf C}}

% identity tensor
\def\I{\;{\bf I}}

% particle displacement vector
\def\uu{{\bf u}}

% particle velocity vector
\def\vv{{\bf v}}

% particle acceleration vector
\def\aa{{\bf a}}

% force vector
\def\ff{{\bf f}}

% wavenumber vector
\def\kk{{\bf k}}

% ray parameter vector
\def\pp{{\bf p}}

% distance vector
\def\hh{ {\boldsymbol{\lambda}} }
\def\xx{{\bf x}}
\def\kkx{{\kk_\xx}}
\def\ppx{{\pp_\xx}}

\def\yy{{\bf y}}

% normal vector
\def\nn{{\bf n}}
\def\ns{\nn_s}
\def\nr{\nn_r}

% source vector
\def\ss{{\bf s}}
\def\kks{{\kk_\ss}}
\def\pps{{\pp_\ss}}

% receiver vector
\def\rr{{\bf r}}
\def\kkr{{\kk_\rr}}
\def\ppr{{\pp_\rr}}

% midpoint vector
\def\mm{{\bf m}}
\def\kkm{{\kk_\mm}}
\def\ppm{{\pp_\mm}}

% offset vector
\def\ho{{\bf h}}
\def\kkh{{\kk_\ho}}
\def\pph{{\pp_\ho}}

% space-lag vector

\def\kkl{{\kk_\hh}}
\def\ppl{{\pp_\hh}}

% CIP vector
\def\cc{ {\bf c}}

% time-lag scalar
\def\tt{\tau}
\def\tts{\tt_s}
\def\ttr{\tt_r}

% frequency
\def\ww{\omega}

%
\def\dd{{\bf d}}

\def\bb{{\bf b}}
\def\qq{{\bf q}}

\def\ii{{\bf i}} % unit vector
\def\jj{{\bf j}} % unit vector

\def\lo{{\bf l}}

% ------------------------------------------------------------

\def\Fop#1{\mathcal{F}     \lb #1 \rb}
\def\Fin#1{\mathcal{F}^{-1}\lb #1 \rb}


% ------------------------------------------------------------
% wave equation operators:

\def\swe#1{ \frac{1}{v^2(\xx)}\dtwo{#1}{t}- \LAPL{#1}}
\def\awe#1{ \frac{1}{v_p^2(\xx) \rho(\xx)}\dtwo{#1}{t}- \DIV{\frac{1}{\rho(\xx)}\GRAD{#1}} }
\def\ewe#1{ \rho(\xx) \dtwo{#1}{t} -v_p(\xx)^2\LAPL{#1}+v_s(\xx)^2\CURL{\CURL{#1}}}

% ------------------------------------------------------------
% partial derivatives

\def\dtwo#1#2{\frac{\partial^2 #1}{\partial #2^2}}
\def\done#1#2{\frac{\partial   #1}{\partial #2  }}
\def\dthr#1#2{\frac{\partial^3 #1}{\partial #2^3}}
\def\mtwo#1#2#3{ \frac{\partial^2#1}{\partial #2 \partial#3} }

\def\larrow#1{\stackrel{#1}{\longleftarrow}}
\def\rarrow#1{\stackrel{#1}{\longrightarrow}}

% ------------------------------------------------------------
% elasticity 

\def\stress{\underline{\textbf{t}}}
\def\strain{\underline{\textbf{e}}}
\def\stiffness{\underline{\underline{\textbf{c}}}}
\def\compliance{\underline{\underline{\textbf{s}}}}

\def\GEOMlaw{
\strain = \frac{1}{2} 
\lb \GRAD{\uu} + \lp \GRAD{\uu} \rp^T \rb
}

\def\HOOKElaw{
\stress = \lambda \; tr \lp \strain \rp {\bf I} + 2 \mu \strain 
}

\def\CONSTITUTIVElaw{
\stress = \stiffness \;\strain 
}


\def\NEWTONlaw{
\rho \ddot{\uu} = \DIV{\stress}
}

\def\NAVIEReq{
\rho \ddot\uu =
\lp \lambda + 2\mu \rp \GRAD{\lp \DIV{\uu} \rp}
             - \mu     \CURL{   \CURL{\uu}}
}

% ------------------------------------------------------------

% potentials
\def\VP{\boldsymbol{\psi}}
\def\SP{\theta}

% stress tensor
\def\ssten{{\bf \sigma}}

\def\ssmat{
\lp \matrix {
 \sigma_{11} &  \sigma_{12}   &  \sigma_{13} \cr
 \sigma_{12} &  \sigma_{22}   &  \sigma_{23} \cr
 \sigma_{13} &  \sigma_{23}   &  \sigma_{33} \cr
} \rp
}

% strain tensor
\def\eeten{{\bf \epsilon}}

\def\eemat{
\lp \matrix {
 \epsilon_{11} &  \epsilon_{12}   &  \epsilon_{13} \cr
 \epsilon_{12} &  \epsilon_{22}   &  \epsilon_{23} \cr
 \epsilon_{13} &  \epsilon_{23}   &  \epsilon_{33} \cr
} \rp
}


% plane wave kernel
\def\pwker{A e^{i k \lp \nn \cdot \xx - v t \rp}}


% ------------------------------------------------------------
% details for expert audience (math, cartoons)
\def\expert{
\colorbox{red}{\textbf{\LARGE \white{!}}}
}

% ------------------------------------------------------------
% image, data, wavefields

\def\RR{R}

\def\R{R(\xx)}
\def\Re{R(\xx,\hh,\tau)}
\def\Rc#1{R^{#1}(\xx)}
\def\Rce#1{R^{#1}(\xx,\hh,\tau)}
\def\Rct#1{R^{#1}(z,\tau)}
\def\Rcl#1{R^{#1}(z,\lambda_x)}

\def\UU{u}
\def\US{{\UU_s}}
\def\USt{{\UU_s^f}}
\def\USr{{\UU_s^b}}
\def\UR{{\UU_r}}
\def\URt{{\UU_r^f}}
\def\URr{{\UU_r^b}}

\def\DD{D}
\def\DS{{\DD_s}}
\def\DR{{\DD_r}}

\def\UUw{\UU}
\def\USw{{\UU_s}}
\def\URw{{\UU_r}}

\def\DDw{\DD}
\def\DSw{{\DD_s}}
\def\DRw{{\DD_r}}

% perturbations

\def\ds{\Delta s}
\def\di{\Delta \RR}
\def\du{\Delta \UU}

\def\dRR{\Delta \RR}
\def\dUU{\Delta \UU}
\def\dUS{\Delta \US}
\def\dUR{\Delta \UR}

\def\dtt{\Delta \tt}
\def\dhh{\Delta \hh}

% ------------------------------------------------------------
% Green's functions

\def\GG{G}

\def\GS{{\GG_s}}
\def\GR{{\GG_r}}

% ------------------------------------------------------------
% elastic data, wavefields

\def\eRR{\textbf{\RR}}

\def\eDS{{\textbf{\DD}_s}}
\def\eDR{{\textbf{\DD}_r}}
\def\eDD{{\textbf{\DD}}}

\def\eUS{{\textbf{\UU}_s}}
\def\eUR{{\textbf{\UU}_r}}
\def\eUU{{\textbf{\UU}}}

% ------------------------------------------------------------
% sliding bar
\def\tline#1{
\put(95,-3){\small \blue{time}}
\put(-4,-1){\small \blue{0}}
\thicklines
\put( 0,0){\color{blue} \vector(1,0){100}}
\put(#1,0){\color{red}  \circle*{2}}
}

% ------------------------------------------------------------
% arrow on figure
\def\myarrow#1#2#3{
\thicklines
\put(#1,#2){\color{green} \vector(-1,-1){5}}
\put(#1,#2){\color{green} \textbf{#3}}
}

\def\bkarrow#1#2#3{
\thicklines
\put(#1,#2){\color{black} \vector(-1,-1){5}}
\put(#1,#2){\color{black} \textbf{#3}}
}


\def\anarrow#1#2#3#4{
\thicklines
\put(#1,#2){\color{#4} \vector(-1,-1){5}}
\put(#1,#2){\color{#4} \textbf{#3}}
}

% ------------------------------------------------------------
% circle on figure
\def\mycircle#1#2#3{
\thicklines
\put(#1,#2){\color{green} \circle{#3}}
}

% ------------------------------------------------------------
% note on figure
\def\mynote#1#2#3{
\put(#1,#2){\color{green} \textbf{#3}}
}

\def\biglabel#1#2#3{
\put(#1,#2){\Huge \textbf{#3}}
}

\def \largelabel#1#2#3{
\put(#1,#2){\Large \textbf{#3}}
}


\def \normallabel#1#2#3{
\put(#1,#2){\normalsize \textbf{#3}}
}




\def\waxes{\put(-0,0){\color{white}\vector(1,0){10}}
\color{white} \klabellarge{10.5}{-1}{\color{white} {x}}
\put(-0,0){\color{white} \vector(0,-1){10}}
\color{white} \klabellarge{-1.2}{-14.5}{\color{white} {z}}}


\def\axes{\put(-0,0){\vector(1,0){10}}
\klabellarge{10.5}{-1}{\bf{x}}
\put(-0,0){\vector(0,-1){10}}
\klabellarge{-1.2}{-14.5}{\bf{z}}}



\def\putaxes#1#2{\put(#1,#2){\axes}}
\def\putwaxes#1#2{\put(#1,#2){\waxes}}

% huge labels
\def\wlabel#1#2#3{ \white{ \biglabel{#1}{#2}{#3} }}
\def\klabel#1#2#3{ \black{ \biglabel{#1}{#2}{#3} }}
\def\rlabel#1#2#3{ \red{   \biglabel{#1}{#2}{#3} }}
\def\glabel#1#2#3{ \green{ \biglabel{#1}{#2}{#3} }}
\def\blabel#1#2#3{ \blue { \biglabel{#1}{#2}{#3} }}
\def\ylabel#1#2#3{ \yellow{\biglabel{#1}{#2}{#3} }}


% large labels
\def\klabellarge#1#2#3{ \black{ \largelabel{#1}{#2}{#3} }}
\def\wlabellarge#1#2#3{ \white{ \largelabel{#1}{#2}{#3} }}

% normal labels
\def\klabelnormal#1#2#3{ \black{ \normallabel{#1}{#2}{#3} }}
\def\wlabelnormal#1#2#3{ \white{ \normallabel{#1}{#2}{#3} }}

% ------------------------------------------------------------
% centering
\def\cen#1{ \begin{center} \textbf{#1} \end{center}}
\def\cit#1{ \begin{center} \textit{#1} \end{center}}

% emphasis (bold+alert)
\def\bld#1{ \textbf{\alert{#1}}}

% huge fonts
\def\big#1{\begin{center} {\LARGE \textbf{#1}} \end{center}}
\def\hug#1{\begin{center} {\Huge  \textbf{#1}} \end{center}}

% ------------------------------------------------------------
% separator
\def\sep{ \vfill \hrule \vfill}
\def\itab{ \hspace{0.5in}}
\def\nsp{\\ \vspace{0.1in}}

% ------------------------------------------------------------
% integrals

\def\tint#1{\!\!\!\int\!\! #1 dt}
\def\xint#1{\!\!\!\int\!\! #1 d\xx}
\def\wint#1{\!\!\!\int\!\! #1 d\ww}
\def\aint#1{\!\!\!\alert{\int}\!\! #1 d\alert{\xx}}

\def\esum#1{\sum\limits_{#1}}
\def\eint#1{\int\limits_{#1}}

% ------------------------------------------------------------
\def\CONJ#1{\overline{#1}}
\def\MOD#1{\left| {#1} \right|}

% ------------------------------------------------------------
% imaging components

\def\IC{\colorbox{yellow}{\textbf{I.C.}}\;}
\def\WR{\colorbox{yellow}{\textbf{W.R.}}\;}
\def\WE{\colorbox{yellow}{\textbf{W.E.}}\;}
\def\SO{\colorbox{yellow}{\textbf{SOURCE}}\;}
\def\WS{\colorbox{yellow}{\textbf{W.S.}}\;}

% ------------------------------------------------------------
% summary/take home message
\def\thm{take home message}

% ------------------------------------------------------------
\def\dx{\Delta x}
\def\dy{\Delta y}
\def\dz{\Delta z}
\def\dt{\Delta t}

\def\dhx{\Delta h_x}
\def\dhy{\Delta h_y}

\def\kz{{k_z}}
\def\kx{{k_x}}
\def\ky{{k_y}}

\def\kmx{k_{m_x}}
\def\kmy{k_{m_y}}
\def\khx{k_{h_x}}
\def\khy{k_{h_y}}

\def\why{ \alert{\widehat{{\khy}}}}
\def\whx{ \alert{\widehat{{\khx}}}}

\def\lx{{\lambda_x}}
\def\ly{{\lambda_y}}
\def\lz{{\lambda_z}}

\def\klx{k_{\lambda_x}}
\def\kly{k_{\lambda_y}}
\def\klz{k_{\lambda_z}}

\def\mx{{m_x}}
\def\my{{m_y}}
\def\mz{{m_z}}
\def\hx{{h_x}}
\def\hy{{h_y}}
\def\hz{{h_z}}

\def\sx{{s_x}}
\def\sy{{s_y}}
\def\rx{{r_x}}
\def\ry{{r_y}}

% ray parameter (absolute value)
\def\modp#1{\left| \pp_{#1} \right|}

% wavenumber
\def\modk#1{\left| \kk_{#1} \right|}

% ------------------------------------------------------------
\def\kzwk{ {\kz^{\kk}}}
\def\kzwx{ {\kz^{\xx}}}
 
\def\PSk#1{e^{\red{#1 i \kzwk \dz}}}
\def\PSx#1{e^{\red{#1 i \kzwx \dz}}}
\def\PS#1{ e^{\red{#1 i k_z   \dz}}}

\def\TT{t}
\def\TS{t_s}
\def\TR{t_r}

\def\oft{\lp t \rp}
\def\ofw{\lp \ww \rp}

\def\ofx{\lp \xx \rp}
\def\ofk{\lp \kk \rp}
\def\ofs{\lp \ss \rp}
\def\ofr{\lp \rr \rp}
\def\ofz{\lp   z \rp}

\def\ofxt{\lp \xx, t  \rp}
\def\ofst{\lp \ss, t  \rp}
\def\ofrt{\lp \rr, t  \rp}

\def\ofxw{\lp \xx, \ww  \rp}
\def\ofsw{\lp \ss, \ww  \rp}
\def\ofrw{\lp \rr, \ww  \rp}

\def\ofxm{\lp \xx,\hh \rp}

\def\ofxmp{\lp \xx+\hh \rp}
\def\ofxmm{\lp \xx-\hh \rp}

\def\ofmm{\lp \mm      \rp}
\def\ofmz{\lp \mm, z   \rp}
\def\ofmw{\lp \mm, \ww \rp}
\def\ofkm{\lp \kkm     \rp}

% ------------------------------------------------------------
% source/receiver data and wavefields

\def\dst{$\DS\ofst$}
\def\drt{$\DR\ofrt$}
\def\ust{$\US\ofxt$}
\def\urt{$\UR\ofxt$}

\def\dsw{$\DS\ofsw$}
\def\drw{$\DR\ofrw$}
\def\usw{$\US\ofxw$}
\def\urw{$\UR\ofxw$}

% ------------------------------------------------------------
\def\Nx{N_x}
\def\Ny{N_y}
\def\Nz{N_z}
\def\Nt{N_t}
\def\Nw{N_{\ww}}
\def\Nm{N_{\mm}}

\def\Nlx{N_{\lambda_x}}
\def\Nly{N_{\lambda_y}}
\def\Nlz{N_{\lambda_z}}
\def\Nlt{N_{\tau}}

\def\wmin{\ww_{min}}
\def\wmax{\ww_{max}}
\def\zmin{z_{min}}
\def\zmax{z_{max}}
\def\tmin{t_{min}}
\def\tmax{t_{max}}
\def\lmin{\hh_{min}}
\def\lmax{\hh_{max}}
\def\xmin{\xx_{min}}
\def\xmax{\xx_{max}}

% ------------------------------------------------------------
% course qualifiers

\def\fun{\hfill \alert{concepts}}
\def\pra{\hfill \alert{applications}}
\def\fro{\hfill \alert{frontiers}}


% ------------------------------------------------------------
% wavefield extrapolation
\def\ws{ {\ww s} }

\def\kows{\lp \frac{\kx}{\ws} \rp}

\def\kmws{\lp \frac{\modk{\mm}}{\ws} \rp}
\def\kzws{\lp \frac{\kz}       {\ws} \rp}

\def\S{\lb\frac{\modk{\mm}}{\ws  }\rb}
\def\C{\lb\frac{\modk{\mm}}{\ws_0}\rb}
\def\K{\lb\frac{\modk{\mm}}{\ww  }\rb}

\def\Cs{\lb\frac{\modk{\mm}^2}{\lp \ws_0 \rp^2}\rb}

\def\SSR#1{  \sqrt{ \lp \ww {#1} \rp^2 - \modk{\mm}^2} }

\def\SQRsum#1{\sum\limits_{n=1}^{\infty} \lp -1 \rp^n
		\displaystyle{\frac{1}{2} \choose n} #1}

\def\TSE#1#2#3#4{\sum\limits_{#4=#3}^{\infty} \lp -1 \rp^#4
		\displaystyle{#2 \choose #4} {#1}^#4}

\def\onefrac#1#2{\frac{#2^2}{a_#1+b_#1 #2^2}}
\def\SQRfrac#1{
	\sum\limits_{n=1}^{\infty}
	\onefrac{n}{#1} }

\def\dkzds { \left. \frac{d {\kz}}  {d s} \right|_{s_b} }
\def\SSX#1#2{\sqrt{ 1 - \lb \frac{\MOD{#2}}{#1} \rb^2} }
\def\SST#1#2{1 + \sum_{j=1}^N c_j \lb \frac{\MOD{#2}}{#1} \rb^{2j} }

% ------------------------------------------------------------
% acknowledgment
\def\ackcwp{\cen{the sponsors of the\\Center for Wave Phenomena\\at\\Colorado School of Mines}}

% ------------------------------------------------------------
% citation in slides
\def\talkcite#1{{\small \sc #1}}

% ------------------------------------------------------------
\def\ise{GPGN302: Introduction to Electromagnetic and Seismic Exploration}
\def\inv{GPGN409: Inversion}

% ------------------------------------------------------------
\def\model{m}
\def\data {d}

\def\Lop{ {\mathbf{L}}}
\def\Sop{ {\mathbf{S}}}
\def\Eop{ {\mathbf{E}}}
\def\Iop{ {\mathbf{I}}}
\def\Aop{ {\mathbf{A}}}
\def\Pop{ {\mathbf{P}}}
\def\Fop{ {\mathbf{F}}}


% ------------------------------------------------------------
\def\mybox#1{
  \begin{center}
    \fcolorbox{black}{yellow}
    {\begin{minipage}{0.8\columnwidth} {#1} \end{minipage}}
  \end{center}
}

\def\hibox#1{
  \begin{center}
    \fcolorbox{black}{LightGreen}
    {\begin{minipage}{0.8\columnwidth} {#1} \end{minipage}}
  \end{center}
}

% ------------------------------------------------------------
% Nota Bene
\def\nbnote#1{
  \vfill
  \begin{center}
    \colorbox{LightGray}
    {\begin{minipage}{\columnwidth} {\textbf{\black{\large N.B.}} #1} \end{minipage}}
  \end{center}
}

\def\highlight#1{
  \begin{center}
    \colorbox{cyan}
    {\begin{minipage}{\columnwidth} {#1} \end{minipage}}
  \end{center}
}

% ------------------------------------------------------------
\def\pcsshaded#1{
  \definecolor{shadecolor}{rgb}{0.8,0.8,0.8}
  \begin{shaded} {#1} \end{shaded}
  \definecolor{shadecolor}{rgb}{1.0,1.0,1.0}
}

% ------------------------------------------------------------
\def\postit#1{
  \begin{center}
    \colorbox{yellow}
    {\begin{minipage}{0.66\columnwidth} {#1} \end{minipage}} 
  \end{center}
}

% ------------------------------------------------------------
\def\graybox#1{
  \begin{center}
    \colorbox{LightGray}
    {\begin{minipage}{1.00\columnwidth} {#1} \end{minipage}}
  \end{center}
}

\def\whitebox#1{
  \begin{center}
    \colorbox{white}
    {\begin{minipage}{1.00\columnwidth} {#1} \end{minipage}}
  \end{center}
}

\def\yellowbox#1{
  \begin{center}
    \colorbox{LightYellow}
    {\begin{minipage}{1.00\columnwidth} {#1} \end{minipage}}
  \end{center}
}

\def\greenbox#1{
  \begin{center}
    \colorbox{LightGreen}
    {\begin{minipage}{1.00\columnwidth} {#1} \end{minipage}}
  \end{center}
}

\def\bluebox#1{
  \begin{center}
    \colorbox{LightBlue}
    {\begin{minipage}{1.00\columnwidth} {#1} \end{minipage}}
  \end{center}
}

\def\redbox#1{
  \begin{center}
    \colorbox{LightRed}
    {\begin{minipage}{1.00\columnwidth} {#1} \end{minipage}}
  \end{center}
}

\def\hyellow#1{ \colorbox{yellow} #1 }
\def\hgreen #1{ \colorbox{green}  #1 }

% ------------------------------------------------------------
% boxes for vectors and matrices

\def\pcsbox#1#2#3#4{
  % #1 = width
  % #2 = height
  % %3 = hmax
  \begin{picture}(#3,#1)
    \linethickness{0.5mm}
    % 
    \multiput(0,#1)(#3, 0){2}{\line(0,-1){#2}}
    \multiput(0,#1)(0,-#2){2}{\line(+1,0){#3}}
    % 
    \put(1,3){#4}
  \end{picture}
}

\def\pcssym#1#2{
  \begin{picture}(1,#1)
    \put(0,3){#2}
  \end{picture}
}

\def\sidebyside#1#2{
  \begin{center}
    \colorbox{LightBlue}{
      \begin{minipage}{1.0\columnwidth} {#1} \end{minipage}
    }
    \colorbox{LightYellow}{
      \begin{minipage}{1.0\columnwidth} {#2} \end{minipage}
    }
  \end{center}
}

\def\sidebysidebyside#1#2#3{
  \begin{center}
    \colorbox{LightBlue}{
      \begin{minipage}{1.0\columnwidth} {#1} \end{minipage}
    }
    \colorbox{LightYellow}{
      \begin{minipage}{1.0\columnwidth} {#2} \end{minipage}
    }
    \colorbox{LightRed}{
      \begin{minipage}{1.0\columnwidth} {#3} \end{minipage}
    }
  \end{center}
}

\maketitle

In this report, I solve a second order hyperbolic PDE that corresponds to the wave equation
\beq
  \swe{{\bf u}} = f_s(\xx,t),
  \label{eq:tawe}
\eeq 
with ${\bf u} = u(\xx,t)$, $\nabla^2 u= u_{xx} + u_{zz}$, and  $f_s(\xx,t) = s(t)\delta(\xx-\xx_s)$ being a point source, injected at a specific location $\xx_s$. Here,
 $\xx = (x,z)$ are the coordinates in our domain $\Omega \in \mathbb{R}^2$.

To formulate the acoustic wave equation in the frequency domain, one has to perform a Fourier transform over 
the time variable $t$
\beq
\left(k(\xx)^2 + \nabla^2 \right)  {\bf u}  = -s(\omega)\delta(\xx-\xx_s).
  \label{eq:wawe}
\eeq
Here $k(\xx) = \omega/v_p(\xx)$, and ${\bf u} = u(\xx,\omega)$. The nice characteristic of this approach is that the
second time derivative (now frequency scaling) is exact in the frequency domain, and makes the problem frequency independent.


This wave equation is a simplified version of the wave phenomena seen in the nature. In reality, this wave phenomena occurs
in a semi-infinite medium and is not acoustic and isotropic as equations 1-2. In the computer, however,
we are unable to simulate an unbounded medium. Therefore, we have to truncate our domain.

Figure~\ref{fig:domain} depicts a cartoon of the ``real'' earth, and the truncated versions in our computer. 
In the earth, the surface ($z=0$) is approximatively free of pressure $u(\xx,t) = 0, \forall z =0$. In the 
other boundaries of the domain, the pressure waves radiate outgoing from the source position $\xx_s$. Therefore,
in the computer to simulate an infinite $x$ and $+z$ axis we have make sure that the waves dies before reaching
to the truncation.

The free surface condition at $z=0$ introduces unwanted reflections from the upper boundary that even though are
realistic, most of the time are not considered in practice (and the data is processed accordantly).

Based on all these considerations, for this project I implement two problems:
\begin{eqnarray}
\nonumber \left(k(\xx)^2 + \nabla^2 \right)  {\bf u}  = -s(\omega)\delta(\xx-\xx_s),  \text{ in }\Omega = (x_0,x_1)\times(0,z), \\
{\bf u} = 0, \text{  in }\delta\Omega_1 = z =0, \\
\nonumber {\bf u} \rightarrow 0, \text{ as } \xx \rightarrow \delta \Omega,\text{  in } \delta \Omega -\delta \Omega_1,
\label{eq:profree}
\end{eqnarray}
which corresponds to the sketch in Figure~\ref{fig:domain}b, and,
\begin{eqnarray}
\nonumber \left(k(\xx)^2 + \nabla^2 \right)  {\bf u}  = -s(\omega)\delta(\xx-\xx_s),  \text{ in }\Omega = (x_0,x_1)\times(0,z), \\
{\bf u} \rightarrow 0, \text{ as } \xx \rightarrow \delta \Omega,\text{  in } \delta \Omega,
\label{eq:proabc}
\end{eqnarray}
which corresponds to Figure~\ref{fig:domain}c.

The boundary conditions are achieved trough the implementation of the Perfectly Matched Layer (PML) approach~\cite{BerengerPML}.


\begin{figure}
\centering
\subfloat[]{\includegraphics[width=0.3\textwidth]{Fig/semi-inf.pdf}}
\subfloat[]{\includegraphics[width=0.3\textwidth]{Fig/semi-infpml1.pdf}}
\subfloat[]{\includegraphics[width=0.3\textwidth]{Fig/semi-infpml2.pdf}}
\caption{Medium simulation (a) real earth: unbounded semi-infinite medium, (b) PML region (lines) avoiding reflections
        from truncated domain with free surface, and (c) simulation of an infinite medium which avoids reflections
        from the truncated domain. Here $z$ is depth.}
\label{fig:domain}
\end{figure}


\section{Implementation}

The domain $\Omega$ is discretized as:
\begin{eqnarray}
 \nonumber x_{i1} = x_0 +i_1 h_x, \text{ with } i_1 = 0,1,...,Nx\\
 z_{i2} = z_0 +i_2 h_z, \text{ with } i_2 = 0,1,...,Nz
\end{eqnarray}
where $N_x,N_z$ are the number of grid points in $x,z$ axes.

The frequency grid can be discretized as:
\beq
\omega_k = w_0 + k \delta\omega, \text{ with }  k = 0,1,...,N_\omega.
\eeq
The frequency discretization is given by the FFT of the source function $S(\omega) = FFT\left\{s(t)\right\}$.

The discrete version of equation~\ref{eq:wawe} can be written as: 

\[
\left(k(\xx)^2 + L_x^2 + L_z^2 \right)  {\bf u}  = -s(\omega_k)\delta(\xx-\xx_s).
\]
Which can be written compactly in matrix form as:
\beq
 {\bf A}(\omega) {\bf U}(\omega) = {\bf F(\omega)}
\eeq

Here, $L_x$ and $L_z$ are discrete second order difference operators that can be thought as a filter $f[k]$ whose
impulse response is similar to one of the continuous derivative operators. To obtain the digital differentiator
filters, I follow~\cite{Fornberg88} approach.

During class, we demonstrated that the second order differentiator for the second derivative 
reads $f = 1/h^2[1,-2,1]$ , then $u_{xx} \approx L_x^2 u(x,z,\omega) = f[x]*u(x,z,\omega) + \mathcal{O}(h^2)$ with $*$ being the convolution operator. 
With Fornberg's algorithm implementation, I can use differentiators  for an arbitrary order. However, 
the computational cost is directly related to the order (and hence filter length) of the differentiator. Figure~\ref{fig:d2} shows the impulse 
response for the second derivative operators for $\mathcal{O}(h^{2k}), k=1,...,6$.

\begin{figure}
\centering
\includegraphics[width=0.5\textwidth]{Fig/SecondDerivative.pdf}
\caption{Impulse response for Fornberg's filters, for orders $2k, k =1,2,3,4,5,6$. The black curve shows the true
        response of the second derivative operator.}
\label{fig:d2}
\end{figure}

The approximate Laplacian operator $\nabla^2 \approx L_x^2 + L_z^2$ can be constructed in matrix form as,
\beq
{\bf L} = {\bf Bh}_x \otimes {\bf I}_z +{\bf Bh}_z \otimes {\bf I}_x,
\label{eq:Lop}
\eeq
 where $\otimes$ is the tensor product between two matrices and 
\begin{equation}
\mathbf{Bh}_x=\frac{1}{h_x^2}
 \begin{pmatrix}
  f_0     & f_1     &  f_2    & \dots   & f_{M/2} & \dots   & 0      & 0       \\
  f_{-1}  & f_0     &  f_1    & f_2     & \ddots  & f_{M/2} & \ddots & 0       \\
  f_{-2}  & f_{-1}  & \ddots  & f_1     & \ddots  &  \ddots & \ddots & f_{M/2} \\
  \vdots  & f_{-2}  & \ddots  & \ddots  & \ddots  & \ddots  & \ddots & \vdots  \\   
  f_{-M/2}& \ddots  & \ddots  & \ddots  & \ddots  & \ddots  & f_1    & f_2    \\   
  0       & \ddots  & \ddots  & \ddots  & \ddots  & f_{-1}  & f_0    & f_1    \\   
  0       & \dots   & 0       & f_{-M/2}& \dots   & f_{-2}  & f_{-1} & f_0       
 \end{pmatrix}_{Nx,Nx},
\label{eq:B}
\end{equation}
where M is the length of the centered filter $f$, and $\bf I_x$ is the identity matrix with size $Nx$. $\bf Bh_z$, $\bf Iz$ are constructed
similarly.


If one implements only Dirichlet BC (undesired for wave phenomena), the impedance matrix is:

\beq
{\bf A} = {\bf K} + {\bf L},
\eeq
where $\bf K$ is a main diagonal matrix with entries $K_{ii} = \frac{\omega^2}{vp_i^2}$. Such an impedance matrix
will produce reflections from the domain boundaries.

\section{PML implementation}

The PML BC~\cite{BerengerPML} is introduced using the so-called stretched coordinates, near the
boundaries of the domain. The PML is frequency dependent, and requires one wavelength to effectively attenuate
the waves and avoid boundary reflections.~\cite{notesPML} explains the PML very simply, it amounts to introduce
the following change:
\beq
  \frac{\partial}{\partial x} \rightarrow \frac{1}{1+i\frac{\sigma(x)}{\omega}}\frac{\partial}{\partial x} = p(x) \frac{\partial}{\partial x}
\eeq 
with,
\beq
\sigma(x) = \twopartdef{1}{ x \not\in {PML}}{\sigma_0 (x-{\partial P})^k}{ x\in {PML}}
\eeq
where $\partial P$ is the boundary of the PML region, and $k$ is a power, generally $k=2$. So, outside the PML
region the differential operator does not change, and inside the PML is scaled by a complex number that attenuates
the incoming waves.

Introducing the PML changes equation~\ref{eq:wawe} to:
\beq
k(\xx)^2 u + p(x)\dot{p}(x)\du{x}{} + p(z)\dot{p}(z)\du{z}{} +p(x)^2\du{x}{2} +p(z)^2\du{z}{2}  = -s(\omega)\delta(\xx-\xx_s).
  \label{eq:wawe2}
\eeq

So, we now need the first derivative as well. Figure~\ref{fig:d1} shows the impulse response for the Fornberg's first
derivative differentiators for different orders.
The weight derivative can be evaluated analytically and has the form:
\beq
\dot{r}(x) = \frac{dr}{d\sigma}\frac{d\sigma}{dx} = \twopartdef{0}{ x \not\in {PML}}{\frac{-i}{(1+i\frac{\sigma(x)}{\omega})^2}\sigma_0 p(x-{\partial P})^{p-1}}{x\in {PML}}
\label{eq:waweP}
\eeq
 
\begin{figure}
\centering
\includegraphics[width=0.5\textwidth]{Fig/FirstDerivative.pdf}
\caption{Impulse response for Fornberg's filters, for orders $2k, k =1,2,3,4,5,6$. The black curve shows the true
        response of the first derivative operator.}
\label{fig:d1}
\end{figure}

Equation~\ref{eq:waweP} can be implemented in matrix form as:
\[
\bf A =\bf K + {\bf Sx} D_x \otimes \bf I_z + {\bf Sz} D_z \otimes \bf I_x +{\bf Rx} B_x \otimes \bf I_z +{\bf Rz} B_z \otimes \bf I_x
\]
with 
\begin{equation}
\mathbf{D_x}=\frac{1}{h}
 \begin{pmatrix}
  f_0     & f_1     &  f_2    & \dots   & f_{M/2} & \dots   & 0      & 0       \\
  f_{-1}  & f_0     &  f_1    & f_2     & \ddots  & f_{M/2} & \ddots & 0       \\
  f_{-2}  & f_{-1}  & \ddots  & f_1     & \ddots  &  \ddots & \ddots & f_{M/2} \\
  \vdots  & f_{-2}  & \ddots  & \ddots  & \ddots  & \ddots  & \ddots & \vdots  \\   
  f_{-M/2}& \ddots  & \ddots  & \ddots  & \ddots  & \ddots  & f_1    & f_2    \\   
  0       & \ddots  & \ddots  & \ddots  & \ddots  & f_{-1}  & f_0    & f_1    \\   
  0       & \dots   & 0       & f_{-M/2}& \dots   & f_{-2}  & f_{-1} & f_0        
 \end{pmatrix}_{Nx,Nx},
\label{eq:D}
\end{equation}
is the first derivative matrix using a Fornberg's filter $f$ of length $M$. Matrix $\bf S$ is a diagonal matrix with entries $S_{ii}= p_i\dot{p}_i$ and 
matrix $\bf R$ is diagonal with entries $ R_{ii} = p_i^2$.



\section{Convergence analysis}
My implementation allows the use of any order for the error in the discrete operators, however, long operators are too expensive to use here and might
need other implementation approach. I do the error analysis for second and fourth order convergence. Table~\ref{tab:coeff} shows 
the filters coefficients for second and fourth orders computed after~\cite{Fornberg88}.

\begin{table}
\centering
  \begin{tabular}{c  c  c c c c c}
Derivative & Accuracy           & $-2h$ & $-h$     & 0   & $h$   & $2h$     \\  \hline
first      & $\mathcal{O}(h^2)$ &       & -1/2     & 0   & 1/2   &          \\  
first      & $\mathcal{O}(h^4)$ &1/12   & -2/3     & 0   & 2/3   & -1/12    \\  
second     & $\mathcal{O}(h^2)$ &       &  1       & -2  & 1     &          \\  
second     & $\mathcal{O}(h^4)$ &-1/12  & 4/3      &-5/2 & 4/3   & -1/12      
  \end{tabular}
\caption{Fornberg's filters coefficients used in the convergence error analysis.}
\label{tab:coeff}
\end{table}

To test the convergence, I used a know function
\[
 u(x,z,\omega=0) = 4\sin(4\pi x)\sin(4\pi z) \mbox{ }\forall x,z\in \Omega =(0,2)\times(0,2)
\]
shown in Figure~\ref{fig:known}.

Tables~\ref{tab:rate2} and~\ref{tab:rate4} show the results for the convergence, both of them
approaches to the expected rates.

To compute the error, I used the $L_2$ norm:
\[
L_2 = ||U(x_n,z_m)-u(x_n,z_m)||_{L_2} = \sqrt{\frac{1}{NM}\sum_n\sum_m (U(x_n,z_m)-u(x_n,z_m))^2}
\]
where $U$ is the computed solution.

The rate is computed as $r = \log(e_h/e_{2h})/\log(2)$ and $r = \log(e_h/e_{4h})/\log(2)$, respectively.

\begin{table}
\centering
  \begin{tabular}{c c c c}
  N&     h     &     r     &     e        \\ \hline
 26& 0.08000000&  --       & 0.019021662  \\
 51& 0.04000000& 1.81205972& 0.0054170659 \\
101& 0.02000000& 1.91133210& 0.0014401105 \\ 
201& 0.01000000& 1.95731802& 0.00037083814
  \end{tabular}
\caption{Convergence analysis for second order filters. Note that the rate converges to 2. I use $N=Nx=Nz$.}
\label{tab:rate2}
\end{table}


\begin{table}
\centering
  \begin{tabular}{c c c c}
  N&     h     &     r     &     e        \\ \hline
 13& 0.16000000& --        & 0.046950859  \\ 
 51& 0.04000000& 3.23384926& 0.0049906592 \\
201& 0.01000000& 3.86656567& 0.00034214143\\
801& 0.00250000& 3.96873277& 2.1852345e-05
  \end{tabular}
\caption{Convergence analysis for fourth order filters. Note that the rate converges to 4. I use $N=Nx=Nz$.}
\label{tab:rate4}
\end{table}

\begin{figure}
\centering
\includegraphics[width=0.5\textwidth]{Fig/known.pdf}
\caption{known function used for error analysis}
\label{fig:known}
\end{figure}


\section{Realistic tests}
I tested the code using the Pluto model~\cite{Pluto}. The Pluto synthetic model (Figure~\ref{fig:Pluto}) was developed to
benchmark migration and multiple\footnote{multiples are strong reflections from the free surface ($z=0$).} 
attenuation algorithms. This model mimics the geology of a Gulf of Mexico play.  

I test the solver injecting the $10Hz$ response of the source function in Figure~\ref{fig:source} at $\xx_s = ( 16.,  0.05)Km$.
I do two tests: one with free surface (problem in equation~\ref{eq:profree}) and one with absorbing boundaries at $z=0$ (problem in 
equation~\ref{eq:profree}).

\begin{figure}
\includegraphics[width=1.0\textwidth]{Fig/Pluto.pdf}
\caption{The Pluto velocity model.}
\label{fig:Pluto}
\end{figure}


\begin{figure}
\centering
\subfloat[]{\includegraphics[width=0.5\textwidth]{Fig/twav.pdf}}
\subfloat[]{\includegraphics[width=0.5\textwidth]{Fig/wwav.pdf}}
\caption{Time (a) and frequency (b) domain representation of the source function $s(t)$.}
\label{fig:source}
\end{figure}

\begin{figure}
\includegraphics[width=1.\textwidth]{Fig/free10Hz.pdf}
\caption{Response for $w=2\pi f$, $f=10Hz$ for free surface (without PML at $z=0$), which implements the model in Figure~\ref{fig:domain}b. The top panel
        shows the real part, and the bottom shows the imaginary part.}
\label{fig:free}
\end{figure}

\begin{figure}
\includegraphics[width=1.\textwidth]{Fig/abc10Hz.pdf}
\caption{Response for $w=2\pi f$, $f=10Hz$ for absorbing boundary at $z=0$, which implements the model in Figure~\ref{fig:domain}c.The top panel
        shows the real part, and the bottom shows the imaginary part. Note how this wavefield is simpler than the one with free surface.}
\label{fig:free}
\end{figure}


\section{Conclusions}
I have developed a Helmholtz solver for inhomogeneous medium (spatially variable). I implemented
the solver such that it could work with an arbitrary accuracy for the spatial derivatives. The 
tests for free surface and absorbing BC at the surface produced coherent results.
The error analysis demonstrated the correct convergence for second and fourth order. Although
not shown here, the rate should convergence for high order filters.

\bibliographystyle{plain}
\bibliography{biblio} 
\end{document}
