\documentclass[10pt]{article}
\usepackage{amsmath, amsthm, amsbsy, rotating,float}
\usepackage{graphicx,algorithm,algorithmic}
\usepackage{setspace,enumerate}
\usepackage[margin=1.0in]{geometry}
\usepackage{graphicx,caption,subfig}

\floatstyle{ruled}
\newcommand{\norm}[1]{\left\lVert#1\right\rVert_p}
\newcommand{\normi}[1]{\left\lVert#1\right\rVert_\infty}
\newcommand{\normt}[1]{\left\lVert#1\right\rVert_2}

\doublespacing
\author{Esteban D\'{i}az}
\title{Homework 3}{}
\floatstyle{ruled}
\newfloat{program}{thp}{lop}
\floatname{program}{Program}
\newfloat{printout}{thp}{lop}
\floatname{printout}{Printout}
\DeclareMathOperator{\sinc}{sinc}

\begin{document}

\maketitle

\section{Problem 1}
\subsection{Part a}
  The matrix $A \in R^{m\times n}$ has rank $n$.
  \begin{proof}
    the matrix $A=[\vec{a}_1;\vec{a}_2;\dots;\vec{a}_n]$. Each 
  column element is of the form $a_{i,1} = 1$, $a_{i,j} = t_ia_{i,j-1}, j=2,\dots,n$. From 
  here we can see that the columns of A are linearly independent. 
  The only way that the columns of A are linearly dependent is 
  when the observations are taken at the same place $t_i=0\forall i=1,\dots,n$.  
  Hence, $Rank A = dim\{\vec{a}_1,\dots,\vec{a}_n\} = n$.
  \end{proof}
\subsection{Part b}
    For the first setup $(m,n)=(100,15)$ the matrix $A^TA$ is 
    not positive definite. Hence, the Cholesky factorization
    cannot be obtained. For $(m,n)=(10,5)$ both methods (i) and (ii)
    give similar results. For $(m,n)=(100,15)$ methods (i)-(iv)
    give accurate results upto 7 significant digits. 


\section{Problem 2}
  Since $A$ is non singular, then there exist $A^{-1}$; and the
  columns of $A$ form a basis such that:
      \[
      \vec{b} = \sum_{j=1}^m x_j \vec{a}_j = A\vec{x}, x_j\in R 
      \]
    So we can think of $\vec{b}$ to be a linear function of $\vec{x}$.
  Hence, a perturbation of $\vec{b}$ results in a perturbation of $\vec{x}$:

   \[
    \vec{\delta b} = A(\vec{x} +\vec{\delta x})-A\vec{x}
    \] 
   \[
    \vec{\delta b} = A(\vec{x} +\vec{\delta x})-\vec{b}
   \] 
    \[
    \vec{\delta b}+\vec{b}= A(\vec{x} +\vec{\delta x})
    \]
    Hence,
    \[
    \vec{\delta b} = A\vec{\delta x}
    \]
    And 
    \[
    \vec{\delta x} = A^{-1} \vec{\delta b}
    \]
    \begin{equation}
    \norm{\vec{\delta x}} =\norm{ A^{-1} \vec{\delta b}}
    \end{equation}

    Now, 
    \[
    \norm{A^{-1}} = \max_{\vec{\delta b}\neq \vec{0}}\frac{\norm{A^{-1}\vec{\delta b}}}{\norm{\vec{\delta b}}} \geq \frac{\norm{A^{-1}\vec{\delta b}}}{\norm{\vec{\delta b}}}
    \]
    Assuming that $\vec{\delta b} \neq \vec{0}$ satisfies the maximum, we have: 
    \[
    \norm{A^{-1}} =\frac{\norm{A^{-1}\vec{\delta b}}}{\norm{\vec{\delta b}}}
      \]
    Using (1), we obtain:
    \[
    \norm{A^{-1}}\norm{\vec{\delta b}} = \norm{\vec{\delta x}}
    \]

    We can divide both sides by $\norm{\vec {x}}$:
    \begin{equation}
    \norm{A^{-1}}\norm{\vec{\delta b}}\frac{1}{\norm{\vec{x}}} = \norm{\vec{\delta x}}\frac{1}{\norm{\vec{x}}}
    \end{equation}

    We now that $\norm{\vec{b}} =\norm{A\vec{x}}$. Following similar
    logic as for the previous steps we have:
    \[
    \norm{A} = \max_{\vec{ x}\neq \vec{0}}\frac{\norm{A\vec{ x}}}{\norm{\vec{x}}} \geq \frac{\norm{A\vec{ x}}}{\norm{\vec{x}}}
    \]
    If we assume that $\vec{x}$ satisfies the equality, then:
    \begin{equation}
      \norm{A}\norm{\vec{x}} = \norm{A\vec{x}} = \norm{\vec{b}}
    \end{equation}
    from which we obtain:
    \[
    \norm{\vec{x}} = \norm{\vec{b}}\frac{1}{\norm{A}}
    \] 
    which we can substitute in equation (2):
    \begin{equation}
    \norm{A^{-1}}\norm{\vec{\delta b}}\frac{\norm{A}}{\norm{\vec{b}}} = \norm{\vec{\delta x}}\frac{1}{\norm{\vec{x}}}
    \end{equation}

    \begin{equation}
 \frac{\norm{\vec{\delta x}}}{\norm{\vec{x}}}=
    \norm{A^{-1}}\norm{A}\frac{\norm{\vec{\delta b}}}{\norm{\vec{b}}} 
    \end{equation}
    \begin{equation}
 \frac{\norm{\vec{\delta x}}}{\norm{\vec{x}}}=
    \kappa_p(A)\frac{\norm{\vec{\delta b}}}{\norm{\vec{b}}} 
    \end{equation}
     
    
\section{problem 3}
 We want to know if $f(x) = \frac{1-\cos x}{x^2}$ is well-conditioned.
 To do that we want to see if the condition number is low at $x=0$. This
 expression gets inaccurate in a computer as $x\approx 0$. 

  \[
  \kappa(f) = \left|\frac{f'(x) x}{f(x)}\right|=
  \left( \frac{-2(1-\cos x)}{x^3} + \frac{\sin x}{x^2}\right)\frac{x^3}{1-\cos x} =
  -2 + \frac{\sin x}{1-\cos x} x 
  \]
  \[
  \lim_{x \to 0} \frac{x \sin x}{1-\cos x} = \lim_{x \to 0} \frac{\sin x + x\cos x}{\sin x} =
   \lim_{x \to 0} 1 +\frac{x\cos x}{\sin x} = 2
  \]
  Hence, $\kappa(f) = 0$. The problem is well-conditioned.
  In order to avoid subtraction of nearly equal numbers one can
  find an equivalent expression where the subtraction does not occur:
  \[
  f(x) = \frac{1-\cos x}{x^2} = \frac{1-\cos x }{x^2}\frac{1+\cos x}{1+\cos x} = \frac{1-\cos^2 x}{x^2(1+\cos x)} = \frac{\sin^2 x}{x^2(1+\cos x)}=\frac{\sinc^2 x}{1+\cos x}
  \]  

\section{problem 4}
  In order to avoid subtraction of nearly equal numbers we can 
  find an equivalent expression in a similar way as in previous problem:
  \[
  f_2(x) = \sqrt{x+1}-\sqrt{x} = \sqrt{x+1}-\sqrt{x}\frac{\sqrt{x+1}+\sqrt{x}}{\sqrt{x+1}+\sqrt{x}} = \frac{1}{\sqrt{x+1}+\sqrt{x}}
  \]
  The expression in this problem gets inaccurate as x gets large. 
  The equivalent expression avoids subtraction, and therefore can 
  be considered the machine representation of the function, so 
we can measure the error of method 1 with respect to method 2:
   \[
    err = \left |\frac{f_2(x)-f_1(x)}{f_2(x)}\right|
  \]
  The number of correct significant digits ranges
  from 13 to 6 for $i=1,\dots,5$, respectively.
  The most innacurate result was for $x=100^4$.

\section{problem 5}
Let $\widetilde{w}=(x\oplus y)\otimes z$ be the machine
representation of $w$. Each machine operation
introduces a round off error $\epsilon\leq u$.
\[
  \widetilde{w}=(x+y)(1+\epsilon_1)z(1+\epsilon_2)
\]
let $\delta=\epsilon_1+\epsilon_2+\epsilon_1\epsilon_2\approx \epsilon_1+\epsilon_2$.
\[
  \widetilde{w}=(x+y)z(1+\delta)=w(1+\delta)
\]
Now, $|\delta| \approx |\epsilon_1+\epsilon_2| \approx <|\epsilon_1|+|\epsilon_2| \approx < 2u$.

So, the relative error is 
\[
  \left|\frac{w-\widetilde{w}}{w}\right| = |\delta| <2u
\]
Hence, the problem is backward stable since a small perturbation on the
input results in a small perturbation on the output. We can think
of $\widetilde{w}$ to be $w(z+\delta)$.

\section{Appendix}
\begin{program}
\begin{verbatim}
function [A,b] = problem1(m,n); 

pts = linspace(0,1,m)';

A = zeros(m,n);
b = zeros(m,1);
A(:,1) = 1;
for j=2:n
  A(:,j) = pts.*A(:,j-1);
end

for i=1:m
  b(i) = exp(sin(4*pts(i)));
end  
\end{verbatim}
  \caption{Problem 1 setup}
\end{program}

\begin{program}
\begin{verbatim}
function [x] = backsub(U,d)
[n,m] = size(U);

x = zeros(n,1);
x(n)= d(n)/U(n,n);

for i=n-1:-1:1
  suma = 0;
  for j=i+1:n
    suma = suma+U(i,j)*x(j);
  end
  x(i) = (d(i)-suma)/U(i,i);
end
\end{verbatim}
  \caption{Problem 1: back substitution routine}
\end{program}

\begin{program}
\begin{verbatim}
function [x] = forwsub(L,d)
[n,m] = size(L);

x = zeros(n,1);
x(1) = d(1)/L(1,1);

for i=2:n
  suma=0;
  for j=1:i-1
    suma = suma +L(i,j)*x(j);
  end
  x(i) = (d(i)-suma)/L(i,i);
end
\end{verbatim}
  \caption{Problem 1: forward substitution routine}
\end{program}

\begin{program}
\begin{verbatim}
function [x_chol] = problem_1bi(A,b);
R = chol(A'*A);
c = A'*b;
y = forwsub(R',c);
x_chol = backsub(R,y);
\end{verbatim}
  \caption{Problem 1: LS solution via Cholesky factorization}
\end{program}

\begin{program}
\begin{verbatim}
function [x_qr] = prob_1bii(A,b)
[Q,R] = qr(A,0); 
c = Q'*b;
x_qr = backsub(R,c);
\end{verbatim}
  \caption{Problem 1: LS solution via QR factorization}
\end{program}

\begin{program}
\begin{verbatim}
function [x_svd] = prob_1biii(A,b)
[U,S,V] = svd(A,0);

[m,n] = size(A);
r = size(S,1);
y = zeros(n,1);
c = U'*b;
for i=1:r
  y(i) = c(i)/S(i,i);
end
x_svd = V*y;
\end{verbatim}
  \caption{Problem 1: LS solution via SVD factorization}
\end{program}

\begin{program}
\begin{verbatim}

format long e;
clear;
clc; 

% m,n=15,10
m=100;
n=15;
[A,b]=problem1(m,n);
%x_chol = prob_1bi(A,b)
fprintf('--------------------\n')
fprintf('    m=100,n=15      \n')
x_qr = prob_1bii(A,b);
x_svd = prob_1biii(A,b);
x_mlab = A\b;
fprintf('QR solution x(n)\n')
x_qr(n)
fprintf('SVD solution x(n)\n')
x_svd(n)
fprintf('Matlab solution x(n)\n')
x_mlab(n)

m=10;
n=5;
xn = 2.006787453080206e03;
[A,b]=problem1(m,n);
x_chol = prob_1bi(A,b);
x_qr = prob_1bii(A,b);
x_svd = prob_1biii(A,b);
x_mlab = A\b;

fprintf('--------------------\n')
fprintf('    m=10,n=5      \n')
fprintf('Cholesky solution x(n)\n')
x_chol(n)
fprintf('QR solution x(n)\n')
x_qr(n)\end{verbatim}
  \caption{Problem 1: driver}
\end{program}

\begin{printout}
\begin{verbatim}
--------------------
    m=100,n=15      
QR solution x(n)

ans =

     2.006786880606435e+03

SVD solution x(n)

ans =

     2.006786880599539e+03

Matlab solution x(n)

ans =

     2.006787731242561e+03

--------------------
    m=10,n=5      
Cholesky solution x(n)

ans =

     2.281522755027906e+01

QR solution x(n)

ans =

     2.281522755029681e+01
\end{verbatim}
  \caption{Problem 1: driver output}
\end{printout}



\begin{program}
\begin{verbatim}
clear;
clc;
format long e;
for i=1:5
  x = (100)^i
  fx1=sqrt(x+1)-sqrt(x)
  fx2=1/(sqrt(x+1)+sqrt(x))
  err = abs(fx2-fx1)/fx2*100
end
\end{verbatim}
  \caption{Problem 4: driver}
\end{program}



\begin{printout}
\begin{verbatim}
x =

   100


fx1 =

     4.987562112088995e-02


fx2 =

     4.987562112089027e-02


err =

     6.538826106910213e-13


x =

       10000


fx1 =

     4.999875006248544e-03


fx2 =

     4.999875006249609e-03


err =

     2.130293683178898e-11


x =

     1000000


fx1 =

     4.999998750463419e-04


fx2 =

     4.999998750000625e-04


err =

     9.255879628564401e-09
\end{verbatim}
  \caption{Problem 4: printout}
\end{printout}

\begin{printout}
\begin{verbatim}

x =

   100000000


fx1 =

     5.000000055588316e-05


fx2 =

     4.999999987500000e-05


err =

     1.361766334613046e-06


x =

     1.000000000000000e+10


fx1 =

     4.999994416721165e-06


fx2 =

     4.999999999875000e-06


err =

     1.116630767065646e-04
\end{verbatim}
  \caption{Problem 4: printout (continued)}
\end{printout}



\end{document}
