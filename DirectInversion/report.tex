\documentclass[10pt]{article}
\usepackage{amsmath, amsthm, amsbsy, rotating,float}
\usepackage{graphicx,algorithm,algorithmic}
\usepackage{amssymb}
\usepackage{setspace,enumerate}

\usepackage{graphicx,caption,subfig}

\floatstyle{ruled}
\newcommand{\norm}[1]{\left\lVert#1\right\rVert_p}
\newcommand{\normi}[1]{\left\lVert#1\right\rVert_\infty}
\newcommand{\normt}[1]{\left\lVert#1\right\rVert_2}

\doublespacing

\author{Esteban D\'{i}az}
\title{Notes on direct inversion}{}

\begin{document}

\maketitle


\section{Constant density acoustics}

Here we consider the following wave equation:

\begin{equation}
  \frac{\omega^2 }{v^2(x)} u(x,\omega) +\nabla^2 u(x,\omega) = f(x,\omega).
  \label{eq:cden}
\end{equation}
 Now, let's imagine we want to solve for the model $m = 1/v^2$, given the wavefield $u$ produced
by the source $f$. The solution for this problem takes the following form:

\begin{equation}
   m = \mathbb{R}\left\{ \frac {{f(x,\omega)} - \nabla^2{ u(x,\omega) }}{\omega^2 u(x,\omega)} \right\}
\end{equation}

In theory, we can obtain the model $m$ for one frequency since we have only one unknown free variable.
In order to get redundancy, one can make use of several frequencies (if we consider the model
to be frequency independent). Then we can do

\begin{equation}
   m =  \mathbb{R}\left\{\sum_\omega \frac {{f(x,\omega)} - \nabla^2{ u(x,\omega) }}{\omega^2 u(x,\omega)} \right\}.
\end{equation}
In general, the area of influece of the source function $f(x,\omega)$ is very small (i.e. is injected in a small area). 
 This fact means that we don't need to know about the source function away from its area of influence. Furthermore,
 for most of the cases the model  should be well known at the source position (water velocity for marine acquisition for instance). 

\subsection{least squares inversion}

I think, perhaps, the right thing to do is to find the least squares 
model estimate. We have great redundancy with the number of frequencies
and surface source locations. That means that we need to solve an 
\emph{over-determined}  problem.

Let $A1(x_s,\omega)$ be the forward
 modeling matrix for one source location and angular frequency $\omega$:
\[
  A1 = diag\{\omega^2 u(x_s,x\omega)\} \in R^{N,N} 
\]
where $N$ is the number of model parameters.

This matrix is sparse (is only non-zero along the main diagonal).
Then, the left hand side of the system is composed by many of this 
matrices concatenated along the rows directions:

\[
 A = 
 \begin{pmatrix}
  A1     \\
  \vdots  \\
  A_N    
 \end{pmatrix};
\in R^{N\times ns \times n\omega,N},
\]
where $ns$ is the number of source locations and $n\omega$ is the number
of frequencies used in the inversion.

On the right hand side of the system we have 
\[
 r_i = f(x_s,\omega) -\nabla^2 u(x_s,\omega).
\]

The right hand side vector is composed by the combination of the individual
right hand side vectors:
\[
 r = 
 \begin{pmatrix}
  f_1     \\
  \vdots  \\
  f_N    
 \end{pmatrix};
\in R^{N\times ns \times n\omega}
\]

Once we have the left hand side matrix $A$ and the right hand side vector $r$, 
we can proceed to solve for the model:
\[
 {\bf A }{\bf m} = \bf r.
\]
The matrix $\bf A$ is very sparse (composed by a main diagonal  for each combination
of angular frequency $\omega$ and source location $x_s$).


\section{Variable density acoustics}

We can further increase the complexity of the wave propagator by introducing variable density: 

\begin{equation}
  \frac{\omega^2 }{v^2(x) \rho(x)} u(x,\omega) +\nabla \cdot(\frac{1}{\rho} \nabla u(x,\omega)) = f(x,\omega).
  \label{eq:vden}
\end{equation}

How could we decouple the effect of two model parameters $m_1 = \frac{1}{v^2\rho}$ and $m_2 = \frac{1}{\rho}$?


in matrix form the equation can be written as:
\begin{equation}
  {\bf A} {\bf m_1} + {\bf B} {\bf m_2} = {\bf f}
\end{equation}

where ${\bf A} = diag\{\omega^2 u(x,\omega)\}$, and 

\begin{equation}
 {\bf B} = -{\bf D_x^T}{ diag\{\bf D_x u\}} - {\bf D_z^T}{ diag\{\bf D_z u\}} 
\end{equation}

If we consider again, the model to be frequency independent; then, we can take advantage of the 
redundancy in the frequency domain to solve for $m_1$ and $m_2$ by using 2 frequencies:
\begin{align}
  {\bf A}_1 {\bf m_1} + {\bf B}_1 {\bf m_2} = {\bf f}_1 \\
  {\bf A}_2 {\bf m_1} + {\bf B}_2 {\bf m_2} = {\bf f}_2
\end{align}

For which we can obtain:
\begin{align}
  \left( {\bf B_2}-{\bf A_2}{\bf A_1^{-1}} \right) {\bf m}_2 = {\bf f_2} -{\bf A_2}{\bf A_1^{-1} f_1} \\
 {\bf A}_1 {\bf m}_1 = ({\bf f_1}-{\bf B_1 m_2}) 
\end{align}






\end{document}
