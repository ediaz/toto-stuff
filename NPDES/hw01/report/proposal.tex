\documentclass[10pt]{article}
../../hw03/report/pcsmacros.tex

\usepackage{amsmath, amsthm, amsbsy, rotating,float}
\usepackage{graphicx,algorithm,algorithmic}
\usepackage{setspace,enumerate}

\usepackage{graphicx,caption,subfig}

\floatstyle{ruled}
\newfloat{program}{thp}{lop}
\doublespacing

\author{Esteban D\'{i}az}
\title{Pseudo-spectral solution for the acoustic wave equation}{}

\begin{document}
\maketitle

In Geophysics,
 we usually implement the acoustic wave equation
\begin{equation}
 \swe{u} = f_s(\xx,t) 
\label{eq:swe}
\end{equation}
in the time domain. Alternatively, we can do a mixed domain $\omega-\xx$ implementation. This should yield into a very accurate approximation
for the time derivative operator (simply becomes $-\omega^2$)
\beq
(\frac{1}{v_p(\xx)^2}\omega^2 +\LAPL{} )u(\xx,\omega) = -f_s(\xx,\omega).
\eeq

This results in a complex sparse linear system of equations:
\beq
 \bf{A} (\omega) {\bf U}^{\omega} = -\bf{F}(\xx,\omega)
\eeq

Note that the previous equation is very easy to parallelize since 
each frequency can be solved independently. Equation~\ref{eq:swe}
cannot be parallelized over time. The parallelization of the time
implementation is done on the spatial derivatives $\LAPL{}$.


I would also like to investigate accurate ways to 
solve $\LAPL{u}$ and therefore
reduce grid dispersion problems.~\cite{chu} describes an interesting
9-point operator to solve the spatial derivatives.\cite{917976} describes
arbitrarily long finite impulse response derivative operators. It 
could be interesting to compare with the simple Taylor
expansion approach.

By implementing the wave equation in the frequency domain, one can reduce
the necessary time steps to accurate solve a wavefield. In the frequency
domain, one just have to follow the Sampling theorem criteria  to reconstruct accurately a signal in
time. In the time domain, the sampling is given by CFL condition, and therefore many time-steps are necessary to ensure stability. 



\bibliographystyle{seg}
\bibliography{BIB}
\end{document}
