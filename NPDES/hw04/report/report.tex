\documentclass[10pt]{article}
\usepackage{amsmath, amsthm, amsbsy, rotating,float}
\usepackage{graphicx,algorithm,algorithmic}
\usepackage{setspace,enumerate}

\usepackage{graphicx,caption,subfig}

\floatstyle{ruled}
\newfloat{program}{thp}{lop}
\doublespacing


\def\dt{\dt}
\def\dxs{{\delta x}^2}
\def\heq{\frac{\partial u}{\partial t} -a\frac{\partial^2 u}{\partial x^2}}

\def\du#1#2{\frac{\partial^{#2} u}{\partial{#1}^{#2}}}

\def\dut#1{\frac{\partial u}{\partial {#1}}}

\def\exp#1{\mathbf{e}^{#1}}
\def\uk#1#2#3{{ U}^{#1}_{#2,#3}}
\def\ukk#1{{\bf U}^{#1}}

\author{Esteban D\'{i}az}
\title{2D implementation of the diffusion and wave equations}{}

\begin{document}
\def\pcscwp{
Center for Wave Phenomena \\ 
Colorado School of Mines \\ 
psava@mines.edu
}

\def\pcscover{
\author[]{Paul Sava}
\institute{\pcscwp}
\date{}
\logo{WSI}
\large
}

\def\WSI{\textbf{WSI}~}

% ------------------------------------------------------------
% colors
\definecolor{darkgreen}{rgb}{0,0.4,0}
\definecolor{LightGray}{rgb}{0.90,0.90,0.90}
\definecolor{DarkGray}{rgb}{0.85,0.85,0.85}

\definecolor{LightGreen} {rgb}{0.792,1.000,0.439}
\definecolor{LightYellow}{rgb}{1.000,0.925,0.545}
\definecolor{LightBlue}  {rgb}{0.690,0.886,1.000}
\definecolor{LightRed}   {rgb}{1.000,0.752,0.796}
\definecolor{Blue} {rgb}{0,0.08,0.45}



\def\red#1{\textcolor{red}{#1}}
\def\blue#1{\textcolor{blue}{#1}}
\def\green#1{\textcolor{green}{#1}}
\def\darkgreen#1{\textcolor{darkgreen}{#1}}
\def\black#1{\textcolor{black}{#1}}
\def\white#1{\textcolor{white}{#1}}
\def\yellow#1{\textcolor{yellow}{#1}}
\def\gray#1{\textcolor{gray}{#1}}
\def\magenta#1{\textcolor{magenta}{#1}}

% ------------------------------------------------------------
% madagascar
\def\mg{\darkgreen{\sc madagascar~}}
\def\mex#1{ \red{ #1 } }
\def\mvbt#1{\small{\blue{\begin{semiverbatim}#1\end{semiverbatim}}}}

% ------------------------------------------------------------
% equations
\def\bea{\begin{eqnarray}}
\def\eea{  \end{eqnarray}}

\def\beq{\begin{equation}}
\def\eeq{  \end{equation}}

% Norm symbol
\newcommand{\norm}[1]{\left|\left|#1\right|\right|}


%\def\req#1{(\ref{#1})}

\def\lp{\left (}
\def\rp{\right)}

\def\lb{\left [}
\def\rb{\right]}

\def\pbox#1{ \fbox {$ \displaystyle #1 $}}

\def\non{\nonumber \\ \nonumber}
\def\lnorm#1{\lVert#1\rVert}
% ------------------------------------------------------------
% REFERENCE (equations and figures)
\def\rEq#1{Equation~\ref{eqn:#1}}
\def\req#1{equation~\ref{eqn:#1}}
\def\rEqs#1{Equations~\ref{eqn:#1}}
\def\reqs#1{equations~\ref{eqn:#1}}
\def\ren#1{\ref{eqn:#1}}

\def\rFg#1{Figure~\ref{fig:#1}}
\def\rfg#1{Figure~\ref{fig:#1}}
\def\rFgs#1{Figures~\ref{fig:#1}}
\def\rfgs#1{Figures~\ref{fig:#1}}
\def\rfn#1{\ref{fig:#1}}

% ------------------------------------------------------------
% field operators

% trace
\def\tr{\texttt{tr}\;}

% divergence
\def\DIV#1{\nabla \cdot {#1}}

% curl
\def\CURL#1{\nabla \times {#1}}

% gradient
\def\GRAD#1{\nabla {#1}}

% Laplacian
\def\LAPL#1{\nabla^2 {#1}}

\def\dellin{
\lb
\begin{matrix}
\done{}{x} \; \done{}{y} \; \done{}{z}
\end{matrix}
\rb
}

\def\delcol{
\lb
\begin{matrix}
\done{}{x} \non
\done{}{y} \non
\done{}{z}
\end{matrix}
\rb
}

\def\aveclin{
\lb
\begin{matrix}
a_x \; a_y \; a_z
\end{matrix}
\rb
}


% ------------------------------------------------------------

% elastic tensor
\def\CC{{\bf C}}

% identity tensor
\def\I{\;{\bf I}}

% particle displacement vector
\def\uu{{\bf u}}

% particle velocity vector
\def\vv{{\bf v}}

% particle acceleration vector
\def\aa{{\bf a}}

% force vector
\def\ff{{\bf f}}

% wavenumber vector
\def\kk{{\bf k}}

% ray parameter vector
\def\pp{{\bf p}}

% distance vector
\def\hh{ {\boldsymbol{\lambda}} }
\def\xx{{\bf x}}
\def\kkx{{\kk_\xx}}
\def\ppx{{\pp_\xx}}

\def\yy{{\bf y}}

% normal vector
\def\nn{{\bf n}}
\def\ns{\nn_s}
\def\nr{\nn_r}

% source vector
\def\ss{{\bf s}}
\def\kks{{\kk_\ss}}
\def\pps{{\pp_\ss}}

% receiver vector
\def\rr{{\bf r}}
\def\kkr{{\kk_\rr}}
\def\ppr{{\pp_\rr}}

% midpoint vector
\def\mm{{\bf m}}
\def\kkm{{\kk_\mm}}
\def\ppm{{\pp_\mm}}

% offset vector
\def\ho{{\bf h}}
\def\kkh{{\kk_\ho}}
\def\pph{{\pp_\ho}}

% space-lag vector

\def\kkl{{\kk_\hh}}
\def\ppl{{\pp_\hh}}

% CIP vector
\def\cc{ {\bf c}}

% time-lag scalar
\def\tt{\tau}
\def\tts{\tt_s}
\def\ttr{\tt_r}

% frequency
\def\ww{\omega}

%
\def\dd{{\bf d}}

\def\bb{{\bf b}}
\def\qq{{\bf q}}

\def\ii{{\bf i}} % unit vector
\def\jj{{\bf j}} % unit vector

\def\lo{{\bf l}}

% ------------------------------------------------------------

\def\Fop#1{\mathcal{F}     \lb #1 \rb}
\def\Fin#1{\mathcal{F}^{-1}\lb #1 \rb}


% ------------------------------------------------------------
% wave equation operators:

\def\swe#1{ \frac{1}{v^2(\xx)}\dtwo{#1}{t}- \LAPL{#1}}
\def\awe#1{ \frac{1}{v_p^2(\xx) \rho(\xx)}\dtwo{#1}{t}- \DIV{\frac{1}{\rho(\xx)}\GRAD{#1}} }
\def\ewe#1{ \rho(\xx) \dtwo{#1}{t} -v_p(\xx)^2\LAPL{#1}+v_s(\xx)^2\CURL{\CURL{#1}}}

% ------------------------------------------------------------
% partial derivatives

\def\dtwo#1#2{\frac{\partial^2 #1}{\partial #2^2}}
\def\done#1#2{\frac{\partial   #1}{\partial #2  }}
\def\dthr#1#2{\frac{\partial^3 #1}{\partial #2^3}}
\def\mtwo#1#2#3{ \frac{\partial^2#1}{\partial #2 \partial#3} }

\def\larrow#1{\stackrel{#1}{\longleftarrow}}
\def\rarrow#1{\stackrel{#1}{\longrightarrow}}

% ------------------------------------------------------------
% elasticity 

\def\stress{\underline{\textbf{t}}}
\def\strain{\underline{\textbf{e}}}
\def\stiffness{\underline{\underline{\textbf{c}}}}
\def\compliance{\underline{\underline{\textbf{s}}}}

\def\GEOMlaw{
\strain = \frac{1}{2} 
\lb \GRAD{\uu} + \lp \GRAD{\uu} \rp^T \rb
}

\def\HOOKElaw{
\stress = \lambda \; tr \lp \strain \rp {\bf I} + 2 \mu \strain 
}

\def\CONSTITUTIVElaw{
\stress = \stiffness \;\strain 
}


\def\NEWTONlaw{
\rho \ddot{\uu} = \DIV{\stress}
}

\def\NAVIEReq{
\rho \ddot\uu =
\lp \lambda + 2\mu \rp \GRAD{\lp \DIV{\uu} \rp}
             - \mu     \CURL{   \CURL{\uu}}
}

% ------------------------------------------------------------

% potentials
\def\VP{\boldsymbol{\psi}}
\def\SP{\theta}

% stress tensor
\def\ssten{{\bf \sigma}}

\def\ssmat{
\lp \matrix {
 \sigma_{11} &  \sigma_{12}   &  \sigma_{13} \cr
 \sigma_{12} &  \sigma_{22}   &  \sigma_{23} \cr
 \sigma_{13} &  \sigma_{23}   &  \sigma_{33} \cr
} \rp
}

% strain tensor
\def\eeten{{\bf \epsilon}}

\def\eemat{
\lp \matrix {
 \epsilon_{11} &  \epsilon_{12}   &  \epsilon_{13} \cr
 \epsilon_{12} &  \epsilon_{22}   &  \epsilon_{23} \cr
 \epsilon_{13} &  \epsilon_{23}   &  \epsilon_{33} \cr
} \rp
}


% plane wave kernel
\def\pwker{A e^{i k \lp \nn \cdot \xx - v t \rp}}


% ------------------------------------------------------------
% details for expert audience (math, cartoons)
\def\expert{
\colorbox{red}{\textbf{\LARGE \white{!}}}
}

% ------------------------------------------------------------
% image, data, wavefields

\def\RR{R}

\def\R{R(\xx)}
\def\Re{R(\xx,\hh,\tau)}
\def\Rc#1{R^{#1}(\xx)}
\def\Rce#1{R^{#1}(\xx,\hh,\tau)}
\def\Rct#1{R^{#1}(z,\tau)}
\def\Rcl#1{R^{#1}(z,\lambda_x)}

\def\UU{u}
\def\US{{\UU_s}}
\def\USt{{\UU_s^f}}
\def\USr{{\UU_s^b}}
\def\UR{{\UU_r}}
\def\URt{{\UU_r^f}}
\def\URr{{\UU_r^b}}

\def\DD{D}
\def\DS{{\DD_s}}
\def\DR{{\DD_r}}

\def\UUw{\UU}
\def\USw{{\UU_s}}
\def\URw{{\UU_r}}

\def\DDw{\DD}
\def\DSw{{\DD_s}}
\def\DRw{{\DD_r}}

% perturbations

\def\ds{\Delta s}
\def\di{\Delta \RR}
\def\du{\Delta \UU}

\def\dRR{\Delta \RR}
\def\dUU{\Delta \UU}
\def\dUS{\Delta \US}
\def\dUR{\Delta \UR}

\def\dtt{\Delta \tt}
\def\dhh{\Delta \hh}

% ------------------------------------------------------------
% Green's functions

\def\GG{G}

\def\GS{{\GG_s}}
\def\GR{{\GG_r}}

% ------------------------------------------------------------
% elastic data, wavefields

\def\eRR{\textbf{\RR}}

\def\eDS{{\textbf{\DD}_s}}
\def\eDR{{\textbf{\DD}_r}}
\def\eDD{{\textbf{\DD}}}

\def\eUS{{\textbf{\UU}_s}}
\def\eUR{{\textbf{\UU}_r}}
\def\eUU{{\textbf{\UU}}}

% ------------------------------------------------------------
% sliding bar
\def\tline#1{
\put(95,-3){\small \blue{time}}
\put(-4,-1){\small \blue{0}}
\thicklines
\put( 0,0){\color{blue} \vector(1,0){100}}
\put(#1,0){\color{red}  \circle*{2}}
}

% ------------------------------------------------------------
% arrow on figure
\def\myarrow#1#2#3{
\thicklines
\put(#1,#2){\color{green} \vector(-1,-1){5}}
\put(#1,#2){\color{green} \textbf{#3}}
}

\def\bkarrow#1#2#3{
\thicklines
\put(#1,#2){\color{black} \vector(-1,-1){5}}
\put(#1,#2){\color{black} \textbf{#3}}
}


\def\anarrow#1#2#3#4{
\thicklines
\put(#1,#2){\color{#4} \vector(-1,-1){5}}
\put(#1,#2){\color{#4} \textbf{#3}}
}

% ------------------------------------------------------------
% circle on figure
\def\mycircle#1#2#3{
\thicklines
\put(#1,#2){\color{green} \circle{#3}}
}

% ------------------------------------------------------------
% note on figure
\def\mynote#1#2#3{
\put(#1,#2){\color{green} \textbf{#3}}
}

\def\biglabel#1#2#3{
\put(#1,#2){\Huge \textbf{#3}}
}

\def \largelabel#1#2#3{
\put(#1,#2){\Large \textbf{#3}}
}


\def \normallabel#1#2#3{
\put(#1,#2){\normalsize \textbf{#3}}
}




\def\waxes{\put(-0,0){\color{white}\vector(1,0){10}}
\color{white} \klabellarge{10.5}{-1}{\color{white} {x}}
\put(-0,0){\color{white} \vector(0,-1){10}}
\color{white} \klabellarge{-1.2}{-14.5}{\color{white} {z}}}


\def\axes{\put(-0,0){\vector(1,0){10}}
\klabellarge{10.5}{-1}{\bf{x}}
\put(-0,0){\vector(0,-1){10}}
\klabellarge{-1.2}{-14.5}{\bf{z}}}



\def\putaxes#1#2{\put(#1,#2){\axes}}
\def\putwaxes#1#2{\put(#1,#2){\waxes}}

% huge labels
\def\wlabel#1#2#3{ \white{ \biglabel{#1}{#2}{#3} }}
\def\klabel#1#2#3{ \black{ \biglabel{#1}{#2}{#3} }}
\def\rlabel#1#2#3{ \red{   \biglabel{#1}{#2}{#3} }}
\def\glabel#1#2#3{ \green{ \biglabel{#1}{#2}{#3} }}
\def\blabel#1#2#3{ \blue { \biglabel{#1}{#2}{#3} }}
\def\ylabel#1#2#3{ \yellow{\biglabel{#1}{#2}{#3} }}


% large labels
\def\klabellarge#1#2#3{ \black{ \largelabel{#1}{#2}{#3} }}
\def\wlabellarge#1#2#3{ \white{ \largelabel{#1}{#2}{#3} }}

% normal labels
\def\klabelnormal#1#2#3{ \black{ \normallabel{#1}{#2}{#3} }}
\def\wlabelnormal#1#2#3{ \white{ \normallabel{#1}{#2}{#3} }}

% ------------------------------------------------------------
% centering
\def\cen#1{ \begin{center} \textbf{#1} \end{center}}
\def\cit#1{ \begin{center} \textit{#1} \end{center}}

% emphasis (bold+alert)
\def\bld#1{ \textbf{\alert{#1}}}

% huge fonts
\def\big#1{\begin{center} {\LARGE \textbf{#1}} \end{center}}
\def\hug#1{\begin{center} {\Huge  \textbf{#1}} \end{center}}

% ------------------------------------------------------------
% separator
\def\sep{ \vfill \hrule \vfill}
\def\itab{ \hspace{0.5in}}
\def\nsp{\\ \vspace{0.1in}}

% ------------------------------------------------------------
% integrals

\def\tint#1{\!\!\!\int\!\! #1 dt}
\def\xint#1{\!\!\!\int\!\! #1 d\xx}
\def\wint#1{\!\!\!\int\!\! #1 d\ww}
\def\aint#1{\!\!\!\alert{\int}\!\! #1 d\alert{\xx}}

\def\esum#1{\sum\limits_{#1}}
\def\eint#1{\int\limits_{#1}}

% ------------------------------------------------------------
\def\CONJ#1{\overline{#1}}
\def\MOD#1{\left| {#1} \right|}

% ------------------------------------------------------------
% imaging components

\def\IC{\colorbox{yellow}{\textbf{I.C.}}\;}
\def\WR{\colorbox{yellow}{\textbf{W.R.}}\;}
\def\WE{\colorbox{yellow}{\textbf{W.E.}}\;}
\def\SO{\colorbox{yellow}{\textbf{SOURCE}}\;}
\def\WS{\colorbox{yellow}{\textbf{W.S.}}\;}

% ------------------------------------------------------------
% summary/take home message
\def\thm{take home message}

% ------------------------------------------------------------
\def\dx{\Delta x}
\def\dy{\Delta y}
\def\dz{\Delta z}
\def\dt{\Delta t}

\def\dhx{\Delta h_x}
\def\dhy{\Delta h_y}

\def\kz{{k_z}}
\def\kx{{k_x}}
\def\ky{{k_y}}

\def\kmx{k_{m_x}}
\def\kmy{k_{m_y}}
\def\khx{k_{h_x}}
\def\khy{k_{h_y}}

\def\why{ \alert{\widehat{{\khy}}}}
\def\whx{ \alert{\widehat{{\khx}}}}

\def\lx{{\lambda_x}}
\def\ly{{\lambda_y}}
\def\lz{{\lambda_z}}

\def\klx{k_{\lambda_x}}
\def\kly{k_{\lambda_y}}
\def\klz{k_{\lambda_z}}

\def\mx{{m_x}}
\def\my{{m_y}}
\def\mz{{m_z}}
\def\hx{{h_x}}
\def\hy{{h_y}}
\def\hz{{h_z}}

\def\sx{{s_x}}
\def\sy{{s_y}}
\def\rx{{r_x}}
\def\ry{{r_y}}

% ray parameter (absolute value)
\def\modp#1{\left| \pp_{#1} \right|}

% wavenumber
\def\modk#1{\left| \kk_{#1} \right|}

% ------------------------------------------------------------
\def\kzwk{ {\kz^{\kk}}}
\def\kzwx{ {\kz^{\xx}}}
 
\def\PSk#1{e^{\red{#1 i \kzwk \dz}}}
\def\PSx#1{e^{\red{#1 i \kzwx \dz}}}
\def\PS#1{ e^{\red{#1 i k_z   \dz}}}

\def\TT{t}
\def\TS{t_s}
\def\TR{t_r}

\def\oft{\lp t \rp}
\def\ofw{\lp \ww \rp}

\def\ofx{\lp \xx \rp}
\def\ofk{\lp \kk \rp}
\def\ofs{\lp \ss \rp}
\def\ofr{\lp \rr \rp}
\def\ofz{\lp   z \rp}

\def\ofxt{\lp \xx, t  \rp}
\def\ofst{\lp \ss, t  \rp}
\def\ofrt{\lp \rr, t  \rp}

\def\ofxw{\lp \xx, \ww  \rp}
\def\ofsw{\lp \ss, \ww  \rp}
\def\ofrw{\lp \rr, \ww  \rp}

\def\ofxm{\lp \xx,\hh \rp}

\def\ofxmp{\lp \xx+\hh \rp}
\def\ofxmm{\lp \xx-\hh \rp}

\def\ofmm{\lp \mm      \rp}
\def\ofmz{\lp \mm, z   \rp}
\def\ofmw{\lp \mm, \ww \rp}
\def\ofkm{\lp \kkm     \rp}

% ------------------------------------------------------------
% source/receiver data and wavefields

\def\dst{$\DS\ofst$}
\def\drt{$\DR\ofrt$}
\def\ust{$\US\ofxt$}
\def\urt{$\UR\ofxt$}

\def\dsw{$\DS\ofsw$}
\def\drw{$\DR\ofrw$}
\def\usw{$\US\ofxw$}
\def\urw{$\UR\ofxw$}

% ------------------------------------------------------------
\def\Nx{N_x}
\def\Ny{N_y}
\def\Nz{N_z}
\def\Nt{N_t}
\def\Nw{N_{\ww}}
\def\Nm{N_{\mm}}

\def\Nlx{N_{\lambda_x}}
\def\Nly{N_{\lambda_y}}
\def\Nlz{N_{\lambda_z}}
\def\Nlt{N_{\tau}}

\def\wmin{\ww_{min}}
\def\wmax{\ww_{max}}
\def\zmin{z_{min}}
\def\zmax{z_{max}}
\def\tmin{t_{min}}
\def\tmax{t_{max}}
\def\lmin{\hh_{min}}
\def\lmax{\hh_{max}}
\def\xmin{\xx_{min}}
\def\xmax{\xx_{max}}

% ------------------------------------------------------------
% course qualifiers

\def\fun{\hfill \alert{concepts}}
\def\pra{\hfill \alert{applications}}
\def\fro{\hfill \alert{frontiers}}


% ------------------------------------------------------------
% wavefield extrapolation
\def\ws{ {\ww s} }

\def\kows{\lp \frac{\kx}{\ws} \rp}

\def\kmws{\lp \frac{\modk{\mm}}{\ws} \rp}
\def\kzws{\lp \frac{\kz}       {\ws} \rp}

\def\S{\lb\frac{\modk{\mm}}{\ws  }\rb}
\def\C{\lb\frac{\modk{\mm}}{\ws_0}\rb}
\def\K{\lb\frac{\modk{\mm}}{\ww  }\rb}

\def\Cs{\lb\frac{\modk{\mm}^2}{\lp \ws_0 \rp^2}\rb}

\def\SSR#1{  \sqrt{ \lp \ww {#1} \rp^2 - \modk{\mm}^2} }

\def\SQRsum#1{\sum\limits_{n=1}^{\infty} \lp -1 \rp^n
		\displaystyle{\frac{1}{2} \choose n} #1}

\def\TSE#1#2#3#4{\sum\limits_{#4=#3}^{\infty} \lp -1 \rp^#4
		\displaystyle{#2 \choose #4} {#1}^#4}

\def\onefrac#1#2{\frac{#2^2}{a_#1+b_#1 #2^2}}
\def\SQRfrac#1{
	\sum\limits_{n=1}^{\infty}
	\onefrac{n}{#1} }

\def\dkzds { \left. \frac{d {\kz}}  {d s} \right|_{s_b} }
\def\SSX#1#2{\sqrt{ 1 - \lb \frac{\MOD{#2}}{#1} \rb^2} }
\def\SST#1#2{1 + \sum_{j=1}^N c_j \lb \frac{\MOD{#2}}{#1} \rb^{2j} }

% ------------------------------------------------------------
% acknowledgment
\def\ackcwp{\cen{the sponsors of the\\Center for Wave Phenomena\\at\\Colorado School of Mines}}

% ------------------------------------------------------------
% citation in slides
\def\talkcite#1{{\small \sc #1}}

% ------------------------------------------------------------
\def\ise{GPGN302: Introduction to Electromagnetic and Seismic Exploration}
\def\inv{GPGN409: Inversion}

% ------------------------------------------------------------
\def\model{m}
\def\data {d}

\def\Lop{ {\mathbf{L}}}
\def\Sop{ {\mathbf{S}}}
\def\Eop{ {\mathbf{E}}}
\def\Iop{ {\mathbf{I}}}
\def\Aop{ {\mathbf{A}}}
\def\Pop{ {\mathbf{P}}}
\def\Fop{ {\mathbf{F}}}


% ------------------------------------------------------------
\def\mybox#1{
  \begin{center}
    \fcolorbox{black}{yellow}
    {\begin{minipage}{0.8\columnwidth} {#1} \end{minipage}}
  \end{center}
}

\def\hibox#1{
  \begin{center}
    \fcolorbox{black}{LightGreen}
    {\begin{minipage}{0.8\columnwidth} {#1} \end{minipage}}
  \end{center}
}

% ------------------------------------------------------------
% Nota Bene
\def\nbnote#1{
  \vfill
  \begin{center}
    \colorbox{LightGray}
    {\begin{minipage}{\columnwidth} {\textbf{\black{\large N.B.}} #1} \end{minipage}}
  \end{center}
}

\def\highlight#1{
  \begin{center}
    \colorbox{cyan}
    {\begin{minipage}{\columnwidth} {#1} \end{minipage}}
  \end{center}
}

% ------------------------------------------------------------
\def\pcsshaded#1{
  \definecolor{shadecolor}{rgb}{0.8,0.8,0.8}
  \begin{shaded} {#1} \end{shaded}
  \definecolor{shadecolor}{rgb}{1.0,1.0,1.0}
}

% ------------------------------------------------------------
\def\postit#1{
  \begin{center}
    \colorbox{yellow}
    {\begin{minipage}{0.66\columnwidth} {#1} \end{minipage}} 
  \end{center}
}

% ------------------------------------------------------------
\def\graybox#1{
  \begin{center}
    \colorbox{LightGray}
    {\begin{minipage}{1.00\columnwidth} {#1} \end{minipage}}
  \end{center}
}

\def\whitebox#1{
  \begin{center}
    \colorbox{white}
    {\begin{minipage}{1.00\columnwidth} {#1} \end{minipage}}
  \end{center}
}

\def\yellowbox#1{
  \begin{center}
    \colorbox{LightYellow}
    {\begin{minipage}{1.00\columnwidth} {#1} \end{minipage}}
  \end{center}
}

\def\greenbox#1{
  \begin{center}
    \colorbox{LightGreen}
    {\begin{minipage}{1.00\columnwidth} {#1} \end{minipage}}
  \end{center}
}

\def\bluebox#1{
  \begin{center}
    \colorbox{LightBlue}
    {\begin{minipage}{1.00\columnwidth} {#1} \end{minipage}}
  \end{center}
}

\def\redbox#1{
  \begin{center}
    \colorbox{LightRed}
    {\begin{minipage}{1.00\columnwidth} {#1} \end{minipage}}
  \end{center}
}

\def\hyellow#1{ \colorbox{yellow} #1 }
\def\hgreen #1{ \colorbox{green}  #1 }

% ------------------------------------------------------------
% boxes for vectors and matrices

\def\pcsbox#1#2#3#4{
  % #1 = width
  % #2 = height
  % %3 = hmax
  \begin{picture}(#3,#1)
    \linethickness{0.5mm}
    % 
    \multiput(0,#1)(#3, 0){2}{\line(0,-1){#2}}
    \multiput(0,#1)(0,-#2){2}{\line(+1,0){#3}}
    % 
    \put(1,3){#4}
  \end{picture}
}

\def\pcssym#1#2{
  \begin{picture}(1,#1)
    \put(0,3){#2}
  \end{picture}
}

\def\sidebyside#1#2{
  \begin{center}
    \colorbox{LightBlue}{
      \begin{minipage}{1.0\columnwidth} {#1} \end{minipage}
    }
    \colorbox{LightYellow}{
      \begin{minipage}{1.0\columnwidth} {#2} \end{minipage}
    }
  \end{center}
}

\def\sidebysidebyside#1#2#3{
  \begin{center}
    \colorbox{LightBlue}{
      \begin{minipage}{1.0\columnwidth} {#1} \end{minipage}
    }
    \colorbox{LightYellow}{
      \begin{minipage}{1.0\columnwidth} {#2} \end{minipage}
    }
    \colorbox{LightRed}{
      \begin{minipage}{1.0\columnwidth} {#3} \end{minipage}
    }
  \end{center}
}


\maketitle

\section{2D diffusion equation}

Here, we want to solve the homogeneous 2D diffusion equation:

\beq
\frac{\partial u}{\partial t} -0.5\nabla^2u = 0,\text{    in  } \Omega^2=(0,1)\times(0,1),\forall \text{  } t>0
\label{eq:poisson}
\eeq
with boundary data:
\beq
u(x,y,t) = \sin(c\pi x)\cos(c\pi y) \exp{-c^2\pi^2t}
\label{eq:boundary}
\eeq
and initial state:
\beq
u(x,y,0) = \sin(c\pi x)\sin(c\pi y),\text{    with } c \in (0,5] 
\label{eq:heqin}
\eeq

This problem represents the diffusion of an initial temperature profile over time. We
should expect that the energy of the solution decays fast over time.

\subsection{Grid discretization}
One can discretize the problem described in 
equations~\ref{eq:poisson}-\ref{eq:heqin} in the following grid:
\[
x_n = hi, i = 0,1,2,...,Nx=N
\]
\[
y_n = hj, j = 0,1,2,...,Ny=N
\]
\[
t_m = {dt} k, k = 0,1,2,...,M
\]
with $h = 1/N$ and $dt=0.1/M$,
from which we have to solve for $U^k_{i,j}$ with $i=1,N-1$, $j=1,N-1$ and $k=1,M$.


Each time-step $k$ can be represented as shown in table~\ref{tab:2darray}.
The model can be rearranged as a 1D vector for of the same 2D space as shown
in equation~\ref{eq:2darray_vform}.


\begin{table}
\centering
  \begin{tabular}{|c | c | c | c|}
\hline
$U_{1,1}$  &  $U_{1,2}$  & $\dots$  & $U_{1,N-1}$ \\  \hline
$U_{2,1}$  &  $U_{2,2}$  & $\dots$  & $U_{2,N-1}$ \\  \hline 
$\vdots$   &  \dots  & \dots    & $\vdots$    \\  \hline
$U_{N-1,1}$  &  $U_{N-1,2}$  & \dots  & $U_{N-1,N-1}$ \\  \hline
  \end{tabular}
\caption{representation of $U^k$ as a 2D array}
\label{tab:2darray}
\end{table}

 \begin{equation}
\mathbf{U^k}=
 \begin{pmatrix}
  U^k_{1,1} \\
  U^k_{1,2} \\
  \vdots   \\
  U^k_{1,N-1} \\
  U^k_{2,1} \\
  U^k_{2,2} \\
  \vdots  \\ 
  U^k_{N-1,N-1} \\
 \end{pmatrix}_{(N-1)^2},
\label{eq:2darray_vform}
\end{equation}


\subsection{Solution}
To solve the problem described in equations (1)-(3), I use the Crank-Nicolson
method, which give us second order accuracy in both space and time.

Let 
\[
L = \delta x^2+ \delta y^2
\]
be the second order discrete representation of the 
Laplacian $\nabla^2$ operator, then the application of this operator to a given $U^k$ is:

\[
h^2 L U^k = \uk{k}{i-1}{j}-2\uk{k}{i}{j}+\uk{k}{i+1}{j} + \uk{k}{i}{j-1}-2\uk{k}{i}{j}+\uk{k}{i}{j+1}.
\]

Then, the C-N solution for equations (1)-(3) can be written as:
\[
\frac{\ukk{k} -\ukk{k-1}}{\dt} -\frac{a}{2h^2}\left[ L\ukk{k-1}+L\ukk{k} \right] =0
\]

We can also represent the negative of the operator $L$ in matrix form as:

\[
-{\bf L} = {\bf I}\otimes {\bf B_h} +{\bf B_h}\otimes {\bf I}.
\]
The above matrix is SPD.

With 
\begin{equation}
\mathbf{B_h}=\frac{1}{h^2}
 \begin{pmatrix}
  2       & -1      &  0      & 0  \\
  -1      &  2      & \ddots  & 0  \\
  0       & \ddots  & \ddots  & -1 \\
  0       &  0      &  -1     &  2  
 \end{pmatrix}_{N-1,N-1},
\label{eq:B}
\end{equation}

and $\bf I$ being the identity matrix. Matrix $-\bf L$ is a block diagonal matrix:

\begin{equation}
-\mathbf{L}=
 \begin{pmatrix}
  {\bf B_h} & 1      &  0      & 0  \\
  0      &   {\bf B_h}      & \ddots  & 0  \\
  0       & \ddots  & \ddots  & 0 \\
  0       &  0      &  0     &  {\bf B_h}   
 \end{pmatrix}_{(N-1)^2,(N-1)^2}+\frac{1}{h^2}
 \begin{pmatrix}
  2I       & -I      &  0      & 0  \\
  -I      &  2I      & \ddots  & 0  \\
  0       & \ddots  & \ddots  & -I \\
  0       &  0      &  -I     &  2I  
 \end{pmatrix}_{(N-1)^2,(N-1)^2}
\end{equation},

so the solution can be written as:
\[
\frac{\ukk{k} -\ukk{k-1}}{\dt} +\frac{a}{2h^2}\left[ \bf L\ukk{k-1}+\bf L\ukk{k} \right] =0.
\]

Let $\alpha = \frac{a\dt}{2h^2}$, then:
\[
\left( \bf I +\frac{\alpha}{2} \bf L \right) \ukk{k} = \left(I -\frac{\alpha}{2} \bf L\right) \ukk{k-1}
\]
I now introduce the boundary data vector $\bf C$ which contains the data not added in the Laplacian operator at the
boundaries for $k$ and $k-1$:
\beq
\left( \bf I +\frac{\alpha}{2} \bf L \right) \ukk{k} = \left(I -\frac{\alpha}{2} \bf L\right) \ukk{k-1} +\bf C
\label{eq:ls}
\eeq

The boundary data can be represented as a 2D array and then reshaped to 1D, the 2D representation is shown in table~\ref{tab:C2darr}. 

\begin{table}
\centering
  \begin{tabular}{|c | c | c | c|}
\hline
$\uk{k}{1}{1}+\ukk{k-1}{1}{1}$    & $\uk{k}{1}{2}+\uk{k-1}{1}{2}$  & $\dots$  &$\uk{k}{1}{N-1}+\uk{k-1}{1}{N-1}$ \\  \hline
$\uk{k}{2}{1}+\uk{k-1}{2}{1}$    &  0  & $\dots$  & $\uk{k}{2}{N-1}+\uk{k-1}{2}{N-1}$ \\  \hline 
$\uk{k}{3}{1}+\uk{k-1}{3}{1}$    &  0  & $\dots$  & $U_{3,N-1}$ \\  \hline 
$\vdots$   &  \dots  & \dots    & $\vdots$    \\  \hline
$\uk{k}{N-1}{1}+\uk{k-1}{N-1}{1}$ & $\uk{k}{N-1}{2}+\uk{k-1}{N-1}{2}$  & \dots  & $\uk{k}{N-1}{N-1}+\uk{k-1}{N-1}{N-1}$ \\  \hline
  \end{tabular}
\caption{representation of $C^k$ as a 2D array}
\label{tab:C2darr}
\end{table}

The lhs of equation~\ref{eq:ls} is SPD, and therefore an unique solution exist and can be efficiently solved.

\subsection{Truncation error}
Similar to the 1D case presented in homework 1, the truncation error can be written as:

\begin{equation}
  \begin{split}
  \tau^{C-N} (x,y,t)=& \left[\dut{t}(x,y,t)+\frac{1}{6}\dut{t}(x,y,t)\left(\frac{\dt}{2}\right)^2\right] +\mathcal{O}({\dt}^2)  \\
                   &-\frac{a}{2} \left[\du{x}{2}(x,y,t+0.5\dt)+\frac{1}{12}\du{x}{4}(x,y,t+0.5\dt)h^2 +\mathcal{O}({h}^4)\right]  \\
                   &-\frac{a}{2} \left[\du{x}{2}(x,y,t-0.5\dt)+\frac{1}{12}\du{x}{4}(x,y,t-0.5\dt)h^2 +\mathcal{O}({h}^4)\right] \\
                   &-\frac{a}{2} \left[\du{y}{2}(x,y,t+0.5\dt)+\frac{1}{12}\du{y}{4}(x,y,t+0.5\dt)h^2 +\mathcal{O}({h}^4)\right]  \\
                   &-\frac{a}{2} \left[\du{y}{2}(x,y,t-0.5\dt)+\frac{1}{12}\du{y}{4}(x,y,t-0.5\dt)h^2 +\mathcal{O}({h}^4)\right] \\
                   & -\du{t}{}(x,t)+a\du{x}{2}(x,y,t)+a\du{y}{2}(x,y,t)
  \end{split}
\end{equation}

As we can see, the space derivatives are shifted $\pm 0.5\dt$, this comes from the fact that we did an averaging operation (inherent 
to the method). Therefore, we have to introduce the truncation error of the average:
\[
\frac{1}{2}\left[ W(t+0.5\dt)+W(t-0.5\dt)\right]=W(t) +\frac{1}{8}\dt^2W^{''}(t) + \mathcal{O}(\dt^4)
\]

Then,
\begin{equation}
  \begin{split}
  \tau^{C-N} (x,t)= &\frac{1}{24} \du{t}{3}(x,t)\dt^2 + \mathcal{O}(\dt^4) \\  
                    &-a \left[\du{x}{2}(x,y,t) +\frac{1}{8}\dt^2 \frac{\partial^4u}{\partial t^2\partial x^2}(x,y,t) +\mathcal{O}(\dt^4) +% 
                      \frac{1}{12}\du{x}{4}(x,y,t)h^2 +\mathcal{O}(\dt^2h^2) \right] \\
                    &-a \left[\du{y}{2}(x,y,t) +\frac{1}{8}\dt^2 \frac{\partial^4u}{\partial t^2\partial y^2}(x,y,t) +\mathcal{O}(\dt^4) +% 
                      \frac{1}{12}\du{x}{4}(x,y,t)h^2 +\mathcal{O}(\dt^2h^2) \right] \\
                    & + a\du{x}{2}(x,y,t) +a\du{y}{2}(x,y,t) = \mathcal{O}(\dt^2+h^2+h^2) 
  \end{split}
\end{equation}

\section{Stability analysis of the C-N method}

I will do a stability analysis similar to homework 1.

let $u(x,t) = W(t) \exp{i2\pi p x}$ be a solution for the heat equation, then the discretized form is:
\[
\uk{k}{x}{y} = W^k \exp{i2 \pi p_x x_n}\exp{i2 \pi p_y y_n}
\]

We can later construct a general solution by linearly combining many Fourier modes $p_x,p_y$.

Now we introduce the solution in the heat equation solved by the C-N method:
\begin{equation}
  \begin{split}
   & \frac{W^{k+1}-W^k}{\dt}\exp{i2 \pi p_x x}\exp{i2 \pi p_y y} \\
   & -\frac{a}{2h^2}\left(W^k+W^{k+1}\right)\exp{i2 \pi p_y y}\left[\exp{i2\pi p_x x_{n+1}}-2\exp{i2\pi p_x x_n} + \exp{i2\pi p_x x_{n-1}}\right] \\
   & -\frac{a}{2h^2}\left(W^k+W^{k+1}\right)\exp{i2 \pi p_x x}\left[\exp{i2\pi p_y y_{n+1}}-2\exp{i2\pi p_y y_n} + \exp{i2\pi p_y y_{n-1}}\right] = 0
  \end{split}
\end{equation}
This then becomes:
\begin{equation}
  \begin{split}
  &\frac{W^{k+1}-W^k}{\dt} \\
  &-\frac{a}{2h^2}\left(W^k+W^{k+1}\right)\left[\exp{i2\pi p_x h}-2+ \exp{-i2\pi p_x h}\right] \\ 
  &-\frac{a}{2h^2}\left(W^k+W^{k+1}\right)\left[\exp{i2\pi p_y h}-2+ \exp{-i2\pi p_y h}\right] = 0
  \end{split}
\end{equation}
\begin{equation}
  \frac{W^{k+1}-W^k}{\dt}% 
  +\frac{\alpha}{2}\left(W^k+W^{k+1}\right)4\sin^2(2\pi p_x h)% 
  +\frac{\alpha}{2}\left(W^k+W^{k+1}\right)4\sin^2(2\pi p_x h) = 0  
\end{equation}


\[
W^{k+1} = W^k \frac{1 -2\alpha\sin^2(2\pi p_x h)-2\alpha\sin^2(2\pi p_y h)}{ 1 +2\alpha\sin^2(2\pi p_x h)+2\alpha\sin^2(2\pi p_y h)}
\] 
The denominator is always greater than the numerator; therefore, $|W^{k+1}|<|W^k |$, hence the C-N method is absolutely stable.


\subsection{Convergence analysis}
Equation~\ref{eq:boundary} is the solution for this problem. Therefore, we can use as benchmark for the code. I tested with different
grid spacing ($h=\dt = 0.02/2^k, k=1,2,3,4$). The convergence rate can be found as:
$r = \log(e_{2h}/e_h)/\log(2)$.

\begin{table}
\centering
  \begin{tabular}{c | c | c | c| c}
 k        &    N=M    &  h &  error($L_\infty$)       &  r          \\  \hline
 1        &   11      & 0.01& 0.0049661009&  -- \\
2&   21& 0.0050& 0.0026126372& 0.92660676 \\ 
3&   41& 0.0025& 0.00069737615&  1.9054979 \\
4&   81& 0.00125000& 0.00017773729&  1.9721906 
  \end{tabular}
\caption{Convergence analysis table. }
\label{tab:table}
\end{table}

Because of computational limitations I could not run the code for greater k values, however for $k=3,4$ the rate
approaches to the expected convergence of 2.






\section{2D Wave equation}
Here, we want to solve the homogeneous 2D wave equation:

\beq
\du{t}{2} -4\nabla^2u = 0,\text{    in  } \Omega^2=(0,1)\times(0,1),\forall \text{  } t>0
\label{eq:wave}
\eeq
and initial state:
\beq
v(x,y) = \left\{
     \begin{array}{lr}
       \cos^2\left(\frac{\pi}{2}\frac{\sqrt{(x-0.5)^2+(y-0.5)^2}}{R}\right)  & : \sqrt{(x-0.5)^2+(y-0.5)^2} \leq R  \\
       0 & : R< \sqrt{(x-0.5)^2+(y-0.5)^2}.
     \end{array}
   \right.
\eeq


This problem mimics a membrane exited at $t=0$ without further energy introduced into the system. There is no
energy loss in this problem; therefore, the membrane should vibrate forever.

The grid discretization is exactly as the diffusion problem. 

\subsection{Solution}
In order to achieve second order convergence in space and time, I discretized the problem using $\dt^2$ and $\delta x^2$ operators.
Therefore, equation~\ref{eq:wave} can be written as:

\beq
\uk{k-1}{i}{j}-2\uk{k}{i}{j}+\uk{k+1}{i}{j} -4\dt \frac{1}{h^2} \left(\uk{k}{i-1}{j}-2\uk{k}{i}{j}+\uk{k}{i+1}{j} + \uk{k}{i}{j-1}-2\uk{k}{i}{j}+\uk{k}{i}{j+1}\right) =0
\eeq

and hence can be solved explicitly as:

\beq
\uk{k+1}{i}{j} = 4\dt \frac{1}{h^2} \left(\uk{k}{i-1}{j}-2\uk{k}{i}{j}+\uk{k}{i+1}{j} + \uk{k}{i}{j-1}-2\uk{k}{i}{j}+\uk{k}{i}{j+1}\right) -\uk{k-1}{i}{j}+2\uk{k}{i}{j}
\eeq

Since this is a three level implementation, in order to start the solution we need  $U^0_{i,j} = v(ih,jh)$, and $U^1_{i,j}$. To solve for $U^1_{i,j}$ we can
make use of the Taylor expansion:
\[
U^1_{i,j} = U^0_{i,j} +\dt \dut{t}(x,y,0) +\frac{\dt^2}{2} \du{t}{2}(x,y,0) + \mathcal{O}(\dt^4)
\]

In this problem $ \dt \dut{t}(x,y,0) =0$. We know that $ \du{t}{2}(x,y,0) = c^2\nabla^2u(x,y,0)=c^2\nabla^2v(x,y,0) \approx c^2 LU^0_{i,j} +\mathcal{O}(\dx^2)$. Therefore,
\beq
U^1_{i,j} = U^0_{i,j} +\frac{\dt^2}{2} c^2 LU^0_{i,j}+ \mathcal{O}(\dt^4)
\label{eq:u1}
\eeq

We now can solve for $U^k_{i,j}, k=2,...,M; i=1,...,N-1; j=1,...,N-1$.

This FD explicit method is subject to the CFL condition, which is necessary but not sufficient for stability. The CFL condition imposes the following 
restriction over the grid:
\beq
  c\frac{\dt}{h} \leq 1,
\eeq
which for our case implies $\dt \leq \frac{h}{c} $.

\subsection{Truncation error}
In this case the truncation error is more simpler, and can be written as:
\begin{equation}
  \begin{split}
  \tau^{C-N} (x,y,t)=& \left[\du{t}{2}(x,y,t)+\frac{1}{12}\du{t}{4}(x,y,t)\left(\frac{\dt}{2}\right)^2\right] +\mathcal{O}({\dt}^4)  \\
                   &-c^2 \left[\du{x}{2}(x,y,t)+\frac{1}{12}\du{x}{4}(x,y,t)\left(\frac{\dx}{2}\right)^2 +\mathcal{O}({\dx}^4) \right]  \\
                   &-c^2 \left[\du{y}{2}(x,y,t)+\frac{1}{12}\du{y}{4}(x,y,t)\left(\frac{\dy}{2}\right)^2 +\mathcal{O}({\dy}^4) \right]  \\
                   & -\du{t}{2}(x,y,t)+c^2\du{x}{2}(x,y,t)+c^2\du{y}{2}(x,y,t) \\
                   =& \mathcal{O}(\dt^2+h^2+h^2) 
  \end{split}
\end{equation}

\subsection{Stability analysis}

We can introduce a solution using Fourier modes of the form:
\[
\uk{k}{x}{y} = W^k \exp{i2 \pi p_x x_n}\exp{i2 \pi p_y y_n},
\]
similar to problem 1. 

Both problems are similar, the only difference is in the time discretization:


\begin{equation}
  \begin{split}
   & \frac{W^{k+1}-2W^k+W^{k-1}}{\dt^2}\exp{i2 \pi p_x x}\exp{i2 \pi p_y y} \\
   & -\frac{c^2}{h^2}W^k\exp{i2 \pi p_y y}\left[\exp{i2\pi p_x x_{n+1}}-2\exp{i2\pi p_x x_n} + \exp{i2\pi p_x x_{n-1}}\right] \\
   & -\frac{c^2}{h^2}W^k\exp{i2 \pi p_x x}\left[\exp{i2\pi p_x y_{n+1}}-2\exp{i2\pi p_y y_n} + \exp{i2\pi p_y y_{n-1}}\right] = 0
  \end{split}
\end{equation}
This then becomes:
\begin{equation}
   W^{k+1}= 2W^k\left(1-\frac{c^2\dt^2}{2h^2}\left[ \sin^2(2\pi p_x h)+\sin^2(2\pi p_y h) \right]\right)-W^{k-1}
\label{eq:weqk}
\end{equation}

to prove stability we need $|W^{k+1}| \leq |W^{k}|$.
Using equation~\ref{eq:u1} we can obtain $W^1$:
\beq
W^1 = W^0(1 -\frac{c^2\dt^2}{2h^2}\left[ \sin^2(2\pi p_x h)+\sin^2(2\pi p_y h) \right]).
\label{eq:weq1}
\eeq
Let $P = (1 -\frac{c^2\dt^2}{2h^2}\left[ \sin^2(2\pi p_x h)+\sin^2(2\pi p_y h) \right])$. 
Then, introducing equation~\ref{eq:weq1} in~\ref{eq:weqk} we obtain:
\[
W^{2} = W^0(P^2 -1).
\]
If the CFL condition is satisfied, then $0\leq P \leq 1$.
Therefore,
\[
|W^2| = |W^0||(P^2-1)| \leq |W^0|
\]
Therefore, the method is stable if and only if the CFL condition is satisfied.


\subsection{Convergence analysis}
I generated an ``exact solution'' using a very fine grid $\dt = .000625, h = 4\dt$. I then,
generated other solutions by varying only $h$. I did this because in this way I know
that since the temporal grid is the same, I can compare the solutions at $(x,y)=(0.5,0.5)$.
Since I am keeping $\dt$ constant, I expect the error to be proportional to:
\[
e \propto \mathcal{O}(\dx^2), 
\]
because the error from the temporal grid is the same in both ``u exact'' and $u_h$. 

I implemented the code in Python, which is not very efficient, so I could not test
other approaches (varying both $\dt$ and $h$) in a timely manner. I am sure that my implementation
is correct, because the propagation is stable and satisfies the expected behaviour for the
vibrating membrane. 

For the last case $N=201$ the convergence approaches to $2$. I might need to run 
the code for other grids, and produce a better ``u exact''. I think it will converge to
 the right rate, I just could not test it due to the extensive computing.


\begin{table}
\centering
  \begin{tabular}{c | c | c | c| c}
 N        &    M    &  h &  error($l_2$)       &  r           \\  \hline
25        & 1601    & 1/24  & 0.00163644115806 & --           \\
51        & 1601    & 1/60  & 0.000476440841329& 1.78019273404 \\ 
101       & 1601    & 1/100 & 0.000153565044924 & 1.63344722726 \\ 
201       & 1601    & 1/200 & 4.21993310577e-05 & 1.86355782687
  \end{tabular}
\caption{Error analysis}
\label{tab:table2}
\end{table}




\section{results}

\subsection{Diffusion problem}

Here I plot some snapshots of the movie. The field diffuses as expected.

\begin{figure}
  \includegraphics[width=0.8\textwidth]{Fig/heq/frame-it00.png}
\caption{Diffusion problem solution $t=0.0$}
\end{figure}

\begin{figure}
  \includegraphics[width=0.8\textwidth]{Fig/heq/frame-it05.png}
\caption{Diffusion problem solution $t=0.025$}
\end{figure}
\begin{figure}
  \includegraphics[width=0.8\textwidth]{Fig/heq/frame-it10.png}
\caption{Diffusion problem solution $t=0.050$}
\end{figure}



\begin{figure}
  \includegraphics[width=0.8\textwidth]{Fig/heq/frame-it15.png}
\caption{Diffusion problem solution $t=0.075$}
\end{figure}

\begin{figure}
  \includegraphics[width=0.8\textwidth]{Fig/heq/frame-it20.png}
\caption{Diffusion problem solution at the terminal time $t=0.1$}
\end{figure}


\subsection{Wave problem}

Here I plot some snapshots of the movie. The problem behaves as expected.

\begin{figure}
  \includegraphics[width=0.8\textwidth]{Fig/weq/frame-it00.png}
\caption{Wavefield snapshot}
\end{figure}

\begin{figure}
  \includegraphics[width=0.8\textwidth]{Fig/weq/frame-it08.png}
\caption{Wavefield snapshot}

\end{figure}
\begin{figure}
  \includegraphics[width=0.8\textwidth]{Fig/weq/frame-it16.png}
\caption{Wavefield snapshot}

\end{figure}

\begin{figure}
  \includegraphics[width=0.8\textwidth]{Fig/weq/frame-it24.png}
\caption{Wavefield snapshot}

\end{figure}

\begin{figure}
  \includegraphics[width=0.8\textwidth]{Fig/weq/frame-it32.png}
\caption{Wavefield snapshot}

\end{figure}

\begin{figure}
  \includegraphics[width=0.8\textwidth]{Fig/weq/frame-it40.png}
\caption{Wavefield snapshot}

\end{figure}


\section{Codes}

\subsection{Diffusion code}
The main code is in 'hw04\_heq.py' which calls the plotting an convergence tests routines.
The main class (which is here) is in heat2dequation.py.

\begin{verbatim}
import numpy as np
from scipy import *
from scipy.sparse import dia_matrix,linalg,kron


class CrankNicolson2d:

  def __init__(self,g,N,c=2.5,a=0.5,jsnap=10):
    self.nt = N  # number of grid points
    self.ot = 0.
    self.dt = .1/(N-1)
    self.ntsnap = N/jsnap+1
    self.dtsnap = self.dt*jsnap

    self.ox = 0.
    self.nx = int(1./self.dt)+1
    self.dx = self.dt

    self.oy = 0.
    self.ny = self.nx
    self.dy = self.dx

    self.jsnap = jsnap # how many snapshots to avoid for the movie?
    self.n = self.nx-2 # number of unknowns within the domain
    self.g = g
    self.a = a # conductivity
    self.c = c
    self.alpha =  self.a*self.dt/(self.dx*self.dx)

    self.um1  = np.zeros(self.n*self.n)    
    self.u0   = np.zeros(self.n*self.n)    
    self.utmp = np.zeros(self.n*self.n)    
    self.umovie = np.zeros((self.ntsnap,self.nx,self.nx)) 
    self.umovex = np.zeros((self.ntsnap,self.nx,self.nx)) 
    self._fillbounds()
    print 'nt = %d   nx = %d  ny = %d'%(self.nt,self.nx,self.ny)

  def solve(self):
    self._fillt0()
    A = self._getA()
    self._Arhs()
    u = np.reshape(self.um1,(self.n,self.n))
    isnap = 0
    self.umovie[isnap,1:self.nx-1,1:self.ny-1] = u
    for it in range(1,self.nt,1):

      b = self._rhs(it)       
      self.u0 = linalg.spsolve(A,b) # linear solver for sparse A
      if ((it)%self.jsnap == 0):
        print '%d/%d'%(it,self.nt)
        isnap+=1
        u = np.reshape(self.u0,(self.n,self.n))    
        self.umovie[isnap,1:self.nx-1,1:self.ny-1] = u
      self.um1 = self.u0  # swap pointers 
      self.u0 = np.zeros(self.n*self.n)
    return self.umovie

  # private methods:
  def _fillt0(self):
    d = self.dx
    n = self.n
    y = x = np.linspace(d,1.-d,n)
    self.um1 = np.reshape(self.um1,(n,n))
    for ix in range(n):  
      for iy in range(n):  
        self.um1[ix,iy] = self.g(self.c,x[ix],y[iy],0.) 
    self.um1 = np.reshape(self.um1,(n*n))

  def _getA(self,lhs=True):
    '''
    gets the left hand side of the linear system of 
    equations.
    If lhs=False it returns the matrix needed     
    in the right hand side
    '''
    n = self.n
    n2 = n*n
    bh = self._laplace() 
    I  = dia_matrix((ones(n2),array([0])),shape=(n2,n2))
    if lhs:
      A = I + 0.5*self.alpha*bh
    else:
      A = I - 0.5*self.alpha*bh
    return A

  def _laplace(self):
    '''
    construct Laplace operator as a matrix
    based on user input filter.
    Actually this matrix can be thought
    as a convolution operator:
    f(x,z)*U(x,z)
    '''
    f=[-1.,2.,-1.]
    nx = nz = self.n
    nf = len(f)
    nonzero = np.ones((nf,nx))
    for i in range(nf):
      nonzero[i] *=f[i]
    offsets = array(range(nf/2,-nf/2 ,-1))
  
    m1 = dia_matrix((nonzero,offsets),shape=(nx,nx))
    m2 = identity(nz)
    k1 = kron(m1,m2)
    nonzero = np.ones((nf,nz))
    for i in range(nf):
      nonzero[i,:] *=f[i]
    m1 = dia_matrix((nonzero,offsets),shape=(nz,nz))
    m2 = identity(nx)
    k2 = kron(m2,m1)   
    return k1+ k2

  def _gett(self,it):
    return it*self.dt +self.ot

  def _getx(self,ix):
    return ix*self.dx +self.ox

  def _Arhs(self):
    self.Arhs = self._getA(False)

  def _rhs(self,it):
    um1 = self.um1 
    b = self.Arhs*um1 
    t0 = self._gett(it) 
    tm1 = self._gett(it-1) 
    b += 0.5*self.alpha*(self._g1(t0)+self._g1(tm1))
    return b

  def _fillbounds(self):
    '''
    this method introduces the boundary data
    '''
    x = np.linspace(0.,1.,self.nx)
    y = np.linspace(0.,1.,self.nx)

    for it in range(len(self.umovie)):
      t = self.ot +it*self.dt*self.jsnap 
      for iy in range(self.ny):
        yy = y[iy]
        self.umovie[it,0,iy] = self.g(self.c,0,yy,t)
        self.umovie[it,self.nx-1,iy] = self.g(self.c,x[self.nx-1],yy,t)

      for ix in range(self.nx):
        xx = x[ix]
        self.umovie[it,ix,0] = self.g(self.c,xx,0,t)
        self.umovie[it,ix,self.ny-1] = self.g(self.c,xx,y[self.ny-1],t)

  
  def _g1(self,t):
    n = self.n
    c = self.c
    u = np.zeros((n,n))
    ux = np.zeros(n)
    uy = np.zeros(n)
    d = 1./(n+1)
    y = x = np.linspace(d,1.-d,n)
    ux0 = self.g(self.c,x,0.,t) 
    ux1 = self.g(self.c,x,1.,t) 
    u0y = self.g(self.c,0.,y,t) 
    u1y = self.g(self.c,1.,y,t)
 
    for ix in range(n):
      u[ix,0  ] = ux0[ix]
      u[ix,n-1] = ux1[ix]

    for iy in range(n):
      u[n-1,iy] += u1y[iy]
      u[0  ,iy] += u0y[iy]
    return np.reshape(u,n**2)


  def uexact(self):
    jsnap = self.jsnap
    x = np.linspace(0.,1.,self.nx)   
    y = np.linspace(0.,1.,self.ny)   
    t = np.linspace(0.,.1,self.ntsnap)   
 
    x,y = np.meshgrid(x,y) 
    umov = np.zeros((self.ntsnap,self.nx,self.ny))
    for it in range(self.ntsnap):
      umov[it] = self.g(self.c,x,y,t[it])
    return umov
  
\end{verbatim}

Here is the main code for the diffusion problem:
\begin{verbatim}
import numpy as np
import heat2dequation as heq
import matplotlib.pyplot as plt
from mpl_toolkits.axes_grid1 import make_axes_locatable
import matplotlib.cm as cm
from mpl_toolkits.mplot3d import Axes3D
import os

def surf(u1,obj,figname):
  nx = obj.nx
  ny = obj.ny
  X = np.linspace(0.,1.,nx)
  Y = np.linspace(0.,1.,ny)
  X, Y = np.meshgrid(X, Y)
  fig = plt.figure()
  ax = fig.gca(projection='3d')
  surf = ax.plot_surface(X, Y, u1, rstride=10, cstride=10, cmap=cm.coolwarm,
        linewidth=0, antialiased=False)
  ax.set_zlim(-1.01, 1.01)
  fig.savefig(figname)


def plot_side(u1,u2,fn):
  scale = 1. 
  fig = plt.figure(fn)
  ax1  = fig.add_subplot(1,2,1)
  ax1.imshow(u1,vmin=-scale,vmax=scale)
  ax2  = fig.add_subplot(1,2,2)
  ax2.imshow(u2,vmin=-scale,vmax=scale)

def error(u,u1):
  n3 = len(u)
  n2 = len(u[0])
  n1 = len(u[0,0])

  e = (abs(u-u1)).max()
  return e

def g(c,x,y,t):
  return np.sin(np.pi*c*x)*np.sin(np.pi*c*y)*np.exp(-c*c*np.pi*np.pi*t)


#######################################################################
# Here I create the movie frames for diffusion example
N = 101
c = 2.5
outfile1 = 'solution-N%03d-C%3.1f.npy'%(N,c)
outfile2 = 'exact-N%03d-C%3.1f.npy'%(N,c)
modeling = 'heq/'
folder  = 'solutions/'

if not os.path.exists(folder+modeling+outfile1):
  CN2d = heq.CrankNicolson2d(g,N,c=c,jsnap=5)
  u = CN2d.solve()
  uex = CN2d.uexact() 
  np.save(folder+modeling+outfile1,u)
  np.save(folder+modeling+outfile2,uex)
  movieframesdir = folder+modeling+'movieN%03d-C%3.1f'%(N,c)
  os.makedirs(movieframesdir)
  for it in range(len(u)):
    figname = movieframesdir+'/frame-it%02d.png'%it
    figname2 = movieframesdir+'/frameex-it%02d.png'%it
    surf(u[it],CN2d,figname)  
    surf(uex[it],CN2d,figname2)  


nk = 5
p = ['']*nk
L2 = 1.
for k in range(1,nk,1):
  h = 0.02/2**k
  N = int(.1/h)+1
  CN2d = heq.CrankNicolson2d(g,N,c=c,jsnap=5)
  u = CN2d.solve()
  uex = CN2d.uexact() 
  e = error(u,uex)
  r = np.log(L2/e)/np.log(2)
  L2 = e
  print '%d& %4d& %10.8f& %10.8g& %10.8g'%(k,N,h,e,r)

\end{verbatim}

\subsection{Wave code}
The main code is in 'hw04\_weq.py' which calls the plotting an convergence tests routines.
The main class (which is here) is in waveequation.py.

\begin{verbatim}
import numpy as np
from scipy import *

class Waveequation2d:
  def __init__(self,v,M,N,R,c=2.,jsnap=10):
    self.nt = M  # number of grid points
    self.ot = 0.
    self.dt = 1./(M-1)
    self.ntsnap = M/jsnap+1
    self.dtsnap = self.dt*jsnap

    self.ox = 0.
    self.nx = N
    self.dx = 1./(N-1)

    self.ih2 = 1./(self.dx*self.dx)

    self.oy = 0.
    self.ny = self.nx
    self.dy = self.dx

    self.jsnap = jsnap # how many snapshots to avoid for the movie?

    self.c = c
    self.c2 = c*c
    self.v = v
    self.R = R

    self.check_CFL()

    self.um1  = np.zeros((self.nx,self.ny))    
    self.u0   = np.zeros((self.nx,self.ny))    
    self.up1  = np.zeros((self.nx,self.ny)) 
    self.utmp = np.zeros((self.nx,self.ny)) 
    
    self.umovie = np.zeros((self.ntsnap,self.nx,self.ny)) 
    self.get_uo()
    self.get_u1()

  def check_CFL(self):
    if self.c*(self.dt/self.dx)>1:
      print "CFL is violated"

  def solve(self):
    dt2 = self.dt*self.dt

    isnap = 0
    self.umovie[isnap] = self.um1
 
    for it in range(1,self.nt-1,1):
      self.laplacian()
      self.up1 = self.utmp*dt2*self.c2 -self.um1 +2.*self.u0

      # save snapshot to movie
      if ((it+1)%self.jsnap == 0):
        isnap+=1
        self.umovie[isnap] = self.u0
        #print it,self.nt 
      # circulate arrays:
      self.um1 = self.u0
      self.u0 = self.up1 
    return self.umovie

  def get_uo(self):
    x = np.linspace(0.,1.,self.nx)   
    y = np.linspace(0.,1.,self.ny)  

    for ix in range(self.nx):
      for iy in range(self.nx):
        self.um1[ix,iy] = self.v(self.R,x[ix],y[iy])

  def get_u1(self):
    c2 = self.c2
    dt = self.dt
    self.u0 = self.um1 +c2*dt*dt*0.5*self.laplacianu0(self.um1)
  
  def laplacian(self):
    for ix in range(1,self.nx-1,1):
      for iy in range(1,self.nx-1,1):
        self.utmp[ix,iy] = self.u0[ix-1,iy] +self.u0[ix+1,iy]+\
                           self.u0[ix,iy-1] +self.u0[ix,iy+1]-\
                           4.*self.u0[ix,iy]
    self.utmp *= self.ih2

  def laplacianu0(self,u):
    uout = np.zeros((self.nx,self.ny))

    for ix in range(1,self.nx-1,1):
      for iy in range(1,self.nx-1,1):
        uout[ix,iy] = u[ix-1,iy] +u[ix+1,iy]+\
                      u[ix,iy-1] +u[ix,iy+1]-\
                           4.*u[ix,iy]
    uout *= self.ih2
    return uout
\end{verbatim}

Here is the main code for the wave problem:
\begin{verbatim}
import numpy as np
import waveequation as weq
import matplotlib.pyplot as plt
from mpl_toolkits.axes_grid1 import make_axes_locatable
import matplotlib.cm as cm
from mpl_toolkits.mplot3d import Axes3D
import os

def surf(u1,obj,title="",figname=None):
  nx = obj.nx
  ny = obj.ny
  X = np.linspace(0.,1.,nx)
  Y = np.linspace(0.,1.,ny)
  X, Y = np.meshgrid(X, Y)
  fig = plt.figure()
  ax = fig.gca(projection='3d')
  surf = ax.plot_surface(X, Y, u1, rstride=1, cstride=1, cmap=cm.coolwarm,
        linewidth=0, antialiased=False)
  ax.set_zlim(-1.01, 1.01)
  ax.set_title(title)
  if not figname == None:
    fig.savefig(figname)


def plot_side(u1,u2,fn):
  scale = 1. 
  fig = plt.figure(fn)
  ax1  = fig.add_subplot(1,2,1)
  ax1.imshow(u1,vmin=-scale,vmax=scale)
  ax2  = fig.add_subplot(1,2,2)
  im2= ax2.imshow(u2,vmin=-scale,vmax=scale)
  cbar = fig.colorbar(im2)


def error(u,u1):

  e = np.sqrt((abs(u-u1)*abs(u-u1)).sum())/len(u)
  return e


def v(R,x,y):
  cx = 0.5
  cy = 0.5
  
  d = np.sqrt((x-cx)**2+(y-cy)**2)
  if R >= d:
    out = np.cos(0.5*np.pi*d/R)
    out *= out
  else:
    out = 0.
  return out





#######################################################################
# Here I create the movie frames for diffusion example
N = 1601
M = (N-1)/4+1
c = 2.0
jsnap = 5


 
outfile1 = 'solution-N%03d-C%3.1f.npy'%(N,c)
#outfile2 = 'exact-N%03d-C%3.1f.npy'%(N,c)
modeling = 'weq/'
folder  = 'solutions/'

if not os.path.exists(folder+modeling+outfile1):
  weq2d= weq.Waveequation2d(v,N,M,0.3,c=2.,jsnap=jsnap)
  u = weq2d.solve()
  np.save(folder+modeling+outfile1,u)
  movieframesdir = folder+modeling+'movieN%03d-C%3.1f'%(N,c)
  os.makedirs(movieframesdir)
  for it in range(len(u)):
    figname = movieframesdir+'/frame-it%02d.png'%it
    tit = "t=%5.3f"%(it*weq2d.dt*weq2d.jsnap)
    surf(u[it],weq2d,title=tit,figname=figname)  


#
nk = 4
#p = ['']*nk
#L2 = 1.

ufine = np.load(folder+modeling+outfile1)
Nf = len(ufine)
Mf = len(ufine[0])
#
#
uex = ufine[:,Mf/2,Mf/2]

e = np.zeros(nk)
e[1] = 1.
k=0
for n in [25,51,101,201]:
  N = 1601 
  M = n
  Ns = N/jsnap+1
  Nr = (Nf-1)/(Ns-1)
  Mr = (Mf-1)/(M-1)


  weq2d= weq.Waveequation2d(v,N,M,0.3,c=2.,jsnap=jsnap)
  umov = weq2d.solve()
  u = umov[:,M/2,M/2] 
  print u.shape
 
  e[k] = error(u,uex) 
  print e[k]
  k+=1

k=0
for n in [25,51,101,201]:
  N = n 
  M = (N-1)/4+1
  dt = 1./(N-1)
  print n,1./(n-1),e[k],np.log(e[k-1]/e[k])/np.log(2)
  k+=1
\end{verbatim}

\end{document}
