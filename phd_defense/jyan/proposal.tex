\title{\textbf{Thesis Proposal} \\ 
\LARGE{Elastic reverse time migration\\ for anisotropic media}}
\author{\large\textup{ Jia Yan}\\ Advisors: Paul Sava \& Ilya Tsvankin}

%\subtitle{Elastic reverse time migration\\ for anisotropic media}
%\author{Jia Yan}
%\organization{CWP, Colorado School of Mines}




% ------------------------------------------------------------
%% 
 % table of contents
 %%




\tableofcontents
\newpage
%\addcontentsline{toc}{section}{List of figures}
%\listoffigures
%\newpage

%------------------------------------------------------------------
\def\pcscwp{
Center for Wave Phenomena \\ 
Colorado School of Mines \\ 
psava@mines.edu
}

\def\pcscover{
\author[]{Paul Sava}
\institute{\pcscwp}
\date{}
\logo{WSI}
\large
}

\def\WSI{\textbf{WSI}~}

% ------------------------------------------------------------
% colors
\definecolor{darkgreen}{rgb}{0,0.4,0}
\definecolor{LightGray}{rgb}{0.90,0.90,0.90}
\definecolor{DarkGray}{rgb}{0.85,0.85,0.85}

\definecolor{LightGreen} {rgb}{0.792,1.000,0.439}
\definecolor{LightYellow}{rgb}{1.000,0.925,0.545}
\definecolor{LightBlue}  {rgb}{0.690,0.886,1.000}
\definecolor{LightRed}   {rgb}{1.000,0.752,0.796}
\definecolor{Blue} {rgb}{0,0.08,0.45}



\def\red#1{\textcolor{red}{#1}}
\def\blue#1{\textcolor{blue}{#1}}
\def\green#1{\textcolor{green}{#1}}
\def\darkgreen#1{\textcolor{darkgreen}{#1}}
\def\black#1{\textcolor{black}{#1}}
\def\white#1{\textcolor{white}{#1}}
\def\yellow#1{\textcolor{yellow}{#1}}
\def\gray#1{\textcolor{gray}{#1}}
\def\magenta#1{\textcolor{magenta}{#1}}

% ------------------------------------------------------------
% madagascar
\def\mg{\darkgreen{\sc madagascar~}}
\def\mex#1{ \red{ #1 } }
\def\mvbt#1{\small{\blue{\begin{semiverbatim}#1\end{semiverbatim}}}}

% ------------------------------------------------------------
% equations
\def\bea{\begin{eqnarray}}
\def\eea{  \end{eqnarray}}

\def\beq{\begin{equation}}
\def\eeq{  \end{equation}}

% Norm symbol
\newcommand{\norm}[1]{\left|\left|#1\right|\right|}


%\def\req#1{(\ref{#1})}

\def\lp{\left (}
\def\rp{\right)}

\def\lb{\left [}
\def\rb{\right]}

\def\pbox#1{ \fbox {$ \displaystyle #1 $}}

\def\non{\nonumber \\ \nonumber}
\def\lnorm#1{\lVert#1\rVert}
% ------------------------------------------------------------
% REFERENCE (equations and figures)
\def\rEq#1{Equation~\ref{eqn:#1}}
\def\req#1{equation~\ref{eqn:#1}}
\def\rEqs#1{Equations~\ref{eqn:#1}}
\def\reqs#1{equations~\ref{eqn:#1}}
\def\ren#1{\ref{eqn:#1}}

\def\rFg#1{Figure~\ref{fig:#1}}
\def\rfg#1{Figure~\ref{fig:#1}}
\def\rFgs#1{Figures~\ref{fig:#1}}
\def\rfgs#1{Figures~\ref{fig:#1}}
\def\rfn#1{\ref{fig:#1}}

% ------------------------------------------------------------
% field operators

% trace
\def\tr{\texttt{tr}\;}

% divergence
\def\DIV#1{\nabla \cdot {#1}}

% curl
\def\CURL#1{\nabla \times {#1}}

% gradient
\def\GRAD#1{\nabla {#1}}

% Laplacian
\def\LAPL#1{\nabla^2 {#1}}

\def\dellin{
\lb
\begin{matrix}
\done{}{x} \; \done{}{y} \; \done{}{z}
\end{matrix}
\rb
}

\def\delcol{
\lb
\begin{matrix}
\done{}{x} \non
\done{}{y} \non
\done{}{z}
\end{matrix}
\rb
}

\def\aveclin{
\lb
\begin{matrix}
a_x \; a_y \; a_z
\end{matrix}
\rb
}


% ------------------------------------------------------------

% elastic tensor
\def\CC{{\bf C}}

% identity tensor
\def\I{\;{\bf I}}

% particle displacement vector
\def\uu{{\bf u}}

% particle velocity vector
\def\vv{{\bf v}}

% particle acceleration vector
\def\aa{{\bf a}}

% force vector
\def\ff{{\bf f}}

% wavenumber vector
\def\kk{{\bf k}}

% ray parameter vector
\def\pp{{\bf p}}

% distance vector
\def\hh{ {\boldsymbol{\lambda}} }
\def\xx{{\bf x}}
\def\kkx{{\kk_\xx}}
\def\ppx{{\pp_\xx}}

\def\yy{{\bf y}}

% normal vector
\def\nn{{\bf n}}
\def\ns{\nn_s}
\def\nr{\nn_r}

% source vector
\def\ss{{\bf s}}
\def\kks{{\kk_\ss}}
\def\pps{{\pp_\ss}}

% receiver vector
\def\rr{{\bf r}}
\def\kkr{{\kk_\rr}}
\def\ppr{{\pp_\rr}}

% midpoint vector
\def\mm{{\bf m}}
\def\kkm{{\kk_\mm}}
\def\ppm{{\pp_\mm}}

% offset vector
\def\ho{{\bf h}}
\def\kkh{{\kk_\ho}}
\def\pph{{\pp_\ho}}

% space-lag vector

\def\kkl{{\kk_\hh}}
\def\ppl{{\pp_\hh}}

% CIP vector
\def\cc{ {\bf c}}

% time-lag scalar
\def\tt{\tau}
\def\tts{\tt_s}
\def\ttr{\tt_r}

% frequency
\def\ww{\omega}

%
\def\dd{{\bf d}}

\def\bb{{\bf b}}
\def\qq{{\bf q}}

\def\ii{{\bf i}} % unit vector
\def\jj{{\bf j}} % unit vector

\def\lo{{\bf l}}

% ------------------------------------------------------------

\def\Fop#1{\mathcal{F}     \lb #1 \rb}
\def\Fin#1{\mathcal{F}^{-1}\lb #1 \rb}


% ------------------------------------------------------------
% wave equation operators:

\def\swe#1{ \frac{1}{v^2(\xx)}\dtwo{#1}{t}- \LAPL{#1}}
\def\awe#1{ \frac{1}{v_p^2(\xx) \rho(\xx)}\dtwo{#1}{t}- \DIV{\frac{1}{\rho(\xx)}\GRAD{#1}} }
\def\ewe#1{ \rho(\xx) \dtwo{#1}{t} -v_p(\xx)^2\LAPL{#1}+v_s(\xx)^2\CURL{\CURL{#1}}}

% ------------------------------------------------------------
% partial derivatives

\def\dtwo#1#2{\frac{\partial^2 #1}{\partial #2^2}}
\def\done#1#2{\frac{\partial   #1}{\partial #2  }}
\def\dthr#1#2{\frac{\partial^3 #1}{\partial #2^3}}
\def\mtwo#1#2#3{ \frac{\partial^2#1}{\partial #2 \partial#3} }

\def\larrow#1{\stackrel{#1}{\longleftarrow}}
\def\rarrow#1{\stackrel{#1}{\longrightarrow}}

% ------------------------------------------------------------
% elasticity 

\def\stress{\underline{\textbf{t}}}
\def\strain{\underline{\textbf{e}}}
\def\stiffness{\underline{\underline{\textbf{c}}}}
\def\compliance{\underline{\underline{\textbf{s}}}}

\def\GEOMlaw{
\strain = \frac{1}{2} 
\lb \GRAD{\uu} + \lp \GRAD{\uu} \rp^T \rb
}

\def\HOOKElaw{
\stress = \lambda \; tr \lp \strain \rp {\bf I} + 2 \mu \strain 
}

\def\CONSTITUTIVElaw{
\stress = \stiffness \;\strain 
}


\def\NEWTONlaw{
\rho \ddot{\uu} = \DIV{\stress}
}

\def\NAVIEReq{
\rho \ddot\uu =
\lp \lambda + 2\mu \rp \GRAD{\lp \DIV{\uu} \rp}
             - \mu     \CURL{   \CURL{\uu}}
}

% ------------------------------------------------------------

% potentials
\def\VP{\boldsymbol{\psi}}
\def\SP{\theta}

% stress tensor
\def\ssten{{\bf \sigma}}

\def\ssmat{
\lp \matrix {
 \sigma_{11} &  \sigma_{12}   &  \sigma_{13} \cr
 \sigma_{12} &  \sigma_{22}   &  \sigma_{23} \cr
 \sigma_{13} &  \sigma_{23}   &  \sigma_{33} \cr
} \rp
}

% strain tensor
\def\eeten{{\bf \epsilon}}

\def\eemat{
\lp \matrix {
 \epsilon_{11} &  \epsilon_{12}   &  \epsilon_{13} \cr
 \epsilon_{12} &  \epsilon_{22}   &  \epsilon_{23} \cr
 \epsilon_{13} &  \epsilon_{23}   &  \epsilon_{33} \cr
} \rp
}


% plane wave kernel
\def\pwker{A e^{i k \lp \nn \cdot \xx - v t \rp}}


% ------------------------------------------------------------
% details for expert audience (math, cartoons)
\def\expert{
\colorbox{red}{\textbf{\LARGE \white{!}}}
}

% ------------------------------------------------------------
% image, data, wavefields

\def\RR{R}

\def\R{R(\xx)}
\def\Re{R(\xx,\hh,\tau)}
\def\Rc#1{R^{#1}(\xx)}
\def\Rce#1{R^{#1}(\xx,\hh,\tau)}
\def\Rct#1{R^{#1}(z,\tau)}
\def\Rcl#1{R^{#1}(z,\lambda_x)}

\def\UU{u}
\def\US{{\UU_s}}
\def\USt{{\UU_s^f}}
\def\USr{{\UU_s^b}}
\def\UR{{\UU_r}}
\def\URt{{\UU_r^f}}
\def\URr{{\UU_r^b}}

\def\DD{D}
\def\DS{{\DD_s}}
\def\DR{{\DD_r}}

\def\UUw{\UU}
\def\USw{{\UU_s}}
\def\URw{{\UU_r}}

\def\DDw{\DD}
\def\DSw{{\DD_s}}
\def\DRw{{\DD_r}}

% perturbations

\def\ds{\Delta s}
\def\di{\Delta \RR}
\def\du{\Delta \UU}

\def\dRR{\Delta \RR}
\def\dUU{\Delta \UU}
\def\dUS{\Delta \US}
\def\dUR{\Delta \UR}

\def\dtt{\Delta \tt}
\def\dhh{\Delta \hh}

% ------------------------------------------------------------
% Green's functions

\def\GG{G}

\def\GS{{\GG_s}}
\def\GR{{\GG_r}}

% ------------------------------------------------------------
% elastic data, wavefields

\def\eRR{\textbf{\RR}}

\def\eDS{{\textbf{\DD}_s}}
\def\eDR{{\textbf{\DD}_r}}
\def\eDD{{\textbf{\DD}}}

\def\eUS{{\textbf{\UU}_s}}
\def\eUR{{\textbf{\UU}_r}}
\def\eUU{{\textbf{\UU}}}

% ------------------------------------------------------------
% sliding bar
\def\tline#1{
\put(95,-3){\small \blue{time}}
\put(-4,-1){\small \blue{0}}
\thicklines
\put( 0,0){\color{blue} \vector(1,0){100}}
\put(#1,0){\color{red}  \circle*{2}}
}

% ------------------------------------------------------------
% arrow on figure
\def\myarrow#1#2#3{
\thicklines
\put(#1,#2){\color{green} \vector(-1,-1){5}}
\put(#1,#2){\color{green} \textbf{#3}}
}

\def\bkarrow#1#2#3{
\thicklines
\put(#1,#2){\color{black} \vector(-1,-1){5}}
\put(#1,#2){\color{black} \textbf{#3}}
}


\def\anarrow#1#2#3#4{
\thicklines
\put(#1,#2){\color{#4} \vector(-1,-1){5}}
\put(#1,#2){\color{#4} \textbf{#3}}
}

% ------------------------------------------------------------
% circle on figure
\def\mycircle#1#2#3{
\thicklines
\put(#1,#2){\color{green} \circle{#3}}
}

% ------------------------------------------------------------
% note on figure
\def\mynote#1#2#3{
\put(#1,#2){\color{green} \textbf{#3}}
}

\def\biglabel#1#2#3{
\put(#1,#2){\Huge \textbf{#3}}
}

\def \largelabel#1#2#3{
\put(#1,#2){\Large \textbf{#3}}
}


\def \normallabel#1#2#3{
\put(#1,#2){\normalsize \textbf{#3}}
}




\def\waxes{\put(-0,0){\color{white}\vector(1,0){10}}
\color{white} \klabellarge{10.5}{-1}{\color{white} {x}}
\put(-0,0){\color{white} \vector(0,-1){10}}
\color{white} \klabellarge{-1.2}{-14.5}{\color{white} {z}}}


\def\axes{\put(-0,0){\vector(1,0){10}}
\klabellarge{10.5}{-1}{\bf{x}}
\put(-0,0){\vector(0,-1){10}}
\klabellarge{-1.2}{-14.5}{\bf{z}}}



\def\putaxes#1#2{\put(#1,#2){\axes}}
\def\putwaxes#1#2{\put(#1,#2){\waxes}}

% huge labels
\def\wlabel#1#2#3{ \white{ \biglabel{#1}{#2}{#3} }}
\def\klabel#1#2#3{ \black{ \biglabel{#1}{#2}{#3} }}
\def\rlabel#1#2#3{ \red{   \biglabel{#1}{#2}{#3} }}
\def\glabel#1#2#3{ \green{ \biglabel{#1}{#2}{#3} }}
\def\blabel#1#2#3{ \blue { \biglabel{#1}{#2}{#3} }}
\def\ylabel#1#2#3{ \yellow{\biglabel{#1}{#2}{#3} }}


% large labels
\def\klabellarge#1#2#3{ \black{ \largelabel{#1}{#2}{#3} }}
\def\wlabellarge#1#2#3{ \white{ \largelabel{#1}{#2}{#3} }}

% normal labels
\def\klabelnormal#1#2#3{ \black{ \normallabel{#1}{#2}{#3} }}
\def\wlabelnormal#1#2#3{ \white{ \normallabel{#1}{#2}{#3} }}

% ------------------------------------------------------------
% centering
\def\cen#1{ \begin{center} \textbf{#1} \end{center}}
\def\cit#1{ \begin{center} \textit{#1} \end{center}}

% emphasis (bold+alert)
\def\bld#1{ \textbf{\alert{#1}}}

% huge fonts
\def\big#1{\begin{center} {\LARGE \textbf{#1}} \end{center}}
\def\hug#1{\begin{center} {\Huge  \textbf{#1}} \end{center}}

% ------------------------------------------------------------
% separator
\def\sep{ \vfill \hrule \vfill}
\def\itab{ \hspace{0.5in}}
\def\nsp{\\ \vspace{0.1in}}

% ------------------------------------------------------------
% integrals

\def\tint#1{\!\!\!\int\!\! #1 dt}
\def\xint#1{\!\!\!\int\!\! #1 d\xx}
\def\wint#1{\!\!\!\int\!\! #1 d\ww}
\def\aint#1{\!\!\!\alert{\int}\!\! #1 d\alert{\xx}}

\def\esum#1{\sum\limits_{#1}}
\def\eint#1{\int\limits_{#1}}

% ------------------------------------------------------------
\def\CONJ#1{\overline{#1}}
\def\MOD#1{\left| {#1} \right|}

% ------------------------------------------------------------
% imaging components

\def\IC{\colorbox{yellow}{\textbf{I.C.}}\;}
\def\WR{\colorbox{yellow}{\textbf{W.R.}}\;}
\def\WE{\colorbox{yellow}{\textbf{W.E.}}\;}
\def\SO{\colorbox{yellow}{\textbf{SOURCE}}\;}
\def\WS{\colorbox{yellow}{\textbf{W.S.}}\;}

% ------------------------------------------------------------
% summary/take home message
\def\thm{take home message}

% ------------------------------------------------------------
\def\dx{\Delta x}
\def\dy{\Delta y}
\def\dz{\Delta z}
\def\dt{\Delta t}

\def\dhx{\Delta h_x}
\def\dhy{\Delta h_y}

\def\kz{{k_z}}
\def\kx{{k_x}}
\def\ky{{k_y}}

\def\kmx{k_{m_x}}
\def\kmy{k_{m_y}}
\def\khx{k_{h_x}}
\def\khy{k_{h_y}}

\def\why{ \alert{\widehat{{\khy}}}}
\def\whx{ \alert{\widehat{{\khx}}}}

\def\lx{{\lambda_x}}
\def\ly{{\lambda_y}}
\def\lz{{\lambda_z}}

\def\klx{k_{\lambda_x}}
\def\kly{k_{\lambda_y}}
\def\klz{k_{\lambda_z}}

\def\mx{{m_x}}
\def\my{{m_y}}
\def\mz{{m_z}}
\def\hx{{h_x}}
\def\hy{{h_y}}
\def\hz{{h_z}}

\def\sx{{s_x}}
\def\sy{{s_y}}
\def\rx{{r_x}}
\def\ry{{r_y}}

% ray parameter (absolute value)
\def\modp#1{\left| \pp_{#1} \right|}

% wavenumber
\def\modk#1{\left| \kk_{#1} \right|}

% ------------------------------------------------------------
\def\kzwk{ {\kz^{\kk}}}
\def\kzwx{ {\kz^{\xx}}}
 
\def\PSk#1{e^{\red{#1 i \kzwk \dz}}}
\def\PSx#1{e^{\red{#1 i \kzwx \dz}}}
\def\PS#1{ e^{\red{#1 i k_z   \dz}}}

\def\TT{t}
\def\TS{t_s}
\def\TR{t_r}

\def\oft{\lp t \rp}
\def\ofw{\lp \ww \rp}

\def\ofx{\lp \xx \rp}
\def\ofk{\lp \kk \rp}
\def\ofs{\lp \ss \rp}
\def\ofr{\lp \rr \rp}
\def\ofz{\lp   z \rp}

\def\ofxt{\lp \xx, t  \rp}
\def\ofst{\lp \ss, t  \rp}
\def\ofrt{\lp \rr, t  \rp}

\def\ofxw{\lp \xx, \ww  \rp}
\def\ofsw{\lp \ss, \ww  \rp}
\def\ofrw{\lp \rr, \ww  \rp}

\def\ofxm{\lp \xx,\hh \rp}

\def\ofxmp{\lp \xx+\hh \rp}
\def\ofxmm{\lp \xx-\hh \rp}

\def\ofmm{\lp \mm      \rp}
\def\ofmz{\lp \mm, z   \rp}
\def\ofmw{\lp \mm, \ww \rp}
\def\ofkm{\lp \kkm     \rp}

% ------------------------------------------------------------
% source/receiver data and wavefields

\def\dst{$\DS\ofst$}
\def\drt{$\DR\ofrt$}
\def\ust{$\US\ofxt$}
\def\urt{$\UR\ofxt$}

\def\dsw{$\DS\ofsw$}
\def\drw{$\DR\ofrw$}
\def\usw{$\US\ofxw$}
\def\urw{$\UR\ofxw$}

% ------------------------------------------------------------
\def\Nx{N_x}
\def\Ny{N_y}
\def\Nz{N_z}
\def\Nt{N_t}
\def\Nw{N_{\ww}}
\def\Nm{N_{\mm}}

\def\Nlx{N_{\lambda_x}}
\def\Nly{N_{\lambda_y}}
\def\Nlz{N_{\lambda_z}}
\def\Nlt{N_{\tau}}

\def\wmin{\ww_{min}}
\def\wmax{\ww_{max}}
\def\zmin{z_{min}}
\def\zmax{z_{max}}
\def\tmin{t_{min}}
\def\tmax{t_{max}}
\def\lmin{\hh_{min}}
\def\lmax{\hh_{max}}
\def\xmin{\xx_{min}}
\def\xmax{\xx_{max}}

% ------------------------------------------------------------
% course qualifiers

\def\fun{\hfill \alert{concepts}}
\def\pra{\hfill \alert{applications}}
\def\fro{\hfill \alert{frontiers}}


% ------------------------------------------------------------
% wavefield extrapolation
\def\ws{ {\ww s} }

\def\kows{\lp \frac{\kx}{\ws} \rp}

\def\kmws{\lp \frac{\modk{\mm}}{\ws} \rp}
\def\kzws{\lp \frac{\kz}       {\ws} \rp}

\def\S{\lb\frac{\modk{\mm}}{\ws  }\rb}
\def\C{\lb\frac{\modk{\mm}}{\ws_0}\rb}
\def\K{\lb\frac{\modk{\mm}}{\ww  }\rb}

\def\Cs{\lb\frac{\modk{\mm}^2}{\lp \ws_0 \rp^2}\rb}

\def\SSR#1{  \sqrt{ \lp \ww {#1} \rp^2 - \modk{\mm}^2} }

\def\SQRsum#1{\sum\limits_{n=1}^{\infty} \lp -1 \rp^n
		\displaystyle{\frac{1}{2} \choose n} #1}

\def\TSE#1#2#3#4{\sum\limits_{#4=#3}^{\infty} \lp -1 \rp^#4
		\displaystyle{#2 \choose #4} {#1}^#4}

\def\onefrac#1#2{\frac{#2^2}{a_#1+b_#1 #2^2}}
\def\SQRfrac#1{
	\sum\limits_{n=1}^{\infty}
	\onefrac{n}{#1} }

\def\dkzds { \left. \frac{d {\kz}}  {d s} \right|_{s_b} }
\def\SSX#1#2{\sqrt{ 1 - \lb \frac{\MOD{#2}}{#1} \rb^2} }
\def\SST#1#2{1 + \sum_{j=1}^N c_j \lb \frac{\MOD{#2}}{#1} \rb^{2j} }

% ------------------------------------------------------------
% acknowledgment
\def\ackcwp{\cen{the sponsors of the\\Center for Wave Phenomena\\at\\Colorado School of Mines}}

% ------------------------------------------------------------
% citation in slides
\def\talkcite#1{{\small \sc #1}}

% ------------------------------------------------------------
\def\ise{GPGN302: Introduction to Electromagnetic and Seismic Exploration}
\def\inv{GPGN409: Inversion}

% ------------------------------------------------------------
\def\model{m}
\def\data {d}

\def\Lop{ {\mathbf{L}}}
\def\Sop{ {\mathbf{S}}}
\def\Eop{ {\mathbf{E}}}
\def\Iop{ {\mathbf{I}}}
\def\Aop{ {\mathbf{A}}}
\def\Pop{ {\mathbf{P}}}
\def\Fop{ {\mathbf{F}}}


% ------------------------------------------------------------
\def\mybox#1{
  \begin{center}
    \fcolorbox{black}{yellow}
    {\begin{minipage}{0.8\columnwidth} {#1} \end{minipage}}
  \end{center}
}

\def\hibox#1{
  \begin{center}
    \fcolorbox{black}{LightGreen}
    {\begin{minipage}{0.8\columnwidth} {#1} \end{minipage}}
  \end{center}
}

% ------------------------------------------------------------
% Nota Bene
\def\nbnote#1{
  \vfill
  \begin{center}
    \colorbox{LightGray}
    {\begin{minipage}{\columnwidth} {\textbf{\black{\large N.B.}} #1} \end{minipage}}
  \end{center}
}

\def\highlight#1{
  \begin{center}
    \colorbox{cyan}
    {\begin{minipage}{\columnwidth} {#1} \end{minipage}}
  \end{center}
}

% ------------------------------------------------------------
\def\pcsshaded#1{
  \definecolor{shadecolor}{rgb}{0.8,0.8,0.8}
  \begin{shaded} {#1} \end{shaded}
  \definecolor{shadecolor}{rgb}{1.0,1.0,1.0}
}

% ------------------------------------------------------------
\def\postit#1{
  \begin{center}
    \colorbox{yellow}
    {\begin{minipage}{0.66\columnwidth} {#1} \end{minipage}} 
  \end{center}
}

% ------------------------------------------------------------
\def\graybox#1{
  \begin{center}
    \colorbox{LightGray}
    {\begin{minipage}{1.00\columnwidth} {#1} \end{minipage}}
  \end{center}
}

\def\whitebox#1{
  \begin{center}
    \colorbox{white}
    {\begin{minipage}{1.00\columnwidth} {#1} \end{minipage}}
  \end{center}
}

\def\yellowbox#1{
  \begin{center}
    \colorbox{LightYellow}
    {\begin{minipage}{1.00\columnwidth} {#1} \end{minipage}}
  \end{center}
}

\def\greenbox#1{
  \begin{center}
    \colorbox{LightGreen}
    {\begin{minipage}{1.00\columnwidth} {#1} \end{minipage}}
  \end{center}
}

\def\bluebox#1{
  \begin{center}
    \colorbox{LightBlue}
    {\begin{minipage}{1.00\columnwidth} {#1} \end{minipage}}
  \end{center}
}

\def\redbox#1{
  \begin{center}
    \colorbox{LightRed}
    {\begin{minipage}{1.00\columnwidth} {#1} \end{minipage}}
  \end{center}
}

\def\hyellow#1{ \colorbox{yellow} #1 }
\def\hgreen #1{ \colorbox{green}  #1 }

% ------------------------------------------------------------
% boxes for vectors and matrices

\def\pcsbox#1#2#3#4{
  % #1 = width
  % #2 = height
  % %3 = hmax
  \begin{picture}(#3,#1)
    \linethickness{0.5mm}
    % 
    \multiput(0,#1)(#3, 0){2}{\line(0,-1){#2}}
    \multiput(0,#1)(0,-#2){2}{\line(+1,0){#3}}
    % 
    \put(1,3){#4}
  \end{picture}
}

\def\pcssym#1#2{
  \begin{picture}(1,#1)
    \put(0,3){#2}
  \end{picture}
}

\def\sidebyside#1#2{
  \begin{center}
    \colorbox{LightBlue}{
      \begin{minipage}{1.0\columnwidth} {#1} \end{minipage}
    }
    \colorbox{LightYellow}{
      \begin{minipage}{1.0\columnwidth} {#2} \end{minipage}
    }
  \end{center}
}

\def\sidebysidebyside#1#2#3{
  \begin{center}
    \colorbox{LightBlue}{
      \begin{minipage}{1.0\columnwidth} {#1} \end{minipage}
    }
    \colorbox{LightYellow}{
      \begin{minipage}{1.0\columnwidth} {#2} \end{minipage}
    }
    \colorbox{LightRed}{
      \begin{minipage}{1.0\columnwidth} {#3} \end{minipage}
    }
  \end{center}
}


\def\CURL#1{\nabla \times {#1}}
\def\GRAD#1{\nabla {#1}}
\def\DIV#1{\nabla \cdot {#1}}
\def\LAPL#1{\nabla^2 {#1}}

\def\uu{\mathbf u}

\def\xx{ x,z}
\def\ofx { \lp \xx   \rp}
\def\ofxt{ \lp \xx,t \rp}

\def\IM#1{  {I}_{#1} \ofx }
\def\US#1{{u_s}_{#1} \ofxt}
\def\UR#1{{u_r}_{#1} \ofxt}

\def\P#1{ P_{#1} \ofxt }
\def\S#1{ S_{#1} \ofxt }
%
%\maketitle


\addcontentsline{toc}{section}{Abstract}

\section{Abstract}
Multicomponent data are not usually processed with specifically designed procedures, but are processed with procedures analogous to the ones used for single-component data. The common industry practice is to take vertical and horizontal components of the data as P and S, and use them to perform acoustic wave equation migration, respectively. This may introduce errors and noises in the migrated images. 

I work on elastic reverse time migration, where vector wavefields are reconstructed and vector imaging conditions are applied. The vector wavefields are reconstructed in reversed time using vector data as sources. The vector imaging condition for vector wavefields can take displacements or potentials, while potential imaging condition leads to images easier to interpret. 
Extended imaging conditions are applied to reconstructed vector wavefields, and the decomposed angle gathers can be used for migration velocity studies for elastic waves, and amplitude versus angle study. 

Future work on anisotropic elastic reverse time migration includes anisotropic wavefield modeling, anisotropic vector imaging condition, anisotropic angle domain common image gathers and field data applications with both isotropic and anisotropic implementations. 


\newpage
\addcontentsline{toc}{section}{Summary of research  goals}
\section{Summary of research  goals}
%  - what is the problem?
%  - why is it important?
%  - what is known about the problem?
%  - what did I do about this until now?
%  - what will I do in the future?
%  - what is the anticipated impact?

% What is the problem
% 1. earth works as solid, but in current insdustry common practice, treated as liquid, use acoustic wave equation to model.
% 2. people realize 1, and want to ..., they record multicomponent data
% 3. these multicomponent data are not used properly
% 4. what was wrong? In many cases, even multicomponent data is acquired, it is not used. Or take vertical as P and horizontal as S. This waste useful information or introduce noise in the image.

Seismic processing is mostly based on acoustic wave equation, which assumes the Earth works as a liquid that propagates compressive waves only. Although useful in practice, this assumption is not theoretically valid, since for the sources that are used in seismic exploration, shear waves are either generated or converted from compressive waves, and are recorded by receivers. For complex geological structures, shear waves can be quite significant. However, in the past, only compressive wave processing is routinely implemented, and most of the time, shear waves are ignored. Recently, in order to image complex geology, to understand petrophysical properties, and to obtain true amplitude images, more and more multicomponent data (both ocean bottom and land) are acquired and used. These multicomponent data, however, are not usually processed with specifically designed procedures, instead, most are processed with ad-hoc procedures borrowed from acoustic wave equation algorithms. Most of the time, in 2D cases, people take the recorded vertical component as P wave, and horizontal component as S wave, and this will certainly introduce errors and noise in the image, as P and S components are mixed on all recorded components. Also, since taking vertical and horizontal components as P and S ignores the P and S in the other orientation, the images are not possible to be true-amplitude images.



% Why do multicomponent imaging this important?
% 
Multicomponent seismic data have the potential to better characterize the subsurface. Multicomponent images can be used to validate bright spot reflections, provide Poisson's ratio estimates, provide reflection images where the P-wave reflectivity is small, image through gas clouds where the P-wave signal is attenuated and detect fractures through shear-wave splitting. %~\cite[]{simmons:1227}. 
Converted wave images have higher resolution than acoustic images, because S waves have shorter wavelength than P waves. For a single shot, converted wave images have better illumination than acoustic images. In short, multicomponent images provide more complete information about the subsurface.
Anisotropy is also a very important feature of the earth that could not be ignored in many areas. The ignorance of anisotropy causes mispositioning of the geologic structure and fuzzy images.%rephrase

% What is known about the problem?
% People do multicomponent Kirchhoff migration, which still suffer from complex area where ray theory breaks down.
% People did reverse time migration using multicomponent data, mostly using vertical as P and horizontal as S
% Almost nothing was done on anisotropic reverse time migration

Multicomponent imaging has long been an interesting and active research area for exploration geophysicists. Quite a few authors proposed Kirchhoff migration and reverse time migration techniques for multicomponent data. The Kirchhoff migration technique itself suffers from areas with complex geology where ray theory breaks down. The published elastic reverse time migration techniques either takes vertical and horizontal data as P and S component respectively, or need to do some separation of wave modes on the recording surface, both of which are followed by acoustic wave equation migration. Other elastic reverse time migration technique do not separate wave modes on the surface, but use some imaging condition that are not general enough for all complex geology. Most multicomponent imaging algorithms are still designed for isotropic medium only, and almost nothing was done on anisotropic reverse time migration.


% What did I do about this until now?
% Elastic isotropic reverse time migration
% Imaging condition on displacement and potentials
% Angle gathers on elastic reverse time migration

So far, I have done research work on isotropic elastic reverse time migration. In my study, I reconstruct vector wavefields with elastic wave equation, and developed imaging conditions for vector displacement and potentials. The potentials are separated from the reconstructed displacement with Helmholtz decomposition, in the subsurface (not on the recording surface). Images obtained from potential imaging condition have clearer physical meaning than displacement imaging condition, and are easier to interpret. Using vector potentials and extended imaging conditions, I was able to do angle decompositions and obtain flattened angle gathers for both pure mode and converted waves. These angle gathers are powerful tools to do wave equation migration velocity analysis (WEMVA) for both P and S waves. 

% what will I do in the future?
Based on the completed research, I propose to work on anisotropic elastic reverse time migration, which will include anisotropic wavefield modeling, anisotropic vector imaging condition, anisotropic angle domain common image gathers and field data applications with both isotropic and anisotropic implementations. 
This study will not only allow us to obtain elastic images, but also can be used to do migration velocity analysis, anisotropy parameter estimations, etc. Ultimately, the elastic images can be used to study petrophysical properties and do amplitude versus angle (AVA) analysis.







\addcontentsline{toc}{section}{Background and previous research}
\section{Background and previous research}
\addcontentsline{toc}{subsection}{Reverse time migration}
\subsection{Reverse time migration}

Reverse time migration was first introduced by \cite{SEG-1983-S10.1} and \cite{GEO48-11-15141524} to post-stack or zero-offset data. The procedure of post-stack reverse time migration is: first, reverse the recorded data in time; second, use this reversed data as multiple sources along the recording surface, and inject these sources and propagate the wavefields; finally, apply the imaging condition that the image is formed at the time zero. The principle of reverse time migration is that the reflectors of interest work as exploding reflectors and that the wave equation used to propagate data can be driven either forward or backward in time by simply reversing the time axis~\cite[]{levin:581}.
The advantages of reverse time migration are that the extrapolation in \emph{time} avoids evanescent energy and that there is no dip limitations of the structure. 

Reverse time migration was then introduced by \cite{chang:67} to pre-stack data. Pre-stack reverse time migration reconstructs wavefield the same as post-stack reverse time migration. However, for pre-stack migration, both source wavefield and receiver wavefield need to be constructed, and a different imaging condition than post-stack imaging condition needs to be applied. \cite{chang:67} applied the so called  excitation-time imaging condition, where images is formed at the time that the source ray takes to travel to this image position. This imaging condition is a special case of cross-correlation imaging condition of \cite{Claerbout.iei}.
 
\cite{chang:513} developed 3-D reverse time migration algorithm for pre-stack data, following the idea of \cite{chang:67}. This algorithm reconstructs source wavefield and receiver wavefield with 3-D acoustic wave equation, and applied the same excitation-time imaging condition.

Reverse time migration has shown by many authors, such as \cite{jones:2140} and \cite{boechat:2427}, its advantages in complex geologic structures. Recently because of the availability of more computation resources, the reverse time migration algorithms that are previously too expensive to implement, are now used by industry to image areas that are difficult to image with other cheaper solutions.


\addcontentsline{toc}{subsection}{Elastic imaging}
\subsection{Elastic imaging}

Multicomponent elastic data are commonly recorded in land or marine (ocean-bottom) seismic experiments. Vector elastic wavefields are not usually processed by specifically designed imaging procedures, but they are processed by extensions of techniques used for scalar wavefields. Thus, seismic data processing does not take full advantage of the information contained by elastic wavefields, e.g. for illumination of complex geology, amplitude preserving imaging, estimation of model parameters, etc.

Elastic wave propagation in isotropic medium follows isotropic elastic wave equations:
\def\u{u_x}
\def\w{u_z}
\def\dux{\frac{\partial \u}{\partial x } }
\def\duz{\frac{\partial \u}{\partial z } }
\def\dwx{\frac{\partial \w}{\partial x } }
\def\dwz{\frac{\partial \w}{\partial z } }
\beqa 
\label{eqn:EWE}
\rho\dtwo{\w}{t} & = &\done{}{z} \lb \lambda\lp\dux+\dwz\rp +2\mu\dwz \rb + \done{}{x} \lb \mu\lp \dwx+\duz  \rp  \rb  \; ,\\
\rho\dtwo{\u}{t} & = &\done{}{x} \lb \lambda\lp\dux+\dwz\rp +2\mu\dux \rb + \done{}{z} \lb \mu\lp \dwx+\duz  \rp  \rb  \; .
\eeqa
where $u_z$ and $u_x$ are the vertical and horizontal displacements; $\rho$ is the density; $\lambda$ and $\mu$ are Lam\'e constants. This wave equation assumes spatially slowly varying density.

The coupling of vertical and horizontal displacements makes it difficult to obtain a dispersion relation to extrapolate wavefields in time, which makes it natural to process elastic wavefields in time. And therefore, there are primarily two options when it comes to elastic imaging that works in time domain: one is Kirchhoff migration, and the other one is reverse time migration. 

Kirchhoff migration is based on diffraction summation, which sums the amplitudes along the diffraction curves in $x-t$ plane and then maps onto $x-z$ plane.
\cite{GEO49-08-12231238} discussed Kirchhoff migration for shot-record elastic data, where PP and PS reflections can be migrated by downward continuate source traveltime and receiver travel time using both P and S wave velocities. 
\cite{hokstad:861} did multicomponent anisotropic Kirchhoff migration for multi-shot, multi-receiver experiments, where pure mode and converted mode images are obtained by downward continuing visco-elastic vector wavefields and apply survey-sinking imaging condition.

Elastic reverse time migration have several kinds. \cite{GEO63-05-16851695} did elastic reverse time with Lam\'e potential methods, where Lam\'e parameters are inverted and the pre-conditioned gradients of $\lambda$ and $\rho$ (density) are shown as elastic images PP and PS. Other elastic imaging methods mimics acoustic imaging where source wave fields and receiver wavefields are reconstructed, and apply some kind of imaging condition. Because PP and PS reflections are mixed in recordings, one need to separate P and S waves. There are two potential approaches to separate wavefields and image elastic seismic wavefields.
\begin{itemize}
\item
The first option is to separate P and S modes on the acquisition
surface from the recorded elastic wavefields. This procedure involves
either approximations about the propagation path and direction of
polarization of the recorded data, or it involves reconstruction of
the seismic wavefields in the vicinity of the acquisition surface by a
numerical solution of the elastic wave-equation, followed by wavefield
decomposition in P and S potentials using div ($\DIV{}$) and curl
($\CURL{}$) operators \cite[]{SEG-1988-S12.4,GEO62-02-05980613}.

An alternative data decomposition in P and S potentials is to
reconstructed wavefields in the subsurface using the elastic
wave-equation, then decompose the wavefields in P and S potential and
finally forward extrapolate the separated wavefields back to the
surface using the acoustic wave-equation with the appropriate
propagation velocity for the P or S wave modes
\cite[]{sun.elastic.rtm}. Once data are decomposed in P and S
potentials, imaging can be done using conventional procedures employed
for scalar wavefields for the separated P and S modes.

\item
The second option is to extrapolate wavefields in the subsurface using
a numeric solution to the elastic wave-equation and then apply an
imaging condition that extracts reflectivity information from the
source and receiver wavefields. In case extrapolation is implemented
by finite-difference methods
\cite[]{chang:67,chang:597}, this procedure is known
as elastic reverse time migration, and is conceptually similar to
acoustic reverse-time migration \cite[]{MSD00-00-04520462}, which is
more frequently used in seismic imaging.
\end{itemize}

Different imaging conditions have been proposed for extracting
reflectivity information from reconstructed elastic wavefields. One possibility is
the excitation-time imaging condition proposed by
\cite{chang:67}.  This imaging condition extracts from extrapolated wavefields reflectivity information at times computed by
ray tracing from the source.
\cite{SEG-1991-1009} applied a different imaging condition, where imaging time is computed by forward propagating wavefields using P-SV elastodynamic wave equation, finding that excitation time imaging condition may lead to wrong images in regions where the ray theory fails. These imaging conditions represent special cases of a more general type of imaging condition involving time cross-correlations of source and receiver wavefields at every location in the subsurface.


\addcontentsline{toc}{subsection}{Anisotropy in imaging}
\subsection{Anisotropy in imaging}
%\cite{hokstad:2044}

It is well known that ignoring anisotropy in imaging may cause an inaccurate depth scale of the image, mispositioning of dipping features, and (in prestack migration) poor focusing of dipping and sometimes, even horizontal reflectors \cite[]{tsvankin.2001}. These migration errors have been studied by \cite{larner:1454}, \cite{alkhalifah:1405} and many others. \rFg{AniImg} shows the migration errors caused by an isotropic migration algorithm to an isotropic model. Migration algorithms can be generalized to anisotropic media using either the appropriate phase-velocity equations or an anisotropic ray tracer in the migration operator \cite[]{tsvankin.2001}.


\inputdir{XFig}
\plot{AniImg}{width=\textwidth}{{\cite{alkhalifah:1405}} showed (a) Model consisting of horizontal reflectors with segments having dips ranging from 30 to 90 degrees on both sides.
(b) Depth migration of the reflector model in (a). The medium is Taylor sandstone, with $\delta = -0.035$ and $\epsilon = 0.11$, where the axis of symmetry is vertical. The vertical velocity is $v(z)= 2000 + 0.6z$ m/s.}
 
Usually, the extension of migration algorithms to anisotropy was limited to P wave, assuming that S wave velocity in the asymmetry axis direction is zero. This was done to eliminate shear waves, and model P waves that are not interfered by S waves in anisotropic media. For example, \cite{alkhalifah:623} and \cite{alkhalifah:1239} exploited acoustic wave equations by setting S velocity at symmetry axis to zero. However, this leads to artifacts such as creating extreme anisotropy that makes possible orthogonality of ray and wavefront normal directions, identical polarization of P- and S-waves propagating along certain rays, and in some degenerate cases, P-wave singularities~\cite[]{grechka:576}. \cite{grechka:576} also suggested: from the point of view of arrival identification and separation, it may be better to use elastic VTI media because shear-waves propagating there have distinctly different polarizations and moveouts from those of P-waves. Therefore, in anisotropic media, it is more important to do modeling and migration with elastic wave equation.

\addcontentsline{toc}{subsection}{Anisotropic wavefield separation}
\subsection{Anisotropic wavefield separation}

Anisotropic elastic wavefield imaging also consists of two components: anisotropic elastic wavefield reconstruction and an imaging condition. As the reconstruction of elastic isotropic wavefields, the reconstruction of elastic anisotropic wavefield can be done without difficulty using more complex form of wave equations. The imaging condition for anisotropic elastic wavefield can be extended from isotropic imaging condition. 

In anisotropic media, Helmholtz decomposition does not separate P and S wave mode, because the scalar and vector potentials are not P and S mode, since ray direction (group velocity vector), phase velocity vectors, and polarization directions are not necessarily orthogonal or aligned. Therefore, separation of anisotropic wavefields is harder than isotropic wavefields. \cite{GEO55-07-09140919} showed that polarization vectors may be used to separate \textit q-P and \textit q-S wavefields. \rFg{qP} showed Fourier domain P mode passer for homogeneous isotropic and anisotropic medium.

\inputdir{XFig}
\plot{qP}{width=\textwidth}{Left: Fourier-domain plot of the operator for passing only P-waves in isotropic media. Right: Fourier domain plot of the operator for passing only P-waves in the anisotropic medium {\cite[]{GEO55-07-09140919}}.}


%\cite[]{SEG-1990-0124}
%\cite[]{SEG-1989-1308}
%\cite[]{SEG-1989-1316}
%\cite[]{SEG-1988-S1.3}
%\cite[]{SEG-1988-S14.7}
%\cite[]{SEG-1985-BHG3.1}
%\cite[]{SEG-1985-BHG3.4}
%\cite[]{SEG-1984-BHG2.8}
%\cite[]{EAE-1999-P061}
%\cite[]{SEG-2000-21112114}
%\cite[]{SEG-2000-00250028}
%\cite[]{SEG-2000-12251228}
%\cite[]{GPR48-04-06970722}
%\cite[]{AE-2000-L0003}
%\cite[]{EAE-2000-P0133}
%\cite[]{EAE-2002-P002}
%\cite[]{GPR51-03-02330245}
%\cite[]{EAE-1991-0516}
%\cite[]{GPR35-07-08030814}
%\cite[]{EAE-1992-0576}
%\cite[]{GEO55-07-09140919}
%\cite[]{SEG-1989-0977}
%
\addcontentsline{toc}{section}{Accomplished research and preliminary results}
\section{Accomplished research and preliminary results}

\addcontentsline{toc}{subsection}{Imaging condition}
\subsection{Imaging condition}


% -------------------------------------------------------------------------------------------------------------------------------

For vector elastic wavefields, the cross-correlation imaging condition has to be implemented on combinations of components separated from the displacement field. The problem with this type of imaging condition is that the source and receiver wavefields contain a mix of P and S wave modes which hampers interpretation of migrated images. An alternative to this type of imaging is to perform wavefield separation in P and S potentials \emph{after} wavefield reconstruction in the imaging volume, but prior to the imaging condition and then cross-correlate pure P and S modes from the source and receiver wavefields, as suggested by \cite{GEO55-07-09140919}.

I investigated this form of imaging using potentials decomposition after extrapolation and demonstrate that it can, indeed, be used to construct images corresponding to combinations of pure P and S modes. The images obtained by this procedure suffer from less  cross-talk between wave modes that are not separated on vertical and horizontal components of the reconstructed wavefields.
% -------------------------------------------------------------------------------------------------------------------------------
\addcontentsline{toc}{subsubsection}{Scalar imaging condition}
\subsubsection{Scalar imaging condition}
Assuming single scattering in the Earth (Born approximation), a conventional imaging procedure consists of two components: wavefield extrapolation and imaging. Wavefield extrapolation is used to reconstruct in the imaging volume the seismic wavefield using the recorded data on the acquisition surface as boundary condition, and imaging is used to extract reflectivity information from the extrapolated source and receiver wavefields.

Assuming scalar recorded data, wavefield extrapolation using a scalar wave-equation reconstructs scalar source and receiver wavefields at every location in the subsurface, $\US{}$ and $\UR{}$. One-way or two-way wavefield extrapolation by differential or integral methods can be used. Using the scalar extrapolated wavefields $u_s$ and $u_r$, a conventional imaging condition \cite[]{Claerbout.iei} can be implemented as cross-correlation at zero-lag time: 
\beq \label{eqn:CIC}
\IM{} = \int \US{} \UR{} dt \;.
\eeq
$\IM{}$ denotes a scalar image obtained from scalar wavefields $\US{}$ and $\UR{}$. \rEq{CIC} generalizes trivially to 3D.
% -------------------------------------------------------------------------------------------------------------------------------
\addcontentsline{toc}{subsubsection}{Displacement imaging condition}
\subsubsection{Displacement imaging condition}

Assuming vector data, wavefield extrapolation using a vector wave-equation reconstructs source and receiver wavefields at every location in the subsurface, $\uu_s\ofx$ and $\uu_r\ofx$. Here, $\uu_s$ and $\uu_r$ represent the source and receiver displacement fields recorded by multicomponent geophones. Using the vector extrapolated wavefields ${\uu}_s=\{{u_s}_x,{u_s}_z\}$ and ${\uu}_r=\{{u_r}_x,{u_r}_z\}$, an imaging condition can be formulated as a straightforward extension of \ren{CIC}. In this case, I can cross-correlate all combinations of components of the source and receiver wavefields. Such an imaging condition can be formulated mathematically as:
\beqa
\label{eqn:EICzz} \IM{zz} &=& \int \US{z} \UR{z} dt \;, \\
\label{eqn:EICxz} \IM{zx} &=& \int \US{z} \UR{x} dt \;, \\
\label{eqn:EICzx} \IM{xz} &=& \int \US{x} \UR{z} dt \;, \\
\label{eqn:EICxx} \IM{xx} &=& \int \US{x} \UR{x} dt \;.
\eeqa
$\IM{zz}$ represents the image component produced by cross-correlation of the $z$ components of the source and receiver wavefields, $\IM{zx}$ represents the image component produced by cross-correlation of the $z$ component of the source wavefield with the $x$ component of the receiver wavefield, etc. \rEqs{EICzz}-\ren{EICxx} generalize trivially to 3D.
The main drawback of applying this type of imaging condition is that the wavefield used for imaging contain a combination of P and S wave modes. Those wavefield interfere with one-another in the imaging condition, since the P and S components are not separated in the  extrapolated wavefield. The cross-talk between various components of  the wavefield creates artifacts and makes it difficult to interpret  the images in terms of pure modes, e.g. PP or PS reflections. This  situation is similar to the case of imaging with acoustic data  contaminated by multiples or other types of coherent noise.   

%\inputdir{/work/cwp/jyan/rtmig/marm2one}
\inputdir{marm2one}
% ------------------------------------------------------------
\multiplot{2}{vp,rx}{width=0.45\textwidth}
{P- and S-wave velocity models (a) and density model (b) used for  isotropic elastic wavefield modeling.}
% ------------------------------------------------------------
% ------------------------------------------------------------
\multiplot*{4}{de1,de2,df1,df2}{width=0.2\textwidth}
{Elastic data simulated in model \rfn{vp}-\rfn{rx}: vertical component (a), horizontal component (b), scalar potential (c) and vector potential (d) of the elastic wavefield. Both vertical and horizontal components, panels (a) and (b), contain a mix of P and S modes, as seen by comparison with panels (c) and (d).}
% ------------------------------------------------------------

For illustration, consider the images obtained for the model depicted  in \rFgs{vp}-\rfn{rx}. \rFg{vp} depicts the P-wave velocity (smooth function between $1.6-3.2$~km/s), and \rfg{rx} shows the density  (variable between $1-2$~g/cm$^3$). The S-wave velocity is a scaled  version of the P-wave velocity with $v_p/v_s=2$. Data are modeled and  migrated in the smooth velocity background to avoid back-scattering;  the modeling density is shown in \rfg{rx}, and the migration density  is a smooth version of it.    Elastic data, \rfgs{de1}-\rfn{de2}, are simulated using a space-time staggered-grid finite-difference solution to the isotropic elastic
wave-equation \cite[]{GEO49-11-19331942,GEO51-04-08890901,GEO52-09-12111228,GEO53-06-07500759}.
I model data by simulating a source located at position $x=6.75$~km and  $z=0.5$~km. Since I am using an explosive source, the simulated  wavefield is represented by P-wave incident energy and the receiver  wavefield is represented by a combination of P- and S-wave reflected  energy. The data contain a mixture of P and S modes, as can be seen by  comparing the displacement field components in \rfgs{de1}-\rfn{de2} with the separated P and S components in \rfgs{df1}-\rfn{df2}.

If imaging data shown in \rfgs{de1}-\rfn{de2} using the imaging  condition \ren{EICzz}-\ren{EICxx}, I obtain the images depicted in  \rfgs{ieall0}-\rfn{ieall3}. The imaging condition does not separate P  and S waves reflectivity, therefore the various panels contain energy from both wave modes.

% --------------------------------------------------------------------------------------------------------------------------------------
\addcontentsline{toc}{subsubsection}{Potential imaging condition}
\subsubsection{Potential imaging condition}
An alternative to the elastic imaging condition  \ren{EICzz}-\ren{EICxx} is to separate the extrapolated wavefield in P and S potentials \emph{after} extrapolation and image using cross-correlations of vector and scalar potentials \cite[]{GEO55-07-09140919}.

Separation of scalar and vector potentials can be achieved by Helmholtz decomposition, which is applicable to any vector field $\uu$:
\beq \label{eqn:helmholtz}
\uu = \GRAD{\phi} + \CURL{\psi} \;.
\eeq
For isotropic elastic wavefield, \req{helmholtz} is not used directly in practice, but the scalar and vector components are obtained indirectly by the application of the $\DIV{}$ and $\CURL{}$ operators to the extrapolated elastic wavefield $\uu$: 
\beqa \label{eqn:PS}
P        &=& \DIV {\uu} = \LAPL{\phi} \;, \\
S	 &=& \CURL{\uu} = -\LAPL{\psi} \;.
\eeqa
For isotropic elastic fields far from the source, quantities $\textbf P$ and $\textbf S$ describe P and S components, respectively \cite[]{akirichards.2002}. 

Using the separated scalar and vector components, I can formulate an imaging condition that combines various incident and reflected wavefield components. In 2D, the vector potential $\S{}$ has only one component. The imaging condition for potentials can be formulated mathematically as:
\beqa 
\label{eqn:EICpp} \IM{PP} &=& \int \P{s} \P{r} dt \;, \\ 
\label{eqn:EICps} \IM{PS} &=& \int \P{s} \S{r} dt \;, \\ 
\label{eqn:EICsp} \IM{SP} &=& \int \S{s} \P{r} dt \;, \\
\label{eqn:EICss} \IM{SS} &=& \int \S{s} \S{r} dt \;,
\eeqa
where $\P{s}$ and $\P{r}$ represent source and receiver P-wave components, and $\S{s}$ and $\S{r}$ represent source and receiver S-wave components. The formed images correspond to various combinations of incident P or S and reflected P or S waves.



For illustration, consider the images obtained for imaging condition \reqs{EICpp}-\ren{EICss} applied to the data used for the preceding  example. Given the explosive source used in our simulation, the source  wavefield contains mostly P-wave energy, while the receiver wavefield contains P- and S-wave energy. Helmholtz decomposition after extrapolation but prior to imaging isolates P and S wavefield components. Therefore, migration produces images of reflectivity corresponding to PP and PS reflections, \rfgs{jeall0}-\rfn{jeall1}, but not reflectivity corresponding to SP or SS reflections, \rfgs{jeall2}-\rfn{jeall3}. As expected, the illumination regions are different between PP and PS images, due to different illumination angles of the two propagation modes for the given acquisition geometry. The PS image, \rfg{jeall1}, also shows the usual polarity reversal for positive and negative angles of incidence measured relative to the reflector normal.



% ------------------------------------------------------------
\multiplot{4}{ieall0,ieall1,ieall2,ieall3}{width=0.45\textwidth}
{Images produced by the application of the displacement imaging
condition \ren{EICzz}-\ren{EICxx}. The image corresponds to one shot
at position $x=6.75$~km. Receivers are located at all locations at
$z=0.55$~km.}
% ------------------------------------------------------------

% ------------------------------------------------------------
\multiplot{4}{jeall0,jeall1,jeall2,jeall3}{width=0.45\textwidth}
{Images produced by the application of the potentials imaging
condition \ren{EICpp}-\ren{EICss}. The image corresponds to one shot
at position $x=7$~km.  Receivers are located at all locations at
$z=0.55$~km.}


% ------------------------------------------------------------------------------------------------------------------------


\addcontentsline{toc}{subsubsection}{Artifacts}
\subsubsection{Artifacts}
Images obtained by potential imaging condition are not free from artifacts, since back-propagation of vector fields from the receivers triggers sources of artificial wave modes that are not part of the recorded wavefield that do not exist in the original  wavefield. 
For example, data corresponding to P-wave or S-wave reflections injected in the model as boundary displacement sources generate both P and S propagating modes. Those artificial events in the receiver wavefield interfere with events in the source wavefield  to generate artifacts in the images. This phenomenon is independent of the complexity of the model or of the proximity of free surface. \rFg{artifact1}-\rfn{artifact3} show the schematic how both reversed P and S data generate both P and S wave modes when injected. Even after wavefield separation, we see that P waves are generated by both P and S recordings, and S waves are generated by both P and S recordings as well, shown in \rfg{artifact4}-\rfn{artifact5}.

An example corresponding to artifacts caused by artificial P waves generated by back-propagating S waves in the receiver wavefield are visible in \rfg{jeall0} around coordinates $x=7.25$~km and $z=0.6$~km. Although not widely reported in the literature, this phenomenon characterizes all elastic reverse-time migration methods, no matter of imaging condition. For example, the same artifacts seen in \rfgs{jeall0}-\rfn{jeall1} are also seen in  \rfgs{ieall0}-\rfn{ieall3}, although in this case the artifacts are more confusing, since their origin is harder to discern among the imaged reflections of different wave modes.

One of the few articles that mentioned this type of artifacts is \cite{sun:286}, who referred to this artifacts as the grid-edge effects that P-S conversion occurs at the edges when inserting the seismograms during downward extrapolation. And they suggested reducing these artifacts by tapering the edges of the data and recording wider area than required and then abandon the near-edge data after P-S wave separation. However such artifacts may occur at small offsets if reflection moveout is large. If we look closely at \rfg{artifact1}-\rfn{artifact3}, we will see that at larger offsets the reconstructed P and S modes have larger separation in space, and smaller separation at smaller offsets, which explains why the artifacts are more obvious at larger offsets. This also means tapering large offset data would not eliminate these artifacts. When moveout in data is large, even at small offsets, the reconstructed wavefield will have both P and S modes no matter PP or PS reflection was used to trigger the wavefield.

\inputdir{XFig}
\multiplot{3}{artifact1,artifact2,artifact3}{width=0.28\textwidth}{Huygens' principle can be used to explain how elastic displacement data used as displacement sources can trigger both P and S wave modes.}
\multiplot{3}{artifact4,artifact5,artifact6}{width=0.28\textwidth}{Even after separation of wave modes, (a) both PP and PS reflection will generate P mode in reconstructed receiver P potential wavefield, (b) both PP and PS reflection will generate S mode in reconstructed receiver S potential  wavefield, and (c) only true wave mode has consistent polarity as the recorded elastic data, and the fake mode does not have consistent polarity as the recorded elastic data.}

% -------------------------------------------------------------------------------------------------------------------------






\addcontentsline{toc}{subsection}{Angle domain common image gathers}
\subsection{Angle domain common image gathers}

\cite{sava:2460} designed the following formulas to compute converted wave angle gathers. \rFg{cwang} shows the schematic and some of the notations used in these formulas.

\plot{cwang}{width=\textwidth}{Wave vectors for converted wave at a common image point location. {\bf $P_s$}, {\bf $P_r$}, {\bf $P_m$} and {\bf $P_h$} are the ray parameters for the source vector, receiver vector, midpoint vector and offset vector. The length of the incidence and reflection wave vectors are inversely proportional to the wave velocity. Vector ${\bf n}$ is the normal of the reflector. {\bf $P_m$} is the summation of {\bf $P_s$} and {\bf $P_r$}, and {\bf $P_s$} is the difference of {\bf $P_s$} and {\bf $P_r$}.}

From geometry, we have:
\def\p{{\mathbf p}}
\def\r{{\mathbf r}}
\def\s{{\mathbf s}}
\def\h{{\mathbf h}}
\def\m{{\mathbf m}}
\beqa \label{eqn:PmPh}
\p_\m &=& \p_\r+\p_\s \;,\\
\p_\h &=& \p_\r-\p_\s \;.
\eeqa
where $\p$ is the ray parameter, and the subscripts $m$,$h$,$s$,$r$ are midpoint, offset, source and receiver, respectively.

The angles have relationship:
\beqa \label{eqn:theta}
2\theta &=& \theta_s+\theta_r \;,\\
2\delta &=& \theta_s-\theta_r \;.
\eeqa
where $2\theta$ is the sum of incidence and reflection rays, and $2\delta$ is the difference of incidence and reflection rays.

From trigonometry, we have:
\beqa 
\label{eqn:PmPh-a}
4 |\p_\h|^2 &=& |\p_\s|^2 + |\p_\r|^2 - 2 |\p_\s| |\p_\r| \cos(2 \theta) \;,
\\
\label{eqn:PmPh-b}
4 |\p_\m|^2 &=& |\p_\s|^2 + |\p_\r|^2 + 2 |\p_\s| |\p_\r| \cos(2 \theta) \;.
\eeqa 

The angle $\theta$ can be solved for:
\def\k{\mathbf k}
\beq
\tan^2\theta = \frac
{ \lp 1+\gamma \rp |\k_\h|^2 - \lp 1-\gamma \rp |\k_\m|^2 }
{ \lp 1+\gamma \rp |\k_\m|^2 - \lp 1-\gamma \rp |\k_\h|^2 }
\eeq
where $\k=\p/\omega$, $\gamma$ is the velocity of the incidence and reflection ray. For PP reflections ($\gamma=1$), this expression reduces to
\beq
\tan^2\theta = \frac
{|\k_\h|^2  }
{|\k_\m|^2  }
\eeq

To avoid computing vertical offset $h_z$, the above expression transforms to:
\def\ahx{a_{hx}}
\def\bmx{b_{mx}}
\def\amx{a_{mx}}
\def\bhx{b_{hx}}

\def\kmx{k_{mx}}
\def\khx{k_{hx}}
\def\kmz{k_{mz}}
\def\khz{k_{hz}}

\beq
\tan\theta = \frac
{ \lp 1+\gamma\rp \lp\ahx+\bmx\rp }
{ 2\gamma\kmz+\sqrt{4\gamma^2\kmz^2+\lp\gamma^2-1\rp \lp\ahx+\bmx\rp \lp\amx+\bhx\rp} }
\eeq
where 
$\ahx=\lp 1+\gamma\rp \khx$,
$\amx=\lp 1+\gamma\rp \kmx$,
$\bhx=\lp 1-\gamma\rp \khx$, and 
$\bmx=\lp 1-\gamma\rp \kmx$.

\cite{sava:2460} in \rfg{cwgat} showed model and extracted angle gathers for five reflectors.
%\multiplot{2}{experimentC,psang-h}{height=0.2\textheight}{converted wave angle gathers (Sava and Fomel, 2006).}
\plot{cwgat}{width=\textwidth}{(a) A common source over reflectors dipping at 0$^\circ$, 15$^\circ$, 30$^\circ$, 45$^\circ$ and 60$^\circ$. The vertical dashed line shows a CIG location. Lines with arrows away from the reflectors shows the reflected P to S conversion. (b) Converted wave angle gather obtained from algorithm described by {\cite{sava:2460}}, notice that converted wave angles are always smaller than incidence angle (in this case, the dip of the reflector) except for normal incidence. }


%\inputdir{/work/cwp/jyan/rtmig/marm2one}
%\plot{rx}{width=\textwidth}{rx}

%\inputdir{/work/cwp/jyan/rtmig/marm2all}
\inputdir{marm2all}
\multiplot{2}{je-0700-PPcig,je-0700-PScig}{height=.5\textheight}{Horizontal lags for PP and PS reflections. The source is at $x=6.75$ km, and CIG is at $x=6.5$ km.}
\multiplot{2}{je-0700-PPang,je-0700-PSang}{height=.5\textheight}{PP and PS angle gather decomposed from the horizontal lag gathers in \rfg{je-0700-PPcig} and \rfn{je-0700-PScig}.}
\multiplot{2}{PPcig,PScig}{height=.5\textheight}{Horizontal lags for PP and PS reflections. The sources are at $x=5.5$ to $7.5$  km, and CIG is at $x=6.5$ km.}
\multiplot{2}{PPcig-ang,PScig-ang}{height=.5\textheight}{PP and PS angle gather decomposed from the horizontal lag gathers in \rfg{PPcig} and \rfn{PScig}.}

Angle domain common image gathers (ADCIGs) can be used for velocity analysis, amplitude versus angle (AVA) studies, etc. 

I used the above algorithm, and computed angle gathers for the Marmousi II example. \rFg{je-0700-PPcig}--\rfn{je-0700-PScig} and \rfg{je-0700-PPang}--\rfn{je-0700-PSang} show the PP and PS horizontal lags and angle gathers for the common image gather (CIG) location in the middle of the reflectivity model, for a single source at $x=6.75$ km. PP and PS horizontal lags are lines dipping at angles related to the incidence angle at the CIG location. PP angles are larger than PS angles at all reflectors, as was shown on the simple synthetic example \rfg{cwgat}. We can observe some artifacts on the angle gathers especially obvious in PS gathers. This is the same conversion artifact showing on the angle gathers.

\rFg{PPcig}--\rfn{PScig} and \rfg{PPzoo} and \rfn{PSzoo} are the PP and PS horizontal lags and angle gathers for the same CIG location, for sources from $x=5.5$ to $7.5$ km. 
The horizontal lags are focused points at zero lag, showing the conventional imaging condition, which takes cross-correlation of source and receiver wavefield at zero lag in space. The lags at zero lag is just the image at the particular CIG location. The PP and PS gathers for many sources are flat gathers, showing the migration were done with correct migration velocity. The angle gathers could potentially be used to do MEWVA and AVA studies.


\addcontentsline{toc}{section}{Proposed work}
\section{Proposed work}

\addcontentsline{toc}{subsection}{Removing conversion artifacts on the surface}
\subsection{Removing conversion artifacts on the surface}
The vector displacement used as sources to propagate wavefield backward in time generate mode conversion artifacts, which can be best understood with a flat layer model. 
\rFg{jeall0flat} and \rfg{jeall1flat} show the PP and PS images obtained using the potential imaging condition for a flat layer model. PS image has stronger artifact than PP image. This is because in receiver wavefield, the P component generated by PS reflection is deeper in depth than that generated by PP reflection, and therefore is weaker in amplitude.
This conversion artifacts make part of PP and PS images look like over-migrated and under-migrated. 

As we have seen from the elastic images \rFg{jeall0flat} and \rfg{jeall1flat}, especially in PS image, artifact can be very strong, which sometimes obscure the real images, and it is of primary importance to remove these artifacts.

Several solutions are feasible to this problem:
\begin{itemize}
\item Utilize angle gathers. \rFg{jePPangALL} and \rfg{jePSangALL} show the angle gathers for images \rFg{jeall0flat} and \rfg{jeall1flat}. The angle image show that for real reflections, angle gathers are flat, but for artifacts, they are not flat. The angle gathers are flat not because the reflectors are flat, but would be flat for all correctly done migration. A flat model was used here to save computation cost. This property of angle gathers can be used to remove the artifact by dip filtering out the non-flat noise in angle gathers and stack over angles to obtain the image for the particular CIG location.
This solution, however, is very expensive, and is the post-migration cosmetic solution to the elastic potential imaging procedure.


\inputdir{flatlayer}
\multiplot{2}{jeall0,jeall1}{width=.45\textwidth}
{PP and PS images for a horizontal reflector at the depth of $z=0.8$
km. Both PP and PS images show reflectors at the correct depth, but
both have conversion artifacts at different depths.}
\multiplot{2}{jePPangALL,jePSangALL}{width=.45\textwidth}
{PP and PS angle gathers for images \rfg{flatlayer-jeall0}
and \rfn{flatlayer-jeall1}. Sources are at $x=0.5$ to $1.5$ km and
$z=0$, CIG location is at $x=1$ km.}

\item Separate wave modes on the surface and use identified P and S modes to image PP and PS reflections. There are many ways to do this, basically, deterministic and stochastic approaches. The deterministic approach need extrapolation of wavefield to some depth and use Helmholtz decomposition to separate P and S, while stochastic approach rotate recorded vertical and horizontal displacement by the polarization angle~\cite[]{SEG-1989-1308}.
This way is not preferred because separating wave modes on the surface
is always followed by scalar imaging, and elastic data are supposed to
be done by vector imaging.


\item Kill unwanted modes at the conversion surface. \rFg{artifact1}-\rfn{artifact3} showed how two modes of reconstructed wavefield are formed. \rFg{artifact4} and \rfn{artifact5} show that after separation of wave modes with Helmholtz decomposition, both PP reflection and PS reflection will create unwanted wave modes. \rFg{artifact6} shows that the unwanted mode has inconsistent polarity with the recorded reflection. This observation can be used to remove the fake mode during modeling. Notice that the fake mode is only initiated at injection surface, therefore, the polarity only has to be analyzed and taken care of at the surface but not at the whole $\{x,z\}$ space, which means the additional polarity handling would not taken too much extra computational time.


\end{itemize}




\addcontentsline{toc}{subsection}{Anisotropic wavefield reconstruction}
\subsection{Anisotropic wavefield reconstruction}
Modeling with full elastic wave equation for anisotropic media is more complex than that for isotropic media, because more stiffness elements are required. For example, for isotropic media, only two elastic parameters are required; for VTI media, five are required for 3D and four are required for 2D; for TTI media, five are required for 2D. However, since the elastic wave equation for all 2D media with different types of anisotropy holds the same, the implementation of anisotropic wavefield modeling is conceptually easy.

\addcontentsline{toc}{subsection}{Anisotropic wavefield separation}
\subsection{Anisotropic wavefield separation}
As I have described in the literature review section, Helmholtz decomposition does not separate anisotropic wavefield into qP and qS modes, because in elastic wave equation P and S mode are coupled. One has to use wave polarization information to do separation. For example, \cite[]{GEO55-07-09140919} did separation of qP and qS for homogeneous VTI media (\rfg{qP}). The separation operator works well in homogeneous 2D medium. So, is there a better way to do anisotropic wavefield separation for heterogeneous medium and 3D medium? This is a open research topic.

\addcontentsline{toc}{subsection}{Imaging condition for anisotropic wavefield}
\subsection{Imaging condition for anisotropic wavefield}
What imaging condition should be applied to elastic anisotropic wavefield imaging? This question directly leads to the interpretation of the anisotropic images. 
Anisotropic images can be obtained by cross-correlating source and receiver wavefield potentials. However, because of group vector, phase vector and polarization vector in anisotropic media do not coincide with each other, opposed to the isotropic case where all three coincide, the physical meaning of elastic anisotropic images are not clear. Understating the physical meanings of the images would help further interpreting them.

\addcontentsline{toc}{subsection}{Anisotropic ADCIG}
\subsection{Anisotropic ADCIG}
Angle gathers of anisotropic images have obscure meanings, because the incidence angle and reflection angles are phase vector angles or group vector angles are not clear. \cite{biondi:S81} obtained angle-domain common-image gathers for acoustic images from anisotropic migration. His method can be used to study the angle gather of the PP reflection in the elastic images. Since previously, no research has been done on anisotropic converted wave imaging, no anisotropic converted wave angle gather decomposition methods were proposed. The anisotropic converted wave angle gather decomposition would add an interesting complement to the acoustic anisotropic ADCIG study.


\addcontentsline{toc}{subsection}{Imaging errors with wrong models}
\subsection{Imaging errors with wrong models}
With anisotropic elastic migration tools, one would be interested to study the imaging errors with wrong models and wrong migration algorithm, i.e., what errors would we get if we use isotropic elastic migration algorithm to models that are really anisotropic. This study is essential for the validation of new migration algorithm.

\addcontentsline{toc}{subsection}{Field data application}
\subsection{Field data application}
If all the theoretical work proposed above is finished in time, appropriate field data will be used to test the robustness of algorithm.


\addcontentsline{toc}{section}{Timeline}
\section{Research plan}
Here, I propose the following approximate research timeline:
\begin{itemize}
\item Artifacts and anisotropic finite difference: Oct. 2007--Dec. 2007
\item Anisotropic wavefield separation: Jan. 2008--June 2008
\item Imaging condition and imaging errors: July 2008--Sept. 2008
\item Anisotropic ADCIG: Oct. 2008--Jan. 2009
\item Field data: Feb. 2009-June. 2009
\item Thesis writing and wrap up: July 2009-Aug. 2009
\item Thesis defense: Aug. 2009
\end{itemize}



\newpage
\addcontentsline{toc}{section}{References}
% ------------------------------------------------------------
\bibliographystyle{seg}
%\bibliography{MISC}

\bibliography{Jia,SEG2005,MISC}
